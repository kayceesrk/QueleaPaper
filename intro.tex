\section{Introduction}

Many real-world web services --- such as those built and maintained by Amazon,
Facebook, Google, Twitter, etc. --- replicate application state and logic
across multiple \emph{replicas} within and across data centers. Replication is
intended not only to improve application throughput and reduce user-perceived
latency, but also to tolerate partial failures without compromising overall
service availability. Traditionally programmers have relied on \emph{strong
consistency} guarantees such as linerarizability~\cite{Herlihy1990} or
serializability~\cite{Serializability} in order to build correct applications.
While strong consistency is an easily stated property, it masks the reality
underlying large-scale distributed systems with respect to non-uniform latency,
availability and network partitions~\cite{Brewer2000,Gilbert2002}. Indeed,
modern web services, which aim to provide an "always-on" experience,
overwhelmingly favour availability and partition tolerance over strong
consistency. To this end, several \emph{weak consistency} models such as
eventual consistency, causal consistency, session guarantees, and timeline
consistency have been proposed.

Under weak consistency, the developer needs to be aware of concurrent
conflicting updates, and has to pay careful attention to avoid unwanted
inconsistencies (e.g., negative balances in a bank account, or having an item
appear in a shopping cart after it has been removed~\cite{Dynamo}). Oftentimes,
the inconsistency leaks from the application and is witnessed by the user.
Ultimately, the developer must decide the consistency level appropriate for a
particular operation; this is understandably an error-prone process requiring
intricate knowledge of both the application as well as the semantics and
implementation of the underlying data store, which typically have only informal
descriptions. Nonetheless, picking the correct consistency level is critical
not only for correctness but also for scalability of the application. While
choosing a weaker consistency level than required may introduce program errors
and anomalies, choosing a stronger one than necessary can negatively impact
program scalability.

Weak consistency also hinders compositional reasoning about programs.  While an
application might be naturally expressed in terms of well-understood and
expressive data types such as maps, trees, queues, or graphs, geo-distributed
stores typically only provide a minimal set of data types with in-built
conflict resolution strategies such as last-writer-wins (LWW) registers,
counters, and sets~\cite{Cassandra,DynamoDB}.  Furthermore, while traditional
database systems enable composability through transactions, geo-distributed
stores typically lack unrestricted serializable transactional access to the
data. Working in this environment thus requires application state to be
suitably coerced to function using only the capabilities of the store.

To address these issues, we describe \name, a declarative programming model and
implementation for eventually consistent geo-distributed data stores. The key
novelty of \name is an expressive \emph{contract language} to declare and
verify fine-grained application-level consistency properties. The programmer
uses the contract language to axiomatically specify the set of legal executions
allowed over the replicated data type. Contracts are constructed using
primitive consistency relations such as \emph{visibility} and \emph{session
order} along with standard logical and relational operators. A \emph{contract
enforcement system} statically maps operations over the datatype to a
particular consistency level available on the store, and provably validates the
correctness of the mapping. The paper makes the following contributions:

\begin{itemize}
\setlength{\itemsep}{2pt}
\item We introduce \name, a shallow extension of Haskell that supports the
	description and validation of replicated data types found on eventually
	consistent stores. Contracts are used to specify fine-grained
	application-level consistency properties, and are statically analyzed to
	assign the most efficient and sound store consistency level to the
	corresponding operation.
\item \name supports coordination-free transactions over arbitrary datatypes.
	We extend our contract language to express fine-grained transaction isolation
	guarantees, and utilize the contract enforcement system to automatically
	assign the correct isolation level for a transaction.
\item We provide metatheory that certifies the soundness of our contract
	enforcement system, and ensures that an operation is only executed if the
	required conditions on consistency are met.
\item An implementation of \name as a transparent shim layer over
	Cassandra~\cite{Cassandra}, a well-known general-purpose data store.
	Experimental evaluation over a set of real-world applications, including a
	Twitter-like micro-blogging site and an eBay-like auction site illustrates
	the practicality of our approach.
\end{itemize}

The rest of the paper is organized as follows. The next section describes the
system model.  We describe the challenges in programming with eventually
consistent data stores, and introduces \name\ contracts as a proposed solution
to overcome these issues in \S~\ref{sec:motivation}. \S~\ref{sec:lang} provides
more details the contract language, and its mapping to the store consistency
levels, along with metatheory for certifying the correctness of the mapping.
\S~\ref{sec:txns} introduces transaction contracts and classification.
\S~\ref{sec:impl} describes the implementation and provides details about the
optimizations needed to make the system practical. \S~\ref{sec:results}
discusses experimental evaluation. \S~\ref{sec:related} and~\ref{sec:concl}
present related work and conclusions.
