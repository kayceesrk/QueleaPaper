%% ----------------------------------------------------------------------------
%% Summarization details
%% ----------------------------------------------------------------------------

Suppose we have a set $O$ of effects on a object, which when summarized yields
a new set of effects $N$. We first instantiate a summarization marker $s$, and
similar to transaction marker, we do not insert it into the store immediately.
We insert the new effects in $N$ with strong consistency, including $s$ as a
dependence. Since $s$ is not yet in the store, effects from $N$ are not made
visible to the clients. Then we insert $s$ with strong consistency, including
$O$ as dependence, and finally delete the rows corresponding to $O$. When the
shim layer node witnesses the summarization marker $s$ on the object, it
removes the effects from $O$ from its cache. But due to strong consistency of
insertion of $N$ and $s$, the new effects $N$ are also observed. Thus, the
clients witness the observably equivalent new set of effects $N$, and not the
old effects $O$. This ensures atomicity of summarization in the backing store.

%% ----------------------------------------------------------------------------
%% Example for MAV
%% ----------------------------------------------------------------------------

\if{0}
RC transactions typically perform writes on multiple objects, and want to
ensure that the write is atomic. For example, consider the process of adding a
new user in the micrblog example. The user information is maintained in the
\cf{User} table, with \cf{userID} as the primary key and \cf{userName} and
\cf{password} as values (written \cf{User :: UserID -> (UserName,Password)}. In
order to lookup user id from user name, we need to maintain a secondary index
(\cf{UserIdx :: UserName -> UserID}). Adding a new user involves insertions
into both the tables, and we would like the insertions to be atomic:

\begin{codehaskell}
addUser userID userName password = do
	x <- dollar(classify $true$)
	atomically x dollar do
	  User.addUser userName userID
		UserIdx.addUserIdx userID (userName, password)
\end{codehaskell}

MAV transactions are typically used to preserve foreign key constraints, and
integrity of secondary indexes. For example, consider the code for getting
password for a user name:

\begin{codehaskell}
getPassword userName = do
  x <- dollar(classify $\psi$)
	atomically x dollar do
	  mUid <- UserIdx.getUserID userName
		case mUid of
		  Nothing -> error "User name does not exist"
			Just uid -> User.getPassword uid
\end{codehaskell}

If \cf{getUserID} successfully returns a user id, indicating that the user name
exists in \cf{UserIdx} table, we expect to see the user in the \cf{User} table
as well. For this purpose, we write the contract $\psi$ as:

\begin{smathpar}
\begin{array}{l}
\forall (a:{\sf GetUserID}), (b:{\sf GetPassword}), (c:{\sf AddUser}), \\
~~(d:{\sf AddUserIdx}). ~\txn{a,b}{c,d} ~\wedge~ \vis{d}{a} ~\wedge~ \sameobj{b}{c} \\
\qquad \qquad \qquad \qquad \wedge~ \so{a}{b} \Rightarrow \vis{c}{b}
\end{array}
\end{smathpar}

\noindent which assigns MAV transaction isolation level to \cf{getPassword}
transaction.
\fi
