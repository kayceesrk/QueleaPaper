%
% LaTeX template for prepartion of submissions to PLDI'15
%
% Requires sigplanconf style file provided on PLDI'15 web site
%
\documentclass[pldi]{sigplanconf}

%
% the following standard packages may be helpful, but are not required
%
\usepackage{SIunits}            % typset units correctly
\usepackage{courier}            % standard fixed width font
\usepackage[scaled]{helvet} % see www.ctan.org/get/macros/latex/required/psnfss/psnfss2e.pdf
\usepackage{url}                  % format URLs
\usepackage{listings}          % format code
\usepackage{enumitem}      % adjust spacing in enums
\usepackage[colorlinks=true,allcolors=blue,breaklinks,draft=false]{hyperref}   % hyperlinks, including DOIs and URLs in bibliography
% known bug: http://tex.stackexchange.com/questions/1522/pdfendlink-ended-up-in-different-nesting-level-than-pdfstartlink
\newcommand{\doi}[1]{doi:~\href{http://dx.doi.org/#1}{\Hurl{#1}}}   % print a hyperlinked DOI



\begin{document}

%
% any author declaration will be ignored  when using 'plid' option (for double blind review)
%

\title{Declarative Programming over Eventually Consistent Databases }

\maketitle
\begin{abstract}
User-facing online services utilize geo-distributed data stores to minimize
latency and tolerate partial failures, with the intention to provide a fast,
always-on experience. However, geo-distribution does not come for free;
application developers have to contend with (1) the lack of abstractions to
composably construct high-level replicated data types and (2) weak consistency
behaviors, neccesitating complex application logic and invariably exposing
inconsistencies to the user. Commercial distributed data stores and sevaral
academic proposals do propose a lattice of stronger consisteny levels. However,
assigning the correct consistency level for an operation is a error prone task.

In this paper, we present Quelea, a declarative programming model for
eventually consistent data stores. Quelea allows expressive convergent
replicated data types to be defined over a \emph{shared log} abstraction.
Quelea is equipped with a \emph{contract} language, capable of expressing
fine-grained application-level consistency properties. A \emph{contract
enforcement system} logically analyses the contracts using an SMT solver, and
automatically maps it to the store's consistency level. We describe an
implementation of Quelea on top of an off-the-shelf eventually consistent data
store, and support coordination-free transactions. Several benchmarks including
two large web applications, illustrate the effectiveness of our approach.
\end{abstract}

\section{Introduction}

-- Strong consistency is not viable due to CAP; Weak consistency is the order
of the day.
-- Geo-distributed data stores provide simple interfaces (key-value maps,
column oriented store, sets and counter types); not suitable for describing
high-level datatypes such as ..., See CORFU. Weak consistency deters composable
reasoning. Example, maybe.
-- Stores provide a myriad of consistency levels; complicated, and oft informal
description. Unclear what consistency level should be picked for an operation.
-- Contribution:

\end{document}
