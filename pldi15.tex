%
% LaTeX template for prepartion of submissions to PLDI'15
%
% Requires sigplanconf style file provided on PLDI'15 web site
%
\documentclass[9pt]{sigplanconf}

%
% the following standard packages may be helpful, but are not required
%
%\usepackage{SIunits}            % typset units correctly
\usepackage{courier}            % standard fixed width font
\usepackage[scaled]{helvet} % see www.ctan.org/get/macros/latex/required/psnfss/psnfss2e.pdf
\usepackage{url}                  % format URLs
\usepackage{listings}          % format code
\usepackage{enumitem}      % adjust spacing in enums
% Links should be black and should'nt be underlined. 
\usepackage[breaklinks,draft=false]{hyperref}   % hyperlinks, including DOIs and URLs in bibliography
\usepackage[tight,hang]{subfigure}
\usepackage{xspace}
\usepackage{graphicx}
\usepackage{mathtools}
\usepackage{mathpartir}
\usepackage{amssymb}
\usepackage{amsmath}
\usepackage{textcomp}
% Captions for tables need to be on the top. For figures, they need to
% be at the bottom
\usepackage[labelfont=bf, tableposition=top]{caption}
% ACM-recommended character encoding
\usepackage[utf8]{inputenc}
\usepackage[T1]{fontenc}
\usepackage{microtype}
% For balance of columns on the last page.
\usepackage{flushend}

\usepackage[usenames,dvipsnames,svgnames,table]{xcolor}
\usepackage{amsthm}
\usepackage[ruled,vlined,linesnumbered]{algorithm2e}
%==============================================================================


\newcommand{\cf}[1]{{\small\tt #1}}
\newcommand{\rcf}[1]{\mathrm{\cf{#1}}}
\newcommand{\conj}{~\wedge~}
\definecolor{darkgreen}{rgb}{0,0.5,0}
\definecolor{darkred}{rgb}{0.5,0,0}


%%%%%%%%%%%%%%%%%%%%%%%%%%%%%%%%%%%%%%%%%%%%%%%%%%%%
%%% General
\font\nine=cmr9

\font\ninerm=cmr9
\font\ninebf=cmbx9
\font\nineit=cmti9
\font\ninesl=cmsl9
\font\ninett=cmtt9
\font\ninemi=cmmi9  \skewchar\ninemi='177
\font\ninesy=cmsy9  \skewchar\ninesy='60
\font\nineex=cmex10

%%%%%%%%%%%%%%%%%%%%%%%
%% Math Fonts

\font\mathninerm=cmr9
\font\mathninei=cmmi9
\font\mathninesy=cmsy9


\def\ninepoint{\def\rm{\fam0\ninerm}%
\textfont0=\mathninerm\scriptfont0=\mathsevenrm\scriptscriptfont0=\mathfiverm%
\textfont1=\mathninei\scriptfont1=\mathseveni\scriptscriptfont1=\mathfivei%
\textfont2=\mathninesy\scriptfont2=\mathsevensy\scriptscriptfont2=\mathfivesy%
\textfont3=\tenex
\let\sc=\sevenrm
 \def\it{\fam\itfam\nineit}%
  \textfont\itfam=\nineit
  \def\bf{\fam\bffam\ninebf}%
  \textfont\bffam=\ninebf
  \def\tt{\fam\ttfam\ninett}%
  \textfont\ttfam=\ninett
  \setbox\strutbox=\hbox{\vrule height9pt depth4pt width0pt}%
  \baselineskip=11pt\rm}


\newenvironment{boxit}{\vbox\bgroup\hrule\hbox\bgroup\vrule\kern3pt
    \vbox\bgroup\kern3pt\advance\hsize by -6.8pt\relax}{\par\kern3pt\egroup\kern3pt\vrule\egroup\hrule\egroup}

\font\matheightrm=cmr8
\font\matheightbf=cmbx8
\font\matheighti=cmmi8
\font\matheightsy=cmsy8
\font\eightit=cmti8
\font\eighttt=cmtt8



\def\eightpoint{%
\textfont0=\matheightrm%
\scriptfont0=\mathsixrm\scriptscriptfont0=\mathfiverm
\textfont1=\matheighti\scriptfont1=\mathsixi\scriptscriptfont1=\mathfivei%
\textfont2=\matheightsy\scriptfont2=\mathsixsy\scriptscriptfont2=\mathfivesy%
\def\rm{\fam0\matheightrm}%
\def\it{\fam\itfam\eightit}%
\textfont\itfam=\eightit%
  \def\bf{\fam\bffam\matheightbf}%
  \textfont\bffam=\matheightbf%
  \def\tt{\fam\ttfam\eighttt}%
  \textfont\ttfam=\eighttt%
  \setbox\strutbox=\hbox{\vrule height7pt depth2pt width0pt}%
 \baselineskip=9pt\rm}

\newfam\smallttfam
\newfam\smallfam
\textfont\smallttfam=\ninett
\textfont\smallfam=\matheightrm

\lstloadlanguages{sql}
\newcommand{\lstsql}{\lstset{
      language=sql,
      basicstyle=\ttfamily\footnotesize\small,
      flexiblecolumns=false,
      %basewidth={0.5em,0.45em},
      %aboveskip={3pt},
      %belowskip={3pt},
      keywordstyle=\color{blue}\bfseries,
      commentstyle=\color{darkgreen}\itshape,
      % ugh, it keywords map/sort/zipwith.  should do this for
      % consistency or figure out how to disable those other keywords:
      morekeywords={string,REFERENCES}
    }}
\lstnewenvironment{codesql}
    { % \centering
			\lstsql
      \lstset{}%
      \csname lst@setfirstlabel\endcsname}
    { %\centering
      \csname lst@savefirstlabel\endcsname}

\lstloadlanguages{python}
\newcommand{\lstpython}{\lstset{
      language=python,
      basicstyle=\ttfamily\footnotesize,
      flexiblecolumns=false,
      %basewidth={0.5em,0.45em},
      %aboveskip={3pt},
      %belowskip={3pt},
      keywordstyle=\color{blue}\bfseries,
      commentstyle=\color{darkgreen}\itshape,
      % ugh, it keywords map/sort/zipwith.  should do this for
      % consistency or figure out how to disable those other keywords:
      morekeywords={atomically},
    }}
\lstnewenvironment{codepython}
    { % \centering
			\lstpython
      \lstset{}%
      \csname lst@setfirstlabel\endcsname}
    { %\centering
      \csname lst@savefirstlabel\endcsname}


\lstloadlanguages{haskell}
\newcommand{\lsthaskell}{\lstset{
      language=haskell,
      basicstyle=\ttfamily\ninett\footnotesize,
      flexiblecolumns=false,
			tabsize=2,
      %basewidth={0.5em,0.45em},
      %aboveskip={3pt},
      %belowskip={3pt},
      keywordstyle=\color{blue}\bfseries,
      commentstyle=\color{darkgreen}\itshape,
      % ugh, it keywords map/sort/zipwith.  should do this for
      % consistency or figure out how to disable those other keywords:
      morekeywords={foldl,fold,Quelea},
			classoffset=1,
			upquote=true,
			morekeywords={sameObj,vis,so,SameObj,So,true,Vis},
			keywordstyle=\color{Fuchsia}\bfseries,
			classoffset=0,
			mathescape=true,
      literate={+}{{$+$}}1 {/}{{$/$}}1 {*}{{$*$}}1 % {=}{{$=$}}1
               {>}{{$>$}}1 {<}{{$<$}}1
							 {dollar}{{\$}}1
               {\\\\}{{\char`\\\char`\\}}1
               {->}{{$\rightarrow$}}2 {>=}{{$\geq$}}2 {<-}{{$\leftarrow$}}2
               {<=}{{$\leq$}}2 {=>}{{$\Rightarrow$}}2
               {\ .}{{$\circ$}}2 {\ .\ }{{$\circ$}}2
               {>>}{{>>}}2 {>>=}{{>>=}}2 {=<<}{{=<<}}2
               {|}{{$\mid$}}1
							 {(-}{{$\in$}}1
						   {psi1}{{$\psi_1$}}1 {psi2}{{$\psi_2$}}1
							 {cup}{{$\cup$}}1
							 {cap}{{$\cap$}}1
							 {forall}{{$\forall$}}1
							 {vee}{{$\vee$}}1
							 {wedge}{{$\wedge$}}1
               {`member`}{{$\in$}}1
               {s.empty}{{\{\}}}1
               {leftbrace}{\{}1
               {rightbrace}{\}}1
               {profile0sing}{{ \{{\tt profile0}\}}}1
               {\$singleton\$startv}{{ \hspace{2.4em} \{{\tt startv}\}}}1
               {\$singleton\$n}{{  \{{\tt n}\}}}1
               {dotdotdot}{{$\ldots$}}3
    }}
\lstnewenvironment{codehaskell}
    { % \centering
			\lsthaskell
      \lstset{}%
      \csname lst@setfirstlabel\endcsname}
    { %\centering
      \csname lst@savefirstlabel\endcsname}
% known bug: http://tex.stackexchange.com/questions/1522/pdfendlink-ended-up-in-different-nesting-level-than-pdfstartlink
\newcommand{\doi}[1]{doi:~\href{http://dx.doi.org/#1}{\Hurl{#1}}}   % print a hyperlinked DOI

\newcommand{\HA}{{\sf HA}}
\newcommand{\SA}{{\sf SA}}
\newcommand{\UA}{{\sf UA}}
\newcommand{\scc}{\psi_{\sf sc}}
\newcommand{\ccc}{\psi_{\sf cc}}
\newcommand{\ecc}{\psi_{\sf ec}}
\newcommand{\rcc}{\psi_{\sf rc}}
\newcommand{\mavc}{\psi_{\sf mav}}
\newcommand{\rrc}{\psi_{\sf rr}}
\newcommand{\coloneqq}{::=}

\newtheorem{theorem}{Theorem}
\newtheorem{lemma}[theorem]{Lemma}
\newtheorem{proposition}[theorem]{Proposition}
\newtheorem{corollary}[theorem]{Corollary}
\newtheorem{definition}[theorem]{Definition}
\newcounter{hno}
\newcounter{gno}
\renewenvironment{proof}{\setcounter{hno}{0}\setcounter{gno}{0}
  \emph{Proof.}}{}
\newcommand{\npp}{\thehno \stepcounter{hno}}
\newcommand{\mpp}{\thegno \stepcounter{gno}}

\newcommand{\stretcharraybig}{\renewcommand*{\arraystretch}{1.25}}
\newcommand{\cureff}{\hat{\eta}}
\newcommand{\eff}{\eta}
\newcommand{\fresh}{\N{\sf fresh}}
\newcommand{\cv}{\psi}
\newcommand{\ALT}{~\mid~}
\newcommand{\eid}{{\iota}}
\newcommand{\ObjType}{{\sf ObjType}}
\newcommand{\AbsType}{{\sf AbsType}}
\newcommand{\dom}{{\sf dom}}
\newcommand{\DtLib}[1]{\mathbb{D}(#1)}
\newcommand{\DtLibZ}{\mathbb{D}}
\newcommand{\Ops}{\Lambda}
\newcommand{\Ctrts}{\Psi}
\newcommand{\true}{\N{\textsf{true}}}

\newcommand{\R}[1]{\textrm{#1}}
\newcommand{\N}[1]{{\normalfont #1}}
\newcommand{\ObjZ}{\N{\textsf{Obj}}}
\newcommand{\Obj}[1]{\N{\textsf{Obj}_{#1}}}
\newcommand{\ReplID}{\mathtt{ReplID}}
\newcommand{\SessID}{\mathtt{SessID}}
\newcommand{\typeFun}[1]{\N{\textsf{type}}(#1)}
\newcommand{\Op}[1]{\N{\textsf{Op}_{#1}}}
\newcommand{\set}[1]{\overline{#1}}
\newcommand{\unitVal}{\N{\textsf{unit}}}
\newcommand{\EffUniv}{\N{\sf Effect}}
\newcommand{\EffID}{\mathtt{SeqNo}}
\newcommand{\TransID}{\N{\sf TransID}}
\newcommand{\AVal}[1]{\N{\textsf{AVal}_{#1}}}
\newcommand{\RVal}[1]{\N{\textsf{RVal}_{#1}}}
\newcommand{\Eff}[1]{\N{\textsf{Eff}_{#1}}}
\newcommand{\loud}[1]{\textbf{\textit{#1}}}
\newcommand{\dt}[1]{\mathcal{D}_{#1}}
\newcommand{\vis}[2]{\N{\textsf{vis}(#1,#2)}}
\newcommand{\visZ}{\N{\textsf{vis}}}
\newcommand{\Rvis}{\N{\textsf{vis}}}
\newcommand{\arZ}{\N{\textsf{ar}}}
\newcommand{\ar}[2]{\N{\textsf{ar}(#1,#2)}}
\newcommand{\stxZ}{\sim}
\newcommand{\stx}[2]{#1\sim#2}
\newcommand{\nstx}[2]{#1\not\sim#2}
\newcommand{\comZ}{\N{\textsf{com}}}
\newcommand{\com}[1]{\comZ(#1)}
\newcommand{\so}[2]{\N{\textsf{so}(#1,#2)}}
\newcommand{\soZ}{\N{\textsf{so}}}
\newcommand{\stZ}{\N{\textsf{st}}}
\newcommand{\Rso}{\N{\textsf{so}}}
\newcommand{\Rst}{\N{\textsf{st}}}
\newcommand{\COM}{\textrm{\sc {\small Commit}}}
\newcommand{\soo}[2]{\N{\textsf{soo}(#1,#2)}}
\newcommand{\sooZ}{\N{\textsf{soo}}}
\newcommand{\hb}[2]{\N{\textsf{hb}(#1,#2)}}
\newcommand{\hbo}[2]{\N{\textsf{hbo}(#1,#2)}}
\newcommand{\hbZ}{\N{\textsf{hb}}}
\newcommand{\hboZ}{\N{\textsf{hbo}}}
\newcommand{\oper}[2]{\N{\textsf{oper}(#1,#2)}}
\newcommand{\operZ}{\N{\textsf{oper}}}
\newcommand{\txnZ}{\N{\textsf{txn}}}
\newcommand{\txn}[2]{\txnZ\{#1\}\{#2\}}
\newcommand{\sameobj}[2]{\N{\textsf{sameobj}(#1,#2)}}
\newcommand{\sameobjZ}{\N{\textsf{sameobj}}}
\newcommand{\sametxn}[2]{\N{\textsf{sametxn}(#1,#2)}}
\newcommand{\sametxnZ}{\N{\textsf{sametxn}}}
\newcommand{\E}{\N{\textsf{E}}}
\newcommand{\EffSoup}{\N{\textsf{A}}}
\newcommand{\obj}{\N{\textsf{obj}}}
\newcommand{\dep}{\N{\textsf{dep}}}
\newcommand{\rval}{\N{\textsf{rval}}}
\newcommand{\repl}{\N{\textsf{repl}}}
\newcommand{\sess}{\N{\textsf{sess}}}
\newcommand{\rdtspec}{\Delta}
\newcommand{\goesto}{\longrightarrow}
\newcommand{\tuplee}[1]{\langle #1 \rangle}
\newcommand{\ctxtFn}{\N{\textsf{ctxt}}}
\newcommand{\rdtredsto}{\N{\leadsto}}
\newcommand{\Exec}{\N{\textsf{(\EffSoup,\allowbreak \visZ,\allowbreak \soZ,\allowbreak \sameobjZ)}}}
\newcommand{\pll}{~\|~}
\newcommand{\Mod}[1]{\N{\textsf{Mod}}(#1)}
\newcommand{\De}[1]{#1}
\newcommand{\Der}[2]{[\![#1,#2]\!]_{r}}
\newcommand{\msentails}[2]{#1 \models #2}
\newcommand{\hasTyp}[2]{#1 \vdash #2}
\newcommand{\auxred}[4]{#1 #2 \;\xrightarrow{#3}\; #4 }
\renewcommand{\qed}{\nobreak \ifvmode \relax \else
      \ifdim\lastskip<1.5em \hskip-\lastskip
      \hskip1.5em plus0em minus0.5em \fi \nobreak
      \vrule height0.75em width0.5em depth0.25em\fi}


% Operational semantics rules
\newcommand{\rulelabel}[1]{\textrm{\sc {\small [#1]}}}
\newcommand{\RULE}[3]
{\frac{\begin{array}{c}#1\end{array}}
		 {\begin{array}{c}#2\end{array}}
~ \rulelabel{#3}
}

\newcommand{\RuleTwo}[2]
{\frac{\begin{array}{c}#1\end{array}}
		 {\begin{array}{c}#2\end{array}}
}

\newenvironment{nop}{}{}
\newenvironment{smathpar}{
\begin{nop}\small\begin{mathpar}}{
\end{mathpar}\end{nop}\ignorespacesafterend}

\newcommand{\rsf}[1]{\R{\sf #1}}
\newcommand{\trans}[4]{\N{\textsf{trans}}\{#1,#2\}\{#3,#4\}}

\newcommand{\name}{{\sc Quelea}\xspace}


%% KC spacing edits -- should be removed eventually -- Fix the text!
\setlength{\floatsep}{5pt}
\setlength{\textfloatsep}{10pt}
\setlength{\dblfloatsep}{5pt}
\setlength{\dbltextfloatsep}{10pt}


\newcommand{\ecds}{ECDS\xspace}


\newcommand{\ecds}{ECDS\xspace}

% To eliminate orphans
\clubpenalty = 10000
\widowpenalty = 10000
\displaywidowpenalty = 10000

\begin{document}
\conferenceinfo{PLDI'15}{, June 13--17, 2015, Portland, OR, USA}
\CopyrightYear{2015}
\crdata{978-1-4503-3468-6/15/06}
%
% any author declaration will be ignored  when using 'pldi' option (for double blind review)
%
\title{Declarative Programming over Eventually Consistent Data Stores }

\authorinfo{KC Sivaramakrishnan}{University of Cambridge, UK}{sk826@cl.cam.ac.uk}
\authorinfo{Gowtham Kaki}{Purdue University, USA}{gkaki@cs.purdue.edu}
\authorinfo{Suresh Jagannathan}{Purdue University, USA}{suresh@cs.purdue.edu}

\maketitle
\begin{abstract}
User-facing online services utilize geo-distributed data stores to minimize
latency and tolerate partial failures, with the intention to provide a fast,
always-on experience. However, geo-distribution does not come for free;
application developers have to contend with weak consistency behaviors, and
the lack of abstractions to composably construct high-level replicated data
types, necessitating the need for complex application logic and invariably
exposing inconsistencies to the user. Some commercial distributed data
stores and several academic proposals provide a lattice of consistency
levels, with stronger consistency guarantees incurring increased latency and
throughput costs. However, correctly assigning the right consistency level
for an operation requires subtle reasoning and is often an error-prone task.

In this paper, we present \name, a declarative programming model for eventually
consistent data stores (\ecds), equipped with a \emph{contract} language,
capable of specifying fine-grained application-level consistency properties. A
\emph{contract enforcement system} analyses contracts, and \emph{automatically}
generates the appropriate consistency protocol for the method protected by the
contract. We describe an implementation of \name\ on top of an off-the-shelf
\ecds, and provide support for \emph{coordination-free} transactions. Several
benchmarks including two large web applications, illustrate the effectiveness
of our approach.
\end{abstract}

% Classification
\category{D.1.3}{Concurrent Programming}{Distributed Programming}
\category{C.2.4}{Distributed Systems}{Distributed databases}
\category{D.3.2}{Language Classifications}{Applicative (Functional) Languages}
\category{F.3.1}{Logics and Meanings of Programs}{Specifying and Verifying and Reasoning about Programs}

\terms
Languages, Performance 

\keywords
Eventual Consistency, Availability, CRDTs, Axiomatic Contracts,
Contract Classification, Distributed Transactions, SMT solvers,
Decidable Logic, Quelea, Cassandra, Haskell 

\section{Introduction}

Many real-world web services --- such as those built and maintained by
Amazon, Facebook, Google, Twitter, etc. --- replicate application state and
logic across multiple \emph{replicas} within and across data
centers. Replication is intended not only to improve application throughput
and reduce user-perceived latency, but also to tolerate partial failures
without compromising overall service availability. Traditionally programmers
have relied on \emph{strong consistency} guarantees such as
linerarizability~\cite{Herlihy1990} or
serializability~\cite{Serializability} in order to build correct
applications.  While strong consistency is an easily stated property, it
masks the reality underlying large-scale distributed systems with respect to
non-uniform latency (i.e., availability) and network
partitions~\cite{Brewer2000,Gilbert}. Indeed, modern web services, which aim
to provide an "always on" experience, overwhelmingly favour availability and
partition tolerance over strong consistency. To this end, several \emph{weak
  consistency} models such as eventual consistency, causal consistency,
session guarantees, and timeline consistency have been proposed.

Under weak consistency, the developer needs to be aware of concurrent
conflicting updates, and has to pay careful attention to avoid unwanted
inconsistencies (e.g., negative balances in a bank account, or having an
item appear in a shopping cart after it has been
removed~\cite{Dynamo}). Oftentimes, the inconsistency leaks from the
application and is witnessed by the user.  Ultimately, the developer must
decide the consistency level appropriate for a particular operation; this is
understandably an error-prone process requiring intricate knowledge of both
the application as well as the semantics and implementation of the
underlying data store, which typically have only informal
descriptions~\cite{}.  Nonetheless, picking the correct consistency level is
critical not only for correctness but also for scalability of the
application. While choosing a weaker consistency level than required may
introduce program errors and anomalies, choosing a stronger one than
necessary can negatively impact program scalability.

%--KC--
% The choice of correct consistency level is critical to the correctness and
% scalability of the application. Choosing a weaker consistency level than
% required introduces anomalies, while choosing a stronger than necessary level
% affects program scalability.
%\KC{Do we include the LWW-Register number here to add emphasis to the fact that
%choosing the correct consistency level is essential for program scalability}.


Weak consistency also hinders compositional reasoning about programs.  While
an application might be naturally expressed in terms of well-understood and
expressive data types such as maps, trees, queues, or graphs,
geo-distributed stores typically only provide a minimal set of data types
with in-built conflict resolution strategies such as last-writer-wins (LWW)
registers, counters, and sets~\cite{Cassandra,DynamoDB}.  Furthermore, while
traditional database systems enable composability through transactions,
geo-distributed stores typically lack unrestricted transactional access to
the data.  Working in this environment thus requires application state to be
suitably coerced to function using only the capabilities of the store.

To address these issues, we describe \name, a declarative programming model
and implementation for eventually consistent geo-distributed data
stores. The key novelty of \name is an expressive \emph{contract language}
to declare and verify fine-grained application-level consistency
properties. The programmer uses the contract language to axiomatically
specify the set of legal executions allowed over the replicated data
type. Contracts are constructed using primitive consistency relations such
as \emph{visibility} and \emph{session order} along with standard logical
and relational operators. A \emph{contract enforcement system} automatically
maps operations over the datatype to a particular consistency level
available on the store, and provably validates the correctness of the
mapping.  The paper makes the following contributions:

\begin{itemize}
\setlength{\itemsep}{2pt}
\item We introduce \name, a language for declaratively describing eventually
	consistent programs that manipulate replicated data types. Contracts are used
	to specify fine-grained application-level consistency properties, and are
	analyzed to assign the most efficient and sound store consistency level to
	the corresponding operation.
\item \name supports coordination-free transactions over arbitrary datatypes.
	We extend our contract language to express fine-grained transaction isolation
	guarantees, and utilize the contract enforcement system to automatically
	assign the correct and most efficient isolation level for a transaction.
\item We provide metatheory that certifies the soundness of our contract enforcement
	system, and ensures that an operation is only executed if the required
	conditions on consistency are met.
\item \name is implemented as a transparent shim layer over
  Cassandra~\cite{Cassandra}, a well-known general-purpose data store.  An
  extensive experimental evaluation over a collection of real-world
  applications, including a Twitter-like micro-blogging site and an
  eBay-like auction site illustrates the practicality of our approach.
\end{itemize}

The rest of the paper is organized as follows. Section~\ref{sec:motivation}
motivates \name and introduces the language through a series of examples.
Section~\ref{sec:lang} describes the contract language, and
Section~\ref{sec:opsem} presents an operational semantics.
Section~\ref{sec:txns} introduces transaction contracts and classification.
Section~\ref{sec:impl} describes the implementation and goes into detail about
the optimizations needed to make the system practical.
Section~\ref{sec:results} discusses experimental evaluation.
Section~\ref{sec:related} and~\ref{sec:concl} present related work and
conclusions.


\section{System Model}
\label{sec:sysmod}

\begin{figure}
\centering
\includegraphics[width=0.75\columnwidth]{Figures/SystemModel}
\caption{\name system model.}
\label{fig:sysmod}
\end{figure}

Figure~\ref{fig:sysmod} provides a schematic diagram of our system model. The
distributed store is composed of a collection of \emph{replicas}, each of which
stores a set of \emph{objects} ($x,y,\ldots$). We assume that every object is
replicated at every replica in the store. The state of an object at any replica
is the set of all updates (\emph{effects}) performed on the object. For
example, the state of $x$ at replica 1 is the set composed of effects $w^x_1$
and $w^x_2$.

Each object is associated with a set of \emph{operations}. The clients interact
with the store by invoking operations on objects. The sequence of operations
invoked by a particular client on the store is called a \emph{session}. The
data store is typically accessed by a large number of clients (and hence
sessions) concurrently. Importantly, the clients are oblivious to which replica
an operation is applied to; the data store may choose to route the operation to
any replica in order to minimize latency, balance load, etc. For example, the
operations \emph{foo} and \emph{bar} invoked by the same session on the same
object, might end up being applied to different replicas because replica 1 (to
which \emph{foo} was applied) might be unreachable when the client invokes
\emph{bar}.

When \emph{foo} is invoked on a object $x$ with arguments \emph{arg}$_1$ at
replica 1, it simply \emph{reduces} over the current set of effects at that
replica on that object ($w^x_1$ and $w^x_2$), produces a result $v1$ that is
sent back to the client, and emits a \emph{single} new effect $w^x_4$ that is
appended to the state of $x$ at replica 1. Thus, every operation is evaluated
over a \emph{snapshot} of the state of the object on which it is invoked. In
this case, the effects $w^x_1$ and $w^x_2$ are \emph{visible} to $w^x_4$,
written logically as $\vis{w^x_1}{w^x_4} \wedge \vis{w^x_2}{w^x_4}$, where
$\visZ$ is the visibility relation between effects. Visibility is an
irreflexive and asymmetric relation, and only relates effects produced by
operations on the same object. Executing a read-only operation is similar
except that no new effects are produced.

The effect added to a particular replica is asynchronously sent to other
replicas, and eventually merged into all other replicas. Two effects $w^x_4$
and $w^x_5$ that arise from the same session are said to be in \emph{session
order} (written logically as $\so{w^x_4}{w^x_5}$). Session order is an
irreflexive, transitive relation. The effects $w^x_4$ and $w^x_5$ arising from
operations applied to the same object $x$ are said to be under the \emph{same
object} relation, written $\sameobj{w^x_4}{w^x_5}$. Finally, we can associate
every effect with the operation that generated the effect with the help of a
relation $\operZ$. In the current example, $\oper{w^x_4}{foo}$ and
$\oper{w^x_5}{bar}$ hold. For simplicity, we assume all operation names across
all object types are distinct.

This model admits all the inconsistencies associated with eventual consistency.
The goal of this work is to identify the precise consistency level for each
operation such that application-level constraints are not violated. In the next
section, we will concretely describe the challenges associated with
constructing a consistent bank account on top of an eventually consistent data
store. Subsequently, we will illustrate how our contract and specification
language, armed with the primitive relations $\visZ$, $\soZ$, $\sameobjZ$ and
$\operZ$, mitigates these challenges.


\section{Motivation}

In this section, we motivate the need for declarative reasoning for eventually
consistent data store.

\subsection{System Model}

We assume a system model where the eventually consistent database is organized
as a collection of key-indexed, object-valued tables. The objects stored in the
tables are instances of the same replicated data type. The database is composed
of a number of \emph{replicas}, and each object is fully replicated across all
the replicas. Under this model, a client request for an operation with no
application-level consistency requirement can be serviced if at least one of
the replicas is reachable. Thus, with the addition of new replicas, and placing
them close to the clients, leads to improved throughput and reduced latencies.

\subsection{Bank account database}

Suppose our goal is to implement a bank account service on top an eventually
consistent data store, with the following schema,

\begin{codesql}
Create Table BankAccount (
  userID int PRIMARY KEY,
  userName string UNIQUE,
  balance float CHECK (balance >= 0))
\end{codesql}

\noindent with the following \emph{operations}.

\begin{codehaskell}
type UserName = String
-- returns false if the user name is already taken
addAccount     :: UserName -> Bool
getBalanceName :: UserName -> Float
depositName    :: UserName -> Float -> ()
-- returns false if the account has insufficient balance
withdrawName   :: UserName -> Float -> Bool
\end{codehaskell}

Notice that the operations manipulating the table take user name as input.
Since eventually consistent databases are non-relational~\cite{}, we need to
maintain a \emph{secondary index} to look up records by user name.

\begin{codesql}
Create Table UserIndex (
  userName string PRIMARY KEY,
  userID int
  FOREIGN KEY (userID)
    REFERENCES BankAccount (userID) )
\end{codesql}

\begin{figure}[t]
\centering
\subfigure[Negative Balance]{\label{fig:negativeBalanceAnomaly}\includegraphics[width=0.31\columnwidth]{Figures/Motivation2}}
\hfill
\subfigure[Missing update]{\label{fig:missingUpdateAnomaly}\includegraphics[width=0.26\columnwidth]{Figures/Motivation1}}
\hfill
\subfigure[Monotonicity violation]{\label{fig:monotonicityAnomaly}\includegraphics[width=0.36\columnwidth]{Figures/Motivation3}}
\caption{Anomalies possible under eventual consistency for the get balance operation.}
\label{fig:cleanliness_examples}
\end{figure}

\subsection{Anomalies}

In a traditional database system, the programmer need only to define the
application-level integrity constraints along with the schema, and the database
management system automatically enforces the necessary coordination. However,
under eventual consistency, the onus is on the programmer to ensure not only
that the integrity constraints are preserved, but also to prevent any
inconsistencies being exposed to the user. For example, it is easy to see that
the withdraw operations must be strongly consistent operations to preserve the
integrity constraint. However, it is possible that the get balance operation
still returns a negative balance.

Figure~\ref{fig:negativeBalanceAnomaly} illustrates such a potential execution.
Assume that all operations are on the same object, and the balance on the
account was initially 0. Here, \cf{vis} edges capture the visibility
relation between the operations. In this case, the withdraw sees the deposit,
but the get balance operation only witnesses the withdraw. Such an execution is
possible under eventual consistency, where the \emph{effect} (update)
corresponding to the withdraw reaches the replica to which the get balance is
applied to before the causally preceding deposit effect. Alternatively, it is
possible that a get balance operation does not see the effects from its own
session (Figure~\ref{fig:missingUpdateAnomaly}), leading the user to
incorrectly conclude that the previous update failed to go through. Here, the
\cf{so} edge represents the session order that exists between the actions
on the same thread.

The anomalies discussed so far arise from weakly consistent behaviors on the
same object. With operations on multiple objects on distinct tables, the
programmer on eventually consistent databases has to take care of preserving
integrity constraints for foreign keys, materialized views and secondary
indexes. For example, Figure~\ref{fig:monotonicityAnomaly} illustrates the
anomaly that arises from the lack of monotonic update visibility. Adding a new
account involves a conditional insertion into the index table, followed by an
insertion into the bank account table. Many eventually consistent data stores
propose write-only transactions~\cite{}, and let us assume that add user
operation is marked as a write-only transaction. Despite this, under eventual
consistency, it is possible for a \cf{getBalanceName} operation to witness the
first insertion but not the second, exposing an inconsistent state. The two
reads may be served by different replicas, where the only the former has
witnessed the effects of add user transaction. Assigning the correct store
consistency level for the operations in this example such that the
application-level constraints are preserved is a non-trivial task.

\subsection{Implementation in Quelea}

Quelea automates this error-prone task; the programmer need only to express the
application-level consistency constraints in the contract language. Replicated
data types in Quelea are defined as reductions over the set of effects similar
to operation-based CRDTs~\cite{}. The following data type definitions capture
the operations and effects on the bank account and user index objects.

\begin{codehaskell}
data BankAcc = AddUser UserName | Deposit Float
             | Withdraw Float | GetBalance
data UserIdx = AddName UserID | GetName
\end{codehaskell}

Recall that the key of the bank account table is user id, which maps to a
particular bank account object. In Quelea, an object state is nothing but the
set of effects (called the \emph{context}) on this object. As such,
\cf{Deposit} and \cf{Withdraw} effects simply include the amount deposited or
withdrawn, and \cf{AddUser} effect includes the user name. Similarly, on the
user index table, where the key is user name, the value is simply an effect
(\cf{AddName}) binding some user id at this key. \cf{GetBalance} and
\cf{GetName} correspond to the corresponding read-only operations. The complete
definition of the operations on bank account and user index objects is given
below.

\begin{codehaskell}
-- Get balance takes no arguments and read-only
getBalance :: [BankAcc] {- context -} -> () {- args -}
  -> (() {- ret val -}, Maybe BankAcc {- effect -})
getBalance ctxt _ = (sum [v | Deposit v (- ctxt]
        - sum [v | Withdraw v (- ctxt], Nothing)

-- withdraw returns True on success
withdraw :: [BankAcc] -> Float -> (Bool, Maybe BankAcc)
withdraw c v = if sel1 $ getBalance c () >= v
               then (True, Just $ Withdraw v)
               else (False, Nothing)

deposit :: [BankAcc] -> v -> ((), Maybe BankAcc)
deposit _ v = ((), Just $ Deposit v)

addUser :: [BankAcc] -> UserName -> ((), Maybe BankAcc)
addUser _ n = ((), Just $ AddUser n)

addName :: [UserIdx] -> UserID -> (Bool, Maybe UserIdx)
addName [] uid = (True, Just $ AddName uid)
addName x:_ _ = (False, Nothing)

getName :: [UserIdx] -> ()
        -> (Maybe UserName, Maybe UserIdx)
getName [] _ = (Nothing, Nothing)
getName (AddUser uid:_) _ = (Just uid, Nothing)
\end{codehaskell}

The definitions are a straight forward encoding of the expected behavior.
Quelea programming model provides the programmer complete freedom regarding the
semantics and desired convergence property of the replicated data type, and
indeed, we can encode the well-known CRDTs~\cite{SSS} in a declarative fashion.
Observe that the operation definitions presented here only capture the
convergence properties of the replicated data type, and are not concerned with
the consistency properties, which is expressed through the contract language.

\subsection{Contracts}

The contract for the strongly consistent \cf{addName} operation is given below:
\begin{smathpar}
\begin{array}{l}
\rsf{addNameCtrt} = \forall (a:\rsf{AddName}). \\
\qquad \sameobj{a}{\cureff} \Rightarrow \vis{a}{\cureff} \vee \vis{\cureff}{a} \vee a = \cureff
\end{array}
\end{smathpar}

In the above definition, $\cureff$ represents the current effect --- the effect
emitted by the \cf{addName} operation. The contract simply states our
high-level observation; for any effect $a$ which is also an \cf{AddName} effect
on the same object, it must be the case that $a$ is visible to $\cureff$ or
vice verse, or $a$ is $\cureff$. Here, $\sameobjZ$ and $\visZ$ are primitive
relations. This contract ensures that any two \cf{addName} operations will be
seen by each other. The contract for withdraw is similar to the
\cf{addNameCtrt}.

Since the deposit operation does not have any restrictions, its contract is
simply $\true$. Same is the case for \cf{addUser} operation. The contract for
get balance operation is:
\begin{smathpar}
\begin{array}{l}
\rsf{getBalCtrt} = \forall (a:\rsf{Deposit}), (b:\rsf{Withdraw}). \\
\qquad \vis{a}{b} \wedge \vis{b}{\cureff} \Rightarrow \vis{a}{\cureff} \\
\qquad \vee (\soZ \cap \sameobjZ) (a,\cureff) \Rightarrow \vis{a}{\cureff} \\
\qquad \vee (\soZ \cap \sameobjZ) (b,\cureff) \Rightarrow \vis{b}{\cureff}
\end{array}
\end{smathpar}

If a withdraw $b$ is visible to the get balance operation, then all deposit
operations $a$ visible to withdraw should be visible to the get balance
operation. This prevents negative balance anomalies. The rest of the contract
says that a get balance operation must witness previous deposit and withdraw
operations on the same object in the same session. This prevents missed update
anomalies.

Similar to contracts on operations, Quelea supports contracts on transactions.
The contract on the \cf{getBalanceName} transaction is given below:
\begin{smathpar}
\begin{array}{l}
\rsf{getBalanceNameCtrt} = \forall (a:\rsf{GetName}), (b:\rsf{GetBalance}), \\
\quad (c:\rsf{AddName}), (d:\rsf{AddUser}). ~\trans{a}{b}{c}{d} \wedge \so{a}{b} \\
\qquad \wedge ~\vis{c}{a} \wedge \sameobj{d}{b} \Rightarrow \vis{d}{b}
\end{array}
\end{smathpar}

$\trans{a}{b}{c}{d}$ says that the action pairs $a,b$ and $c,d$ are in the same
transaction, and the two transactions are distinct. The contract says
specifically forbids the anomaly presented in
Figure~\ref{fig:monotonicityAnomaly}. This is the desired semantics for the
\cf{getUserName} transaction.

The programmer simply defines such contracts on the operations and the
transactions. Quelea logically analyzes the contracts, and maps the
corresponding operation to the precise store consistency and isolation level.
Thus, Quelea equips the programmer with a declarative model for reasoning and
expressing eventually consistent programs.


\section{Contract Language}
\label{sec:contract-lang}

% Contract Language Syntax
% ------------------------
\begin{figure}
\begin{smathpar}
\stretcharraybig
\begin{array}{rclcl}
\multicolumn{5}{l}{
  {x,y,z} \in \mathtt{EffVar} \qquad
  {\cureff} \in \mathtt{CurEff} \qquad
  {\sf Op} \in \mathtt{OperName}
}\\
\cv 		& \in & \mathtt{Contract} 	& \coloneqq & \forall (x : \tau).\cv
        \ALT \pi \\
\tau		& \in	& \mathtt{EffType}	& \coloneqq &  {\sf Op}
        \ALT \tau \vee \tau \\
\pi			&	\in & \mathtt{Prop} & \coloneqq & \true \ALT R(x,y)
        \ALT \pi \vee \pi \\
			  & 		&	 &  \ALT & \pi \wedge \pi \ALT \pi \Rightarrow \pi \\
R				& \in & \mathtt{Relation}	& \coloneqq & \visZ \ALT \soZ
        \ALT \sameobjZ \ALT R^+ \\
				&			&	 &  \ALT & R \cup R \ALT R \cap R \\
\end{array}
\end{smathpar}
\caption{The Contract Language}
\label{fig:contract-lang}
\end{figure}


In this section, we formalize the contract language of \name, and describe our
contract classification scheme, which analyzes a contract and maps it to the
weakest store consistency level sufficient to satisfy its consistency
requirements.

% classifies contracts on the basis of the weakest
% store-level consistency guarantee

\subsection{Syntax}

The syntax of our core contract language is shown in Fig.
~\ref{fig:contract-lang}. The language is based on first-order logic
(FOL), and admits prenex universal quantification over typed effect
variables. We use a special effect variable ($\cureff$) to denote the
effect of \emph{current operation} - the operation for which a
contract is being written. The type of an effect is simply the name of
the operation (eg: \cf{withdraw}) that induced the effect. We admit
disjuntion in types to let an effect variable range over multiple
operation names.

Quantifier-free propositions in our contract language are
conjunctions, disjunctions and implications of predicates expressing
relations between pairs of effect variables. The syntactic class of
relations is seeded with primitive $\visZ$, $\soZ$, and $\sameobjZ$
relations, and also admits derived relations that are expressible as
union, intersection, or transitive closure\footnote{Strictly speaking,
$R^{+}$ is not the transitive closure of $R$, as transitive closure is
not expressible in FOL.  Instead, $R^{+}$ in our language denotes
\emph{a} superset of transitive closure of $R$. Formally, $R^{+}$ is
any relation $R'$ such that forall $x$, $y$, and $z$, a) $R(x,y)
\Rightarrow R'(x,y)$, and b) $R'(x,y) \conj R'(y,z) \Rightarrow
R'(x,z)$} of primitive relations.  Commonly used derived relations are
the \emph{same object session order} ($\small \sooZ = \soZ ~\cap~
\sameobjZ$), 
%the \emph{happens before order} ($\small \hbZ = (\soZ
%~\cup~ \visZ)^+$), 
and the \emph{same object happens-before order} ($\small \hboZ = (\sooZ
~\cup~ \visZ)^+$).


\subsection{Capturing Store Semantics}
\label{sec:store_sem}

An important aspect of our contract classification system is that the store
semantics is also captured using the same contract language used to describe
application-level consistency. In this regard, similar
to~\cite{Burckhardt2014}, we can rigorously define a wide variety of store
semantics including those that combine any subset of session and causality
guarantees, and multiple consistency levels. For example, a store that offers
strong consistency is captured by the contract:

\vspace{-1em}
\begin{smathpar}
\scc = \forall a.~\sameobj{a}{\cureff} \Rightarrow \vis{a}{\cureff} ~\vee~ \vis{\cureff}{a} ~\vee~ a = \cureff
\end{smathpar}

\noindent Similarly, a store that offers per-object causal consistency is captured by the
contract:

\vspace{-1em}
\begin{smathpar}
\ccc = \forall a.~(\hboZ \cap \sameobjZ) (a,\cureff) \Rightarrow \vis{a}{\cureff}
\end{smathpar}

\noindent This ability to represent store semantics and application-level consistency in
the same language is vital to contract classification. Observe that out
contract language does not incorporate real (i.e., wall-clock) time. Hence, the
contract language cannot describe store semantics based on real time such as
recency or bounded-staleness guarantees offered by certain
stores~\cite{Pileus}.

\subsection{Contract Comparison}

Our goal is to classify contracts, and map the operation they describe to
the \emph{weakest} store-level consistency level that  nonetheless
satisfies the contract's constraints. To this end, we need a mechanism to 
compare the ``strength'' of a contract. Let $\cv_{op}$ be a contract for a
particular operation $op$, and $\cv_{st}$ capture a particular store
consistency level. We would like to determine whether $op$ can be
\emph{safely discharged} at the store consistency level $\cv_{st}$ such that
the resulting execution does not violate $\cv_{op}$.

Since our contracts represent axiomatic definition of program
executions, let $\Mod{\cv}$ be the set of all executions under which
contract $\cv$ is satisfied. If every execution $\E \in
\Mod{\cv_{st}}$ is also a member of $\Mod{\cv_{op}}$, then $op$ can be
safely discharged under the store consistency level $\cv_{st}$.
Formally, $\Mod{\cv_{st}} \subseteq \Mod{\cv_{op}}$. This is the
model-theoretic consequence relation, written as $\cv_{st} \models_m
\cv_{op}$. In this case, we say that $\cv_{op}$ is \emph{weaker than}
$\cv_{st}$ (written $\cv_{op} \le \cv_{st}$). 

Observe that our contract language (stripped of its syntactic sugar) is a
carefully chosen subset of first-order logic that is known to be
decidable~\cite{epr}.  Since first-order logic is sound and complete
~\cite{completeness}, $\cv_{st} \models_m \cv_{op}$ if and only if
$\cv_{st} \Rightarrow \cv_{op}$.  Due to the decidability of our
contract language, this implication check is automatically discharged
with the help of a theorem prover. 


\subsection{Contract Classification}

While the store semantics offered by commercial and research data stores
vary widely, for our purposes, we identify three particular consistency
levels -- eventual, causal, and strong, commonly offered by many distributed
stores with tunable consistency, with increasing overhead in terms of
latency and availability. Indeed, the techniques presented here can be
extended to other consistency stratifications. We assign each
application-level contract into one of these following classes:


\begin{itemize}
\setlength{\itemsep}{2pt}

\item \textbf{Eventually consistency}: Eventually consistent operations can
  be satisfied as long as the client can reach at least one replica. For
  example, \cf{deposit} is an eventually consistent operation; its semantics
  does not require its action to manifest on all replicas before other
  operations in its session are allowed to proceed. While eventually
  consistent data store typically offer \emph{basic} eventual consistency
  with all possible anomalies, we assume that our store provides stronger
  semantics that remain highly-available~\cite{BailisHAT,COPS}; the store
  always exposes a causal cut of the updates. This semantics can be formally
  captured in terms of the following contract definition:
\vspace{-0.6em}
\begin{smathpar}
\ecc = \forall a,b. \hbo{a}{b} \wedge \vis{b}{\cureff} \Rightarrow \vis{a}{\cureff}
\end{smathpar}
\noindent where $\small \hboZ = ((\soZ \cap \sameobjZ) \cup \visZ)^+$.

\item \textbf{Causal consistency}: Operations with causally consistent
  contracts are required to see a causally consistent snapshot of the object
  state, including the actions performed on the same session.  The latter
  requirement entails that if two operations $o_1$ and $o_2$ from the same
  session are applied to two different replicas $r_1$ and $r_2$, the second
  operation cannot be discharged until the effect of $o_1$ is merged
  with $o_2$ in both $r_1$ and $r_2$. The \cf{getBalance} operation
  requires causal consistency, as it requires the operations from the
  same session to be visible, which cannot be guaranteed under
  eventual consistency. We assume that causality is only tracked
  through operations on the same object; two operations in the same
  session but on different objects are considered causally unrelated
  under this definition. Stores typically avoid tracking causality
  across objects to mitigate overheads when causality tracking is
  unnecessary. The corresponding store semantics is captured by the
  contract $\ccc$ presented in Section~\ref{sec:store_sem}.

\item \textbf{Strong Consistency}: Strongly consistent operations may block
  indefinitely under network partitions. An example is the total-order
  contract on \cf{withdraw} operation. The corresponding store semantics is
  captured by the contract $\scc$ presented in Section~\ref{sec:store_sem}.

\end{itemize}

\noindent Observe that the contracts $\scc$, $\ccc$ and $\ecc$ are
themselves totally ordered with respect to the $\le$ relation: $\ecc \le
\ccc \le \scc$. This concurs with the intuition that any contract
satisfiable under $\ecc$ or $\ccc$ is satisfiable under $\scc$, and any
contract that is satisfiable under $\ecc$ is satisfiable under
$\ccc$. Nonetheless, we determine the class of a contract based on the
\emph{weakest} guarantee (among $\ecc$, $\ccc$, and $\scc$) required to
satisfy the contract. The classification scheme is presented formally 
in Figure~\ref{sem:classify}. Along with three straightforward rules
that classify contracts into consistency classes, the classification
scheme also presents a rule that judge well-formedness of a contract.
A contract is well-formed if and only if it is satisfiable under
$\scc$ - the strongest possible consistency guarantee any store can
provide. The $\le$ relation used by the rules is defined formally
below:

\begin{definition}
A contract $\cv_{op}$ is said to be weaker than $\cv_{st}$ (written $\cv_{op}
\le \cv_{st}$ ) if and only if $\Delta \vdash \cv_{st} \Rightarrow \cv_{op}$.
\begin{center}
\end{center}
\end{definition}
\vspace{-1em}
The $\Delta$ referred in the above defintion captures assumptions
about the nature of primitive relations, such as $\Rvis$ and $\Rso$
are irreflexive, and the happens-before relation $\hbZ$ is acyclic,
preventing thin-air reads. \KC{Ideally, we need to list all the axioms
somewhere.}

% Contract classification rules
% ------------------------------
\newcommand{\DDe}[1]{#1}
\begin{figure}
\begin{smathpar}
\begin{array}{c}
\hspace{-0.5em}
\vspace{3mm}
\RuleTwo
{\DDe{\cv} \le \DDe{\scc}}
{{\sf WellFormed}(\cv)}  \qquad

\RuleTwo
{\DDe{\cv} \le \DDe{\ecc}}
{{\sf EventuallyConsistent}(\cv)} \\

\hspace{-0.5em}
\vspace{3mm}
\RuleTwo
{\DDe{\cv} \not\le \DDe{\ecc}
\quad \DDe{\cv} \le \DDe{\ccc}}
{{\sf CausallyConsistent}(\cv)} \qquad

\RuleTwo
{\DDe{\cv} \not\le \DDe{\ccc}
\quad \DDe{\cv} \le \DDe{\scc}}
{{\sf StronglyConsistent}(\cv)}

\end{array}
\end{smathpar}
\vspace{-5mm}

\caption{Contract classification.}
\label{sem:classify}
\end{figure}




\section{Transaction Contracts}

While contracts on individual operations offer the programmer object-level
declarative reasoning, real-world scenarios often involve operations that span
multiple objects. In order to address this problem, several recent
systems~\cite{COPS,BurckhardtESOP,BailisHAT} have proposed a variety of
transactions, with varying semantics, in order to compose operations on
multiple objects. However, given that classical transaction models such as
serializability~\cite{} and snapshot isolation~\cite{} require inter-replica
coordination, these systems espouse \emph{coordination-free transactions} that
remain available under network partitions, but only provide weaker isolation
guarantees. Coordination-free transactions have intricate consistency semantics
and widely varying runtime overheads. As with operation-level consistency, the
onus is on the programmer to pick the correct transaction kind, the choice of
which is complicated by the consistency requirement of the operations it
composes.

\subsection{Syntax Extension}

\name automates the choice of assigning the correct and most efficient
transaction isolation level. Similar to contracts on individual operations, the
programmer associates contracts with transactions, declaratively expressing its
consistency specification. We extend the contract language with a new term
under quantifier-free propositions ${\small \txnZ}~S_1~S_2$, where $S_1$ and
$S_2$ are sets of effects, and introduce a new primitive equivalence relation
$\small \sametxnZ$ that holds for effects from the same transaction. $\small
\txn{a,b}{c,d}$ is just syntactic sugar for $\small \sametxn{a}{b} ~\wedge~
\sametxn{c}{d} ~\wedge~ \neg\sametxn{a}{c}$, where $a$ and $b$ considered to
belong to the \emph{current} transaction. We assume that operations not part of
any transaction to belong to their own unique transaction. While the
transactions may have varying isolation guarantees, we make the standard
assumption that all transactions provide atomicity. Hence, we include the
following axiom in $\Delta$: $\small \forall a,b,c.~\txn{a}{b,c} ~\wedge~
\sameobj{a}{b} ~\wedge~ \sameobj{b}{c} ~\wedge~ \vis{b}{a} \Rightarrow
\vis{c}{a}$.

\subsection{Transactional Bank Account}

In order to illustrate the utility of declarative reasoning for transactions,
let us extend our running bank account example with use two accounts (objects)
-- current ($c$) and savings ($s$). Each account provides operations
\cf{withdraw}, \cf{deposit} and \cf{getBalance}, with the same contracts as
defined previously. We consider two transactions -- \cf{save(amt)}, which
transfers \cf{amt} from current to savings, and \cf{totalBalance} returns
the sum of the balances of individual accounts. The pseudo code for the
transactions is given below:

\noindent \begin{minipage}[t]{0.5\columnwidth}
\begin{codepython}
def save(amt):
  atomically psi1:
    b = c.withdraw(amt)
    if (b): s.deposit(amt)
\end{codepython}
\end{minipage}
\begin{minipage}[t]{0.5\columnwidth}
\begin{codepython}
def totalBalance():
  atomically psi2:
    b1 = c.getBalance()
    b2 = s.getBalance()
    return b1 + b2
\end{codepython}
\end{minipage}

\noindent where $\cv_1$ and $\cv_2$ are the contracts on the corresponding
transactions. Our goal is to ensure that \cf{totalBalance} returns the result
obtained from a consistent snapshot of the object states.

While making both transactions serializable would ensure correctness,
distributed stores rarely offer serializable transactions. Moreover,
serializability is unavailable and hinders scalability. As we will see, these
transactions can be satisfied with much weaker isolation guarantees. Despite
the atomicity offered by the transaction, anomalies are still possible. For
example, the two \cf{getBalance} operations in \cf{totalBalance} transactions
might be served by different replicas with distinct set of committed \cf{save}
transactions. If the first(second) \cf{getBalance} operation witness a
\cf{save} transaction that is not witnessed by the second(first)
\cf{getBalance} operation, then the balance returned will be less(greater) than
the actual balance. It is not immediately apparent which weakest isolation
guaratee will be sufficient to prevent the anomaly.

Instead, \name requires the programmer to simply state the consistency
requirement as a contract. Since we would like both the \cf{getBalance}
operations to witness the same set of \cf{save} transactions, we define
$\psi_2$ as:

\begin{smathpar}
\begin{array}{l}
\cv_{2} = \forall a:\rcf{getBalance}, b:\rcf{getBalance}, (c:\rcf{withdraw} \vee \rcf{deposit}), d. \\
\qquad \txn{a,b}{c,d} ~\wedge~ \vis{d}{b} ~\wedge~ \sameobj{a}{c} \Rightarrow \vis{c}{a}
\end{array}
\end{smathpar}


\section{Implementation}
\label{sec:impl}

\name is implemented as a shallow extension of GHC Haskell and runs on top of
Cassandra, an off-the-shelf eventually consistent distributed data (or backing)
store responsible for all data management issues (i.e., replication, fault
tolerance, availability, and convergence).  Template Haskell is used to
implement static contract classification, and proof obligations are discharged
with the help of the Z3~\cite{Z3} SMT solver. Figure~\ref{fig:impl_mod}
illustrates the overall system architecture.

\begin{figure}
\begin{center}
\includegraphics[width=\columnwidth]{Figures/ImplModel}
\end{center}
\caption{Implementation Model.}
\label{fig:impl_mod}
\end{figure}

Replicated data types and various consistency semantics are implemented and
enforced in the \emph{shim layer}. Our implementation supports eventual,
causal, and strong consistency for data type operations, and RC, MAV, and RR
semantics for transactions.  This functionality is implemented entirely on
top of the standard interface exposed by Cassandra. From an engineering
perspective, leveraging an off-the-shelf data store enables an
implementation comprising roughly only 2500 lines of Haskell code, which is
packaged as a library.

Each new effect in \name is realized as a row insertion in Cassandra, and
the state of an object is the set of all corresponding rows. The shim layer
maintains a causally consistent in-memory snapshot of a subset of objects in
the backing store, by explicitly tracking dependencies introduced between
effects due to visibility, session and same transaction
relations. Dependence tracking is similar to the techniques presented
in~\cite{BoltOn} and~\cite{Eiger}. Because Cassandra provides durability,
convergence, and fault tolerance, each shim layer node simply acts as a
soft-state cache, with no inter-node communication, and can safely be
terminated at any point. Similarly, new shim layer nodes can be spawned on
demand.

The shim layer nodes periodically fetch updates from the backing store for
eventually consistent operations, and on-demand for causally consistent and
strongly consistent operations. Strongly consistent operations are performed
after obtaining exclusive leases on objects. The lease mechanism is
implemented with the help of Cassandra's support for conditional updates and
expiring columns. Cassandra does not provide general-purpose
transactions. Since the transaction guarantees provided by \name are
coordination-free~\cite{BailisHAT}, we realize efficient implementations by
explicitly tracking dependencies between operations and transactions.
Importantly, the weaker isolation semantics of transactions in \name permit
transactions to be discharged if at least one shim layer node is reachable.

We utilize the \cf{summarize} function (\S~\ref{sec:summarize}) to summarize
the object state both in the shim layer node and the backing store, typically
when the number of effects on an object crosses a tunable threshold. Shim layer
summarization is straight-forward; a summarization thread takes the local lock
on the cached object, and replaces its state with the summarized state. The
shim layer node only remains unavailable for that particular object during
summarization (usually a few milliseconds). Performing summarization in the
backing store is more complicated since the whole process needs to be atomic
from a client's perspective, but Cassandra does not provide multi-row
transactions. We have engineered and implemented a scalable summarization
mechanism for the backing store that permits concurrent client operations, but
nonetheless prohibits these operations from witnessing intermediate states of
the summarization process.


\section{Evaluation}
\label{sec:results}

We present an evaluation study of our implementation, report contract
profiles of benchmark programs, and illustrate the performance benefits of
fine-grained consistency classification on operations and transactions. We
also evaluate the impact of the summarization. We have implemented the
following applications, which includes individual RDTs as well as larger
applications composed of several RDTs:

\begin{itemize}[noitemsep]
\item \textbf{LWW register}: A last-write-wins register that provides read
	and write operations, where the read returns the value of the latest write.

\item \textbf{DynamoDB register}: An integer register that allows eventual
  and strong puts and gets, conditional puts, increment and decrement
  operations.

\item \textbf{Bank account}: Our running example.

\item \textbf{Shopping list}: A collaborative shopping list that allows
  concurrent addition and deletion of items.

\item \textbf{Online store}: An online store with shopping cart
  functionality and dynamically changing item prices.  The checkout process
  verifies that the customer only pays the accepted price.

\item \textbf{RUBiS}: An eBay-like auction site~\cite{RUBiS}. The
  application allows users to browse items, bid for items on sale, and pay
  for items from a wallet modeled after a bank account.

\item \textbf{Microblog}: A twitter-like microblogging site, modeled after
  Twissandra~\cite{Twissandra}. The application allows adding new users,
  adding and replying to tweets, following, unfollowing and blocking users,
  and fetching a user's timeline, userline, followers and following.
\end{itemize}

\begin{table}
\setlength{\tabcolsep}{4pt}
{\sffamily \small
\begin{center}
\begin{tabular} {|l|r|r|r|r|r|r|r|r|}
\hline
{\bf Benchmark} & {\bf LOC} & {\bf \#T} & {\bf EC} & {\bf CC} & {\bf SC} & {\bf RC} & {\bf MAV} & {\bf RR} \\
\hline
{LWW Reg} & 108 & 1 & 2 & 2 & 2 & 0 & 0 & 0 \\
{DynamoDB} & 126 & 1 & 3 & 1 & 2 & 0 & 0 & 0 \\
{Bank Account} & 155 & 1 & 1 & 1 & 1 & 1 & 0 & 1 \\
{Shopping List} & 140 & 1 & 2 & 1 & 1 & 0 & 0 & 0 \\
{Online store} & 340 & 4 & 9 & 1 & 0 & 2 & 0 & 1 \\
{RUBiS} & 640 & 6 & 14 & 2 & 1 & 4 & 2 & 0 \\
{Microblog} & 659 & 5 & 13 & 6 & 1 & 6 & 3 & 1 \\
\hline
\end{tabular}
\end{center} }
\caption{The distribution of classified contracts. \#T refers to the number of
tables in the application. The columns 4-6 (7-9) represent operations
(transactions) assigned to this consistency (isolation) level.}
\label{tab:ctrts}
\end{table}

The distribution of contracts in these applications is given in
Table~\ref{tab:ctrts}. We see that majority of the operations and transactions
are classified as eventually consistent and RC, respectively. Operation
contracts are used to enforce integrity and visibility constraints on
individual fields in the tables. Transactions are mainly used to consistently
modify and access related fields across tables. In \name, the contract
classification process is completely performed at compile time and has no
overheads at runtime. The proof obligations associated with contract
classification is discharged through the Z3 SMT Solver. Across our benchmarks,
classifying a contract took 11.5 milliseconds on average.

For our performance evaluation, we deploy \name applications in
\emph{clusters}, where each cluster is composed of five fully replicated
Cassandra replicas within the same datacenter. We instantiate one shim layer
node co-located on the same VM as a Cassandra replica. Clients are instantiated
within the same data center as the store, and run transactions. We deploy each
cluster and client node on an \cf{c3.4xlarge} Amazon EC2 instance. We call this
a \cf{1DC} configuration. For our geo-distributed experiments (\cf{2DC}), we
instantiate 2 clusters, each with five nodes, and place the clusters on US-east
(Virginia) and US-west (Oregon) locations. The average inter-region latency was
85ms.

\begin{figure*}
  \centering
  \subfigure[Bank account]{\label{grf:BA-tp-vs-lat}\includegraphics[width=0.24\textwidth]{graphs/BA-tp-vs-lat.pdf}}
  \subfigure[LWW transactions]{\label{grf:LWW-txn}\includegraphics[width=0.24\textwidth]{graphs/LWW-txn.pdf}}
  \subfigure[RUBiS bidding mix]{\label{grf:rubis}\includegraphics[width=0.24\textwidth]{graphs/Rubis.pdf}}
  \subfigure[Impact of summarization]{\label{grf:summarization}\includegraphics[width=0.235\textwidth]{graphs/summarization.pdf}}
	\caption{Quelea Performance.}
  \label{grf:LWW_perf}
\end{figure*}

Figure~\ref{grf:BA-tp-vs-lat} shows throughput vs. latency of operations in the
bank account example as we increase the number of clients in a \cf{1DC}
configuration. Our client workload was generated using the YCSB
benchmark~\cite{YCSB}. The benchmark uniformly chooses from 100,000 keys, where
the operation spread was 25\% withdraw, 25\% deposit and 50\% getBalance, which
corresponds to the default 50:50 read:write mix in YCSB. We increased the
number of clients from 128 to 1024, and each experiment ran for 180 seconds.

The lines marked EC and CC correspond to all operations (including
\cf{withdraw}) being assigned EC and CC consistency levels. These levels
compromise correctness as \cf{withdraw} has to be an SC operation. The SC line
corresponds to a configuration where all operations are strongly consistent;
this ensures application correctness, at the cost of performance. \name
corresponds to our implementation, which classifies operations based on their
contract specifications. With 512 clients, the \name implementation was within
41\% of the latency and 18\% of the throughput of EC, whereas SC operations had
162\% higher latency and 52\% lower throughput than EC operations. Observe that
there is a point in each case after which the latency increases while the
throughput decreases; these correspond to points where the store becomes
saturated with client requests. In a \cf{2DC} configuration (not shown here),
the average latency of SC operations with 512 clients increased by 9.4$\times$
due to the cost of geo-distributed coordination, whereas \name operations were
only 2.2$\times$ slower, mainly due to the increased cost of \cf{withdraw}
operations. Importantly, the latency of \cf{getBalance} and \cf{deposit}
remained almost the same, illustrating the benefit of fine-grained contract
classification.

We compare the performance of different transaction isolation level choices in
Figure~\ref{grf:LWW-txn} using the LWW register. The numbers were obtained
under a 1DC configuration. The YCSB workload was modified to issue 10
operations per transaction, with a default 50:50 read:write mix. Each operation
is assumed to be eventually consistent. NoTxn corresponds to a configuration
that does not use transactions. Compared to this, RC is only 12\% shower in
terms of latency with 512 clients, whereas RR is 2.3X slower. The difference
between RC and NoTxn is due to the meta-data overhead of recording transaction
information in the object state. For RR transactions, the cost of capturing and
maintaining a snapshot is the biggest source of overhead.

We also compared (not shown) the performance of EC LWW operations directly
against Cassandra, which uses last-writer-wins as the only convergence
semantics. While Cassandra provides no stronger-than-eventual consistency
properties, \name was within 30\%(20\%) of latency(throughput) of Cassandra
with 512 clients, supporting our thesis that programmers only have to incur
relatively low overhead for a more expressive programming model which provides
stronger provable consistency guarantees.

Figure~\ref{grf:rubis} compares the \name implementation of RUBiS in a \cf{1DC}
configuration against a single replica (NoRep) and a strongly replicated
(StrongRep) \cf{1DC} deployment. The benchmark uses the default RUBiS bidding
mix, which has 15\% read-write interactions, which is representative of the
auction workload.  Without replication, NoRep trivially provides strong
consistency. However, this deployment does not scale beyond 1750 operations per
second. Strong replication offers better throughput at the cost of greater
latency due to inter-replica coordination. The \name deployment offers the
benefit of replication, while only paying the cost of coordination when
necessary.

Finally, we study the impact of summarization in
Figure~\ref{grf:summarization}. We use 128 clients and a single \name replica,
with all clients operating on the \emph{same} LWW register to stress test the
summarization mechanism. The shim layer cache (memory) is summarized every 64
updates, while the updates in the backing store (disk) are summarized every
4096 updates. Each point in the graph represents the average latency of the
previous 1000 operations. Each experiment is run for one minute.  Without
summarization, the average latency of operations increases exponentially to
almost one second, and only 13K operations were performed in a minute. Since
every operation has to reduce over the set of all previous operations,
operations take increasingly more time to complete since they must contend with
an ever growing set.  With summarization only in memory, performance still
degrades due to the cost of fetching all previous updates from the backing
store into the shim layer. Fetching the latest updates from the backing store
is essential for SC operations. With summarization enabled on both disk and
memory, latency does not increase over time, and the implementation realizes
throughput of 67K operations/minute.


\section{Related Work}
\label{sec:related}

Operation-based RDTs have been widely studied in terms of their algorithmic
properties~\cite{SSS,Burckhardt2014}, and several systems utilize this model to
construct distributed data structures~\cite{Cassandra,Bayou,Tango}. These
systems typically propose to implement the datatypes directly over a cluster of
nodes, and only focus on basic eventual consistency. Hence, these systems
implement custom solutions for durability and fault-tolerance. \name realizes
RDTs stronger consistency models on top of off-the-shelf eventually consistent
distributed stores. In this respect, \name is similar to~\cite{BoltOn} where
causal consistency is achieved through a shim layer on Cassandra, which
explicitly tracks and enforces dependencies between updates.
However,~\cite{BoltOn} does not support user-defined RDTs, automatic contract
classification and transactions.

Since eventual consistency alone is insufficient to build correct applications,
several systems~\cite{Bayou,Pileus,RedBlue} propose a lattice of stronger
consistency levels. Similarly, traditional database processing
systems~\cite{Berenson95} and their replicated variants~\cite{BailisHAT}
propose weaker isolation levels for performance. In these systems, the onus is
on the developer to choose the correct consistency(isolation) level for
operations(transactions). \name relieves the developer of this burden, and
instead expects programmers to declare application-specific consistency
requirements as first-order formulas. In this regard, \name is similar to Indigo
\cite{indigo}, which requires programmers to declare application-specific
invariants that are expected to be met, along with pre- and post-conditions on
operations as formulas in first-order logic. Indigo then performs a static
analysis on formulas to identify $I$-offender sets - sets of operations,
which, when performed concurrently, result in violation of one or more of the
stated invariants. For each offending set of operation, if programmer chooses 
invariant-violation avoidance over violation repair, the system employs various
techniques, such as escrow reservation, to ensure that the offending set of
operations are never run concurrently.

While serializing operations is sufficient to ensure that the application state
never violates invariants (e.g: \cf{balance $\ge$ 0}), there exist a range of
desirable application-level properties, which cannot be captured in Indigo's
specification language, or for which serialization is an overkill. For instance,
it is neither efficient nor straightforward in Indigo to declare and enforce a
condition to prevent the missing update anomaly
(Figure~\ref{fig:missingUpdateAnomaly}) in the bank account application. The
only way is to encode specifications of all bank account operations such that
every pair of operations are identified as $I$-offending, for some invariant
($I$), resulting in them being serialized. In contrast, \name's contract
language allows for precise encoding of the desired property, and the contract
classification scheme ensures that the it can be enforced without the need to
serialize operations. Furthermore, Indigo also does not distinguish between
various transaction isolation guarantees, which \name does.

Our contract language is inspired by the axiomatic description of RDT semantics
proposed by~\cite{Burckhardt2014}. While they use axioms for formal
verification of correctness of an RDT implementation, we utilize them as a
means for the user to express the desired consistency guarantees in the
application. Similar to their work, our contract language does not incorporate
real (i.e., wall-clock) time. Hence, it cannot describe store semantics such as
recency or bounded-staleness guarantees offered by certain
stores~\cite{Pileus}.

Several conditions have been proposed to judge whether an operation on
a replicated data object needs coordination or not. ~\cite{Calm}
defines \emph{logical monotonicity} as a sufficient condition for
coordination freedom, and proposes a consistency analysis that marks
code regions performing non-monotonic reasoning (eg: aggregations,
such as \cf{COUNT}) as potential coordination points.
~\cite{IConfluence} and ~\cite{Sieve} define \emph{invariant
confluence} and \emph{invariant safety}, respectively, as conditions
for safely executing an operation without coordination. ~\cite{Sieve}
also proposes a program analysis that conservatively marks certain
operations as \emph{blue} (coordination not required), while marking
the remaining as \emph{red} (coordination required). Unlike \name,
these works focus on a coarse-grained classification of consistency as
eventual or strong, and do not focus on finer-grained transaction
isolation levels. However, the analyses they propose relieve
programmers of the burden to tag operations with consistency levels.
Indeed, we do consider automatic inference of consistency contracts
from application-specific integrity constraints as the next step for
\name.


\section{Conclusions}
\label{sec:concl}

In this paper, we have presented \name a shallow Haskell extension for
declarative programming over eventually consistent data stores. The key idea of
\name is the automatic classification of fine-grained consistency contracts on
operations and distributed transactions with respect to the consistency and
isolation levels offered by the store. Our contract language is carefully
crafted from a decidable subset of first-order logic, enabling the use of
automatic theorem prover to discharge the proof obligations associated with
contract classification. We provide meta-theory that certifies the correctness
of contract classification. We realize an instantiation of \name on top of
off-the-shelf distributed store, Cassandra, and illustrate the benefit of
fine-grained contract classification by implementing and evaluating several
scalable applications.


\bibliographystyle{abbrvnat}
\small
\bibliography{all}

\end{document}
