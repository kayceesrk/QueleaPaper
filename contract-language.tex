\section{Contract Language}
\label{sec:contract-lang}

% Contract Language Syntax
% ------------------------
\begin{figure}
\begin{smathpar}
\begin{array}{rclcl}
\multicolumn{5}{l}{
  {x,y,z} \in \mathtt{EffVar} \qquad
  {\cureff} \in \mathtt{CurEff} \qquad
  {\sf Op} \in \mathtt{OperName}
}\\
\cv 		& \in & \mathtt{Contract} 	& \coloneqq & \forall (x : \tau).\cv
        \ALT \pi \\
\tau		& \in	& \mathtt{EffType}	& \coloneqq &  {\sf Op}
        \ALT \tau \vee \tau \\
\pi			&	\in & \mathtt{Prop} & \coloneqq & \true \ALT R(x,y)
        \ALT \pi \vee \pi \\
			  & 		&	 &  \ALT & \pi \wedge \pi \ALT \pi \Rightarrow \pi \\
R				& \in & \mathtt{Relation}	& \coloneqq & \visZ \ALT \soZ
        \ALT \sameobjZ \ALT R^+ \\
				&			&	 &  \ALT & R \cup R \ALT R \cap R \\
\end{array}
\end{smathpar}
\caption{The Contract Language}
\label{fig:contract-lang}
\end{figure}


In this section, we formalize the contract language of \name, and
describe our contract classification scheme, which analyzes a contract
and maps it to the weakest store consistency level sufficient to
satisfy its consistency requirements.

% classifies contracts on the basis of the weakest
% store-level consistency guarantee

\subsection{Syntax}

The syntax of our core contract language is shown in Fig.
~\ref{fig:contract-lang}. The language is based on first-order logic
(FOL), and admits prenex quantification over typed effect variables.
We use a special effect variable ($\cureff$) to denote the effect of
\emph{current operation} - the operation for which a contract is being
written. The type of an effect is simply the label of the operation
(eg: \cf{withdraw}) that engendered the effect. We admit disjuntion in
types to let an effect variable range over effects of multiple
operations. Since each replicated data type comes with a unique set of
operations (therefore, unique set of types for effects), our contract
language is implictly parameterized over a replicated data type
specification.
\GK{I think sec 2. is the right place to introduce the idea of rdt
spec via the bank account example.}

Quantifier-free propositions in our contract language are
conjunctions, disjunctions and implications of predicates expressing
relations between pairs of effect variables. Syntactic class of
relations is seeded with primitive $\visZ$, $\soZ$, and $\sameobjZ$
relations, and also admits derived relations that are expressible as
union, intersection, or transitive closure\footnote{Strictly speaking,
$R^{+}$ is not the transitive closure of $R$, as transitive closure is
not expressible in FOL. Instead, $R^{+}$ in our language denotes
\emph{a} superset of transitive closure of $R$. Formally, $R^{+}$ is
any relation $R'$ such that forall $x$, $y$, and $z$, a) $R(x,y)
\Rightarrow R'(x,y)$, and b) $R'(x,y) \conj R'(y,z) \Rightarrow
R'(x,z)$} of primitive relations. Commonly used derived relations are
the \emph{same object session order} ($\sooZ$), and the the
\emph{happens before order} ($\hboZ$) defined as following:
\begin{smathpar}
\begin{array}{lcl}
{\sf soo} & = &  {\sf so} \cap {\sf sameobj}\\
{\sf hbo} & = & ({\sf soo} \cup {\sf vis})^{+}\\
\end{array}
\end{smathpar}

Note that our contract language (stripped of its syntactic sugar) is a
carefully chosen subset of first-order logic that is known to be
decidable~\cite{epr}. In particular, for any two contracts ($\cv_1$
and $\cv_2$) in our language, the implication check $\cv_1 \Rightarrow
\cv_2$ is decidable, allowing us to automatically deduce that $\cv_2$
is weaker than $\cv_1$. We formally define the \emph{weaker than}
relation among contracts as following: 
\begin{definition}
A contract $\cv_2$ is said to be weaker than $\cv_1$ (denoted $\cv_2
\le \cv_1$ ) if and only if:
\begin{center}
$\Delta \vdash \cv_1 \Rightarrow \cv_2$
\end{center}
\end{definition}
\noindent The context $\Delta$ in the above definition captures assumptions
about the nature of primitive relations, such as the assumption that
$\Rvis$ and $\Rso$ are irreflexive.

This ability to statically compare contracts is key to contract
classification and subsequent enforcement. This is described in the
following subsection.

\subsection{Contract Classification}

Application-level consistency requirements expressed by \name
contracts can have varying impact on the available of the data store
trying to meet the requirements. We identify three major classes of
contracts that differ on this parameter: 
% For instance, total-order requirement
% expressed by the \cf{withdraw} contract (Sec. ~\ref{sec:motivation})
% imposes global coordination among all replicas, and consequently makes
% the store unavailable for any concurrent session. In contrast, the
% trivial (\cf{true}) contract of the \cf{deposit} operation expresses
% no special consistency requirements; any replica can serve a deposit
% request immediately, and the store remains available to concurrent
% sessions. Based on how a contract affects the availability of the data
% store, we classify it into one of the following three categories:

\begin{itemize}
\item \textbf{Unavailable}: These are the contracts that are
satisfiable only under strong consistency. The need for global
coordination to satisfy such contracts makes the store unavailable for
concurrent sessions. An example is the total-order contract on
\cf{withdraw} (Sec. ~\ref{sec:motivation}).
\item \textbf{Sticky-available}: These are the contracts that require
see a \emph{causally consistent} snapshot of the store that
\emph{must} include some or all of the operations from the same
session. By applying all operations of a session to the same logical
replica (colloquially, \emph{sticking} a session to a replica), it is
possible for a store to satisfy such contracts, while remaining highly
available. The \cf{getBalance} contract that is sufficient to prevent
the missing update anomaly is a sticky available contract.
\item \textbf{Highly-available}: Contracts that only require to see
\emph{some} causally consistent snapshop of the store are said to
require \emph{causal visiblity} guarantee. This guarantee can be
achieved by forcing every replica to reveal only those effects whose
preceding effects are already included in the replica. Consequently,
such contracts can always be satisfied by the store, letting it remain
highly available for other requests.
\end{itemize}

It is important to note that, although finer distinctions can be made
between availability implications of contracts, the above
classification scheme is the one that is known to have practical
significance\cite{bailisHAT}. More importantly, the consistency levels
described in the classification scheme, namely strong consistency
({\sf sc}), causal consistency ({\sc cc}), and causal visibility ({\sf
cv}), can be implemented efficiently on most off-the-shelf eventually
consistent data stores\cite{...}. By determining which of the
aformentioned classes a contract falls into, the corresponding
implementation of the consistency level on the data store can be used
to satisfy the contract efficiently. Nonetheless, our classification
scheme is extendible, meaning that a new availability class can be
carved out of existing ones with the advent of new data stores
supporting efficient enforcement of finer consistency levels.

To classify contracts into one of the aforementioned availability
classes, we need to determine if the corresponding consistency
guarantee is sufficient to satisfy the contract. Towards this end, we
encode the {\sf sc}, {\sf cc}, and {\sf cv} consistency guarantees as
formulas in our contract language ($\hboZ$, the happens-before order,
was defined previously in the same section):
\begin{smathpar}
\stretcharraybig
\begin{array}{rcl}
{\sf sc} & = 		 & \forall a. (\hbo{a}{\cureff} \vee \hbo{\cureff}{a} \vee a = \cureff) \\
         & & \conj (\hbo{a}{\cureff} \Rightarrow \vis{a}{\cureff}) \\
				 & & \conj (\hbo{\cureff}{a} \Rightarrow \vis{\cureff}{a}) \\
{\sf cc} & = & \forall a. \hbo{a}{\cureff} \Rightarrow \vis{a}{\cureff} \\
{\sf cv} & = & \forall a,b. \hbo{a}{b} \wedge \vis{b}{\cureff} \Rightarrow \vis{a}{\cureff} \\
\end{array}
\end{smathpar}
Comparing ($\le$) a contract formula against the above formulas of
{\sf sc}, {\sf cc}, and {\sf cv} will let us determine if the contract
can be satisfied under strong consistency, causal consistency, and
causal visiblity consistency guarantees, respectively. Note that {\sf
sc}, {\sf cc} and {\sf cv} are themselves totally ordered with respect
to the $\le$ relation:
\begin{center}
$\cvc \le \ccc \le \scc$
\end{center}
This concurs with the intuition that any contract satisfiable under
$\ccc$ or $\cvc$ is satisfiable under $\scc$ , and any contract that
is satisfiable under $\cvc$ is satisfiable under $\ccc$. Nonetheless,
we determine the availability class of a contract based on the
\emph{weakest} guarantee (among $\cvc$, $\ccc$, and $\scc$) required
to satisfy the contract. The classification scheme is presented
formally as rules in Figure ~\ref{sem:classfiy}.
% Contract classification rules
% ------------------------------
\newcommand{\DDe}[1]{#1}
\begin{figure}
\begin{smathpar}
\begin{array}{c}
\hspace{-0.5em}
\vspace{3mm}
\RuleTwo
{\DDe{\cv} \le \DDe{\scc}}
{{\sf WellFormed}(\cv)}  \qquad

\RuleTwo
{\DDe{\cv} \le \DDe{\ecc}}
{{\sf EventuallyConsistent}(\cv)} \\

\hspace{-0.5em}
\vspace{3mm}
\RuleTwo
{\DDe{\cv} \not\le \DDe{\ecc}
\quad \vdash \DDe{\cv} \le \DDe{\ccc}}
{{\sf CausallyConsistent}(\cv)} \qquad

\RuleTwo
{\DDe{\cv} \not\le \DDe{\ccc}
\quad \DDe{\cv} \le \DDe{\scc}}
{{\sf StronglyConsistent}(\cv)}

\end{array}
\end{smathpar}
\vspace{-5mm}

\caption{Contract classification.}
\label{sem:classify}
\end{figure}

The rules to classify contracts into availability classes are
straightforward. The remaining rule judges a contract as {\sf
WellFormed} if and only if it can be satisfied under strong
consistency. We only consider well-formed contracts for classification
and subsequent enforcement at run-time. The rest are reported as
errors.

As a result of our automatic classification scheme, a store that can
enforce $\scc$, $\ccc$, and $\cvc$ contracts can also enforce any
well-formed \name contract by determining its availability class.  The
following section formally describes operational semantics of one such
store, along with a meta-theoretic result that certifies the soundness
of our classification-based contract enforcement strategy.
