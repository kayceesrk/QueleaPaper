\section{Contract Language}
\label{sec:contract-lang}

% Contract Language Syntax
% ------------------------
\begin{figure}
\begin{smathpar}
\begin{array}{rclcl}
\multicolumn{5}{l}{
  {x,y,z} \in \mathtt{EffVar} \qquad
  {\cureff} \in \mathtt{CurEff} \qquad
  {\sf Op} \in \mathtt{OperName}
}\\
\cv 		& \in & \mathtt{Contract} 	& \coloneqq & \forall (x : \tau).\cv
        \ALT \pi \\
\tau		& \in	& \mathtt{EffType}	& \coloneqq &  {\sf Op}
        \ALT \tau \vee \tau \\
\pi			&	\in & \mathtt{Prop} & \coloneqq & \true \ALT R(x,y)
        \ALT \pi \vee \pi \\
			  & 		&	 &  \ALT & \pi \wedge \pi \ALT \pi \Rightarrow \pi \\
R				& \in & \mathtt{Relation}	& \coloneqq & \visZ \ALT \soZ
        \ALT \sameobjZ \ALT R^+ \\
				&			&	 &  \ALT & R \cup R \ALT R \cap R \\
\end{array}
\end{smathpar}
\caption{The Contract Language}
\label{fig:contract-lang}
\end{figure}


In this section, we formalize the contract language of \name, and describe our
contract classification scheme, which analyzes a contract and maps it to the
weakest store consistency level sufficient to satisfy its consistency
requirements.

% classifies contracts on the basis of the weakest
% store-level consistency guarantee

\subsection{Syntax}

The syntax of our core contract language is shown in Fig.
~\ref{fig:contract-lang}. The language is based on first-order logic (FOL), and
admits prenex quantification over typed effect variables. We use a special
effect variable ($\cureff$) to denote the effect of \emph{current operation} -
the operation for which a contract is being written. The type of an effect is
simply the name of the operation (eg: \cf{withdraw}) that engendered the
effect. We admit disjuntion in types to let an effect variable range over
multiple operation names.

Quantifier-free propositions in our contract language are conjunctions,
disjunctions and implications of predicates expressing relations between pairs
of effect variables. Syntactic class of relations is seeded with primitive
$\visZ$, $\soZ$, and $\sameobjZ$ relations, and also admits derived relations
that are expressible as union, intersection, or transitive
closure\footnote{Strictly speaking, $R^{+}$ is not the transitive closure of
$R$, as transitive closure is not expressible in FOL. Instead, $R^{+}$ in our
language denotes \emph{a} superset of transitive closure of $R$. Formally,
$R^{+}$ is any relation $R'$ such that forall $x$, $y$, and $z$, a) $R(x,y)
\Rightarrow R'(x,y)$, and b) $R'(x,y) \conj R'(y,z) \Rightarrow R'(x,z)$} of
primitive relations. Commonly used derived relations are the \emph{same object
session order} ($\small \sooZ = \soZ ~\cap~ \sameobjZ$), the \emph{happens
before order} ($\small \hbZ = (\soZ ~\cup~ \visZ)^+$), and the per-object
happens-before order ($\small \hboZ = (\sooZ ~\cup~ \visZ)^+$).


\subsection{Capturing Store Semantics}
\label{sec:store_sem}

An important aspect of our contract classification system is that the store
semantics is also captured using the same contract language used to describe
application-level consistency. In this regard, similar
to~\cite{Burckhardt2014}, we can rigorously define a wide variety of store
semantics including those that combine any subset of session and causality
guarantees, and multiple consistency levels. For example, a store that offers
strong consistency is captured by the contract:

\vspace{-1em}
\begin{smathpar}
\scc = \forall a.~\sameobj{a}{\cureff} \Rightarrow \vis{a}{\cureff} ~\vee~ \vis{\cureff}{a} ~\vee~ a = \cureff
\end{smathpar}

Similarly, a store that offers per-object causal consistency is captured by the
contract:

\vspace{-1em}
\begin{smathpar}
\ccc = \forall a.~(\hboZ \cap \sameobjZ) (a,\cureff) \Rightarrow \vis{a}{\cureff}
\end{smathpar}

This ability to represent store semantics and application-level consistency in
the same language is vital to contract classification. Observe that out
contract language does not incorporate real (i.e., wall-clock) time. Hence, the
contract language cannot describe store semantics based on real time such as
recency or bounded-staleness guarantees offered by certain
stores~\cite{Pileus}.

\subsection{Contract Comparison}

Our goal is to classify the contracts, and map the corresponding operation to
the \emph{weakest} store level consistency. To this end, we need to be able to
compare the strengths of contracts. Let $\cv_{op}$ be a contract for a
particular operation $op$, and $\cv_{st}$ capture a particular store
consistency level. We would like to determine whether $op$ can be \emph{safely
discharged} at the store consistency level $\cv_{st}$ such that the resulting
execution does not violate $\cv_{op}$.

Since our contracts represent axiomatic definition of program executions, let
$\Mod{\cv}$ be the set of all executions under which $\cv$ is satisfied. If
every execution $\E \in \Mod{\cv_{st}}$ is also a member of $\Mod{\cv_{op}}$,
then $op$ can be safely discharged under the store consistency level
$\cv_{st}$. Formally, $\Mod{\cv_{st}} \subseteq \Mod{\cv_{op}}$. This is the
model-theoretic consequence relation, written as $\cv_{st} \models_m \cv_{op}$.

Observe that our contract language (stripped of its syntactic sugar) is a
carefully chosen subset of first-order logic that is known to be
decidable~\cite{epr}. Due to the completeness theorem~\cite{completeness},
$\cv_{st} \models_m \cv_{op}$ if and only if $\cv_{st} \Rightarrow \cv_{op}$.
Due to the decidability of our contract language this implication check is
automatically discharged with the help of a theorem prover. In this case, we
say that $\cv_{op}$ is \emph{weaker than} $\cv_{st}$ (written $\cv_{op} \le
\cv_{st}$). Formally,

\begin{definition}
A contract $\cv_{op}$ is said to be weaker than $\cv_{st}$ (written $\cv_{op}
\le \cv_{st}$ ) if and only if $\Delta \vdash \cv_{st} \Rightarrow \cv_{op}$.
\begin{center}
\end{center}
\end{definition}
\vspace{-1em}

\noindent where $\Delta$ captures assumptions about the nature of primitive
relations, such as $\Rvis$ and $\Rso$ are irreflexive, and the happens-before
relation $\hbZ$ is acyclic, preventing thin-air reads. \KC{Ideally, we need to
list all the axioms somewhere.}

\subsection{Contract Classification}

While the store semantics offered by commercial and research data stores vary
widely, for our exposition, we identify three particular consistency levels --
eventual, causal and strong, commonly offered by many distributed store with
tunable consistency, with increasing overhead in terms of latency and
availability. Indeed, the techniques presented here can be extended to other
consistency stratifications in a straight forward manner. We assign each
application-level contract into one of these following classes:


\begin{itemize}
\setlength{\itemsep}{2pt}

\item \textbf{Eventually consistency}: Eventually consistent operations can be
satisfied as long as the client can reach at least one replica. For example,
\cf{deposit} is a highly-available operation. While eventually consistent data
store typically offer \emph{basic} eventual consistency with all possible
anomalies, we assume that our store provides stronger semantics that remains
highly-available~\cite{BailisHAT,COPS}; the store always exposes causal cut of
the updates. This semantics can be formally captured as:

\vspace{-1em}
\begin{smathpar}
\ecc = \forall a,b. ~\hbo{a}{b} \wedge \vis{b}{\cureff} \Rightarrow \vis{a}{\cureff}
\end{smathpar}

\noindent where $\small \hboZ = ((\soZ \cap \sameobjZ) \cup \visZ)^+$.

\item \textbf{Causal consistency}: Operations with causally consistent
contracts are required to see a causally consistent snapshot of the object
state, including the actions performed on the same session.  The latter
requirement entails that if two operations $o_1$ and $o_2$ from the same
session are applied to two different replicas $r_1$ and $r_2$, the second
operation cannot be discharged until the effect of $o_1$ is merged with $r_2$.
We assume that causality is only tracked through operations on the same object;
two operations in the same session but on different objects are considered
causally unrelated under this definition. Stores typically avoid tracking
causaility across objects to mitigate the overhead due to unintended causality.
The corresponding store semantics is captured by the contract $\ccc$ presented
in Section~\ref{sec:store_sem}.

\item \textbf{Strong Consistency}: Strongly consistent operations may be block
indefinitely under network partitions. An example is the total-order contract
on \cf{withdraw} operation. The corresponding store semantics is captured by
the contract $\scc$ presented in Section~\ref{sec:store_sem}.

\end{itemize}

Observe that the contracts $\scc$, $\ccc$ and $\ecc$ are themselves totally
ordered with respect to the $\le$ relation: $\ecc \le \ccc \le \scc$. This
concurs with the intuition that any contract satisfiable under $\ecc$ or $\ccc$
is satisfiable under $\scc$ , and any contract that is satisfiable under $\ecc$
is satisfiable under $\ccc$. Nonetheless, we determine the class of a contract
based on the \emph{weakest} guarantee (among $\ecc$, $\ccc$, and $\scc$)
required to satisfy the contract. The classification scheme is presented
formally as rules in Figure~\ref{sem:classify}.

% Contract classification rules
% ------------------------------
\newcommand{\DDe}[1]{#1}
\begin{figure}
\begin{smathpar}
\begin{array}{c}
\hspace{-0.5em}
\vspace{3mm}
\RuleTwo
{\DDe{\cv} \le \DDe{\scc}}
{{\sf WellFormed}(\cv)}  \qquad

\RuleTwo
{\DDe{\cv} \le \DDe{\ecc}}
{{\sf EventuallyConsistent}(\cv)} \\

\hspace{-0.5em}
\vspace{3mm}
\RuleTwo
{\DDe{\cv} \not\le \DDe{\ecc}
\quad \vdash \DDe{\cv} \le \DDe{\ccc}}
{{\sf CausallyConsistent}(\cv)} \qquad

\RuleTwo
{\DDe{\cv} \not\le \DDe{\ccc}
\quad \DDe{\cv} \le \DDe{\scc}}
{{\sf StronglyConsistent}(\cv)}

\end{array}
\end{smathpar}
\vspace{-5mm}

\caption{Contract classification.}
\label{sem:classify}
\end{figure}


The rules to classify contracts into availability classes are straightforward.
The remaining rule judges a contract as {\sf WellFormed} if and only if it can
be satisfied under strong consistency. The rest are reported as errors.
