\section{Contract Language}
\label{sec:contract-lang}

% Contract Language Syntax
% ------------------------
\begin{figure}
\begin{smathpar}
\begin{array}{rclcl}
\multicolumn{5}{l}{
  {x,y,z} \in \mathtt{EffVar} \qquad
  {\cureff} \in \mathtt{CurEff} \qquad
  {\sf Op} \in \mathtt{OperName}
}\\
\cv 		& \in & \mathtt{Contract} 	& \coloneqq & \forall (x : \tau).\cv
        \ALT \pi \\
\tau		& \in	& \mathtt{EffType}	& \coloneqq &  {\sf Op}
        \ALT \tau \vee \tau \\
\pi			&	\in & \mathtt{Prop} & \coloneqq & \true \ALT R(x,y)
        \ALT \pi \vee \pi \\
			  & 		&	 &  \ALT & \pi \wedge \pi \ALT \pi \Rightarrow \pi \\
R				& \in & \mathtt{Relation}	& \coloneqq & \visZ \ALT \soZ
        \ALT \sameobjZ \ALT R^+ \\
				&			&	 &  \ALT & R \cup R \ALT R \cap R \\
\end{array}
\end{smathpar}
\caption{The Contract Language}
\label{fig:contract-lang}
\end{figure}


In this section, we formalize the contract language of \name, and describe our
contract classification scheme, which analyzes a contract and maps it to the
weakest store consistency level sufficient to satisfy its consistency
requirements.

% classifies contracts on the basis of the weakest
% store-level consistency guarantee

\subsection{Syntax}

The syntax of our core contract language is shown in Fig.
~\ref{fig:contract-lang}. The language is based on first-order logic (FOL), and
admits prenex universal quantification over typed effect variables. We use a
special effect variable ($\cureff$) to denote the effect of \emph{current
operation} - the operation for which a contract is being written. The type of
an effect is simply the name of the operation (eg: \cf{withdraw}) that induced
the effect. We admit disjuntion in types to let an effect variable range over
multiple operation names. The contract $\small \forall (a : \tau_1 \vee
\tau_2).~\psi$ is just syntactic sugar for $\small \forall a. (\oper{a}{\tau_1}
\vee \oper{a}{\tau_2}) \Rightarrow \psi$.

Quantifier-free propositions in our contract language are conjunctions,
disjunctions and implications of predicates expressing relations between pairs
of effect variables. The syntactic class of relations is seeded with primitive
$\visZ$, $\soZ$, and $\sameobjZ$ relations, and also admits derived relations
that are expressible as union, intersection, or transitive
closure\footnote{Strictly speaking, $R^{+}$ is not the transitive closure of
$R$, as transitive closure is not expressible in FOL.  Instead, $R^{+}$ in our
language denotes \emph{a} superset of transitive closure of $R$. Formally,
$R^{+}$ is any relation $R'$ such that forall $x$, $y$, and $z$, a) $R(x,y)
\Rightarrow R'(x,y)$, and b) $R'(x,y) \conj R'(y,z) \Rightarrow R'(x,z)$} of
primitive relations.  Commonly used derived relations are the \emph{same object
session order} ($\small \sooZ = \soZ ~\cap~ \sameobjZ$), and the \emph{same
object happens-before order} ($\small \hboZ = (\sooZ ~\cup~ \visZ)^+$).

% \subsection{Capturing Store Semantics}
% \label{sec:store_sem}

% An important aspect of our contract classification system is that the store
% semantics is also captured using the same contract language used to describe
% application-level consistency. In this regard, similar
% to~\cite{Burckhardt2014}, we can rigorously define a wide variety of store
% semantics including those that combine any subset of session and causality
% guarantees, and multiple consistency levels. For example, a store that offers
% strong consistency is captured by the contract:

% \vspace{-1em}
% \begin{smathpar}
% \scc = \forall a.~\sameobj{a}{\cureff} \Rightarrow \vis{a}{\cureff} ~\vee~ \vis{\cureff}{a} ~\vee~ a = \cureff
% \end{smathpar}

% \noindent Similarly, a store that offers per-object causal consistency is captured by the
% contract:

% \vspace{-1em}
% \begin{smathpar}
% \ccc = \forall a.~(\hboZ \cap \sameobjZ) (a,\cureff) \Rightarrow \vis{a}{\cureff}
% \end{smathpar}

% \noindent This ability to represent store semantics and application-level consistency in
% the same language is vital to contract classification. 

% \subsection{Contract Comparison}

% Our goal is to classify contracts, and map the operation they describe to
% the \emph{weakest} store-level consistency level that  nonetheless
% satisfies the contract's constraints. To this end, we need a mechanism to
% compare the ``strength'' of a contract. Let $\cv_{op}$ be a contract for a
% particular operation $op$, and $\cv_{st}$ capture a particular store
% consistency level. We would like to determine whether $op$ can be
% \emph{safely discharged} at the store consistency level $\cv_{st}$ such that
% the resulting execution does not violate $\cv_{op}$.

% Since our contracts represent axiomatic definition of program executions, let
% $\Mod{\cv}$ be the set of all executions under which contract $\cv$ is
% satisfied. If every execution $\E \in \Mod{\cv_{st}}$ is also a member of
% $\Mod{\cv_{op}}$, then $op$ can be safely discharged under the store
% consistency level $\cv_{st}$. Formally, $\Mod{\cv_{st}} \subseteq
% \Mod{\cv_{op}}$. This is the model-theoretic consequence relation, written as
% $\cv_{st} \models_m \cv_{op}$.

% Observe that our contract language (stripped of its syntactic sugar) is a
% carefully chosen subset of first-order logic that is known to be
% decidable~\cite{epr}.  Since first-order logic is sound and complete
% ~\cite{completeness}, $\cv_{st} \models_m \cv_{op}$ if and only if $\cv_{st}
% \Rightarrow \cv_{op}$.  Due to the decidability of our contract language, this
% implication check is automatically discharged with the help of a theorem
% prover.

\subsection{Capturing Store Semantics}

An important aspect of our contract language is its ability to capture
store-level consistency guarantees, along with application-level
consistency requirements. Similar to~\cite{Burckhardt2014}, we can
rigorously define a wide variety of store semantics including those
that combine any subset of session and causality guarantees, and
multiple consistency levels.  However, for our purposes, we identify
three particular consistency levels -- eventual, causal, and strong,
commonly offered by many distributed stores with tunable consistency,
with increasing overhead in terms of latency and availability. For
each of the these three consistency levels, we capture the semantics
as a formula in our contract language, and informally describe the kind
of application-level consistency requirements that can met under the
consistency level:

% Indeed, the techniques presented here can be
% extended to other consistency stratifications. We assign each
% application-level contract into one of these following classes:

\begin{itemize}
\setlength{\itemsep}{2pt}

\item \textbf{Eventually consistency}: Eventually consistent operations can
  be satisfied as long as the client can reach at least one replica. For
  example, \cf{deposit} is an eventually consistent operation; its semantics
  does not require its action to manifest on all replicas before other
  operations in its session are allowed to proceed. While eventually
  consistent data store typically offer \emph{basic} eventual consistency
  with all possible anomalies, we assume that our store provides stronger
  semantics that remain highly-available~\cite{BailisHAT,COPS}; the store
  always exposes a causal cut of the updates. This semantics can be formally
  captured in terms of the following contract definition:
  \vspace{-0.8em}
  \begin{smathpar}
  \ecc = \forall a,b. ~\hbo{a}{b} \wedge \vis{b}{\cureff} \Rightarrow \vis{a}{\cureff}
  \end{smathpar}
  \noindent where $\small \hboZ = ((\soZ \cap \sameobjZ) \cup \visZ)^+$.

\item \textbf{Causal consistency}: Causally consistent operations
  require to see a causally consistent snapshot of the object state,
  including the actions performed on the same session.  The latter
  requirement entails that if two operations $o_1$ and $o_2$ from the
  same session are applied to two different replicas $r_1$ and $r_2$,
  the second operation cannot be discharged until the effect of $o_1$ is
  merged with $o_2$ in both $r_1$ and $r_2$. The \cf{getBalance}
  operation requires causal consistency, as it requires the operations
  from the same session to be visible, which cannot be guaranteed under
  eventual consistency. We assume that causality is only tracked through
  operations on the same object; two operations in the same session but
  on different objects are considered causally unrelated under this
  definition. Stores typically avoid tracking causality across objects
  to mitigate overheads when causality tracking is unnecessary. The
  corresponding store semantics is captured by the contract $\ccc$
  defined below:
  \vspace{-0.8em}
  \begin{smathpar}
  \ccc = \forall a.~(\hboZ \cap \sameobjZ) (a,\cureff) \Rightarrow \vis{a}{\cureff}
  \end{smathpar}

\item \textbf{Strong Consistency}: Strongly consistent operations may block
  indefinitely under network partitions. An example is the total-order
  contract on \cf{withdraw} operation. The corresponding store semantics is
  captured by the $\scc$ contract definition:
  \vspace{-0.8em}
  \begin{smathpar}
  \scc = \forall a.~\sameobj{a}{\cureff} \Rightarrow \vis{a}{\cureff} ~\vee~ \vis{\cureff}{a} ~\vee~ a = \cureff
  \end{smathpar}

\end{itemize}

Observe that out contract language does not incorporate real (i.e.,
wall-clock) time. Hence, the contract language cannot describe store
semantics based on real time such as recency or bounded-staleness
guarantees offered by certain stores~\cite{Pileus}.


% Contract classification rules
% ------------------------------
\newcommand{\DDe}[1]{#1}
\begin{figure}
\begin{smathpar}
\begin{array}{c}
\hspace{-0.5em}
\vspace{3mm}
\RuleTwo
{\DDe{\cv} \le \DDe{\scc}}
{{\sf WellFormed}(\cv)}  \qquad

\RuleTwo
{\DDe{\cv} \le \DDe{\ecc}}
{{\sf EventuallyConsistent}(\cv)} \\

\hspace{-0.5em}
\vspace{3mm}
\RuleTwo
{\DDe{\cv} \not\le \DDe{\ecc}
\quad \vdash \DDe{\cv} \le \DDe{\ccc}}
{{\sf CausallyConsistent}(\cv)} \qquad

\RuleTwo
{\DDe{\cv} \not\le \DDe{\ccc}
\quad \DDe{\cv} \le \DDe{\scc}}
{{\sf StronglyConsistent}(\cv)}

\end{array}
\end{smathpar}
\vspace{-5mm}

\caption{Contract classification.}
\label{sem:classify}
\end{figure}


\subsection{Contract Comparison and Classification}

Our goal is to map application-level consistency constraints of
operations to appropriate store-level consistency guarantees capable
of satisfying constraints.  The ability to express both of them as
contracts in our contract language lets us compare and determine if
contract ($\cv_{op}$) of an operation ($op$) is weak enough to be
satisfied under a store consistency level identified by the contract
$\cv_{st}$. Towards this end, we define a binary \emph{weaker than}
relation for our contract language as following:
\begin{definition}
A contract $\cv_{op}$ is said to be weaker than $\cv_{st}$ (written $\cv_{op}
\le \cv_{st}$ ) if and only if $\Delta \vdash \cv_{st} \Rightarrow \cv_{op}$.
\begin{center}
\end{center}
\end{definition}

\vspace{-2em} \noindent The $\Delta$ referred in the above defintion captures
assumptions about the nature of primitive relations, such as $\Rvis$
and $\Rso$ are irreflexive, and the happens-before relation $\hbZ$ is
acyclic, preventing thin-air reads. 

If the contract ($\cv_{op}$) of an operation ($op$) is \emph{weaker
than} a store contract ($\cv_{st}$), then constraints expressed by the
former are implied by guarantees provided by the later. Consequently,
$\cv_{op}$ is satisfiable under $\cv_{st}$. Observe that the contracts
$\scc$, $\ccc$ and $\ecc$ are themselves totally ordered with respect
to the $\le$ relation: $\ecc \le \ccc \le \scc$.  This concurs with
the intuition that any contract satisfiable under $\ecc$ or $\ccc$ is
satisfiable under $\scc$, and any contract that is satisfiable under
$\ecc$ is satisfiable under $\ccc$. However, we are interested in the
\emph{weakest} guarantee (among $\ecc$, $\ccc$, and $\scc$) required
to satisfy the contract. We define the corresponding consistency level
as the \emph{consistency class of the contract}. The classification
scheme is presented formally in Figure~\ref{sem:classify}.  

Along with three straightforward rules that classify contracts into
consistency classes, the classification scheme also presents a rule
that judges well-formedness of a contract. A contract is well-formed
if and only if it is satisfiable under $\scc$ - the strongest possible
consistency guarantee any store can provide. Otherwise, it is
considered ill-formed, and rejected statically.
