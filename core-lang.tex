\section{Operational Semantics}
\label{sec:core-opsem}

We demonstrate our ideas using a replicated data store formalized as a
triple $\E;\Theta;\Sigma$, where $\E$ denotes the execution state,
$\Theta$ maps replicas to their local state, and $\Sigma$, the session
soup, is the set of client sessions interacting with the store.
Operational semantics given in Figure~\ref{sem:oper} succinctly
capture the behaviour of the store that is relevant to demonstrate the
operational significance of contract classification. For the sake of
clarity, we only consider a single replicated object of well-defined
type (for eg: a replicated object of type \cf{BankAccount}) in our
formalization.  Our semantics are parametric over the specification of
this replicated data type.  Figure~\ref{sem:oper} formalizes
replicated data type (RDT) specification as tuple
$(\tau,\Ops,\Ctrts)$, where $\tau$ is the data type, $\Ops$ maps
labels ($op$) of operations on $\tau$ to their definitions, while
$\Ctrts$ maps them to their consistency contracts ($\cv$). The
definition of an operation is expected to be a lambda expression in
the server-side language\footnote{Since a replicated object is
maintained by a server, we call the language in which replicated data
type operations are defined as \emph{server-side language}.}, although
we do not enforce this in our formalization. We intentionally leave
the server-side language underspecified to demonstrate that our ideas
are independent of the choice of the programming language used to
implement the replicated data type. 

% Core Operational Semantics
% --------------------------
\begin{figure}
\begin{smathpar}
\begin{array}{lclcl}
\multicolumn{5}{l}{
  {\eff} \in \mathtt{Effect} \qquad
  {op} \in \mathtt{Operation} \qquad
  {\cv} \in \mathtt{Contract}
}\\
\multicolumn{5}{l}{
  \set{\eff} \in \mathtt{Set\; of\; Effects} \qquad
  \tau \in \mathtt{Consistency\; Class}
}\\
%{op} & \in & {\sf Operation}& & \\
%\cv & \in & {\sf Contract} & &\\
\cv(\tau) & \in & \mathtt{Store\; Contract\; of \; \tau} & \coloneqq & \scc,
  \ccc, \ecc\\
%{a_s} & \in & \mathtt{Action} & \coloneqq &  \\
{\sigma} & \in & \mathtt{Session} & \coloneqq & \cdot \ALT (op,\tau); \sigma \\
%\eff & \in & \mathtt{Effect} & \coloneqq &  (s,~i,~op,~v)\\
\EffSoup & \in & \mathtt{EffSoup}	  & \coloneqq & \set{\eff} \\
\visZ, \soZ, &	\in & \mathtt{Relations} & \coloneqq &
  \set{\eff}\times\set{\eff} \\
%	\sameobjZ		&     &  & \\
{\E} 		& \in & \mathtt{ExecState}  & \coloneqq & (\EffSoup,\visZ,\soZ)\\
\Sigma 	& \in & \mathtt{Session\;Soup}   & \coloneqq &
  \langle s,{\sigma} \rangle \pll \Sigma \ALT \emptyset \\
				&			&	\mathtt{Store}		  & \coloneqq & \E;\Sigma \\
\end{array}
\end{smathpar}

\caption{Abstract model of a data store}
\label{sem:oper}
\end{figure}




Clients interact with the store via multiple concurrent sessions. For
technical reasons, we tag every session in the session soup ($\Sigma$)
with a unique {\sf SessID} and the sequence number of next operation
in that session. As shown in Fig.~\ref{sem:oper}, we model a session
simply as a sequence of actions on the replicated object maintained by
the store, where an action is defined as an application of replicated
data type operation to a value. As with the server-side language, our
development is also orthogonal to the choice of the client-side
language, but we require both the languages to agree on the syntactic
class of values that replicated data type operations ($op$) produce
and consume. 

An execution state $\E$ is a tuple $\Exec$ where $\EffSoup$ is a set
of effects, and $\visZ,\soZ \subseteq \EffSoup \times
\EffSoup$ are visibility and session order relations over effects,
respectively. An effect ($\eff$), modeled as a tuple $(s,i,op,v)$,
records the fact that $i^{th}$ action in session with {\sf SessID}
$s$, which is an operation $op$ on the replicated object, has been
successfully executed on some replica yielding a return value $v$.
Note that the combination of $s$ and $i$ uniquely identifies the
effect. Session order relation ($\soZ$) relates effects generated by
the same session. An effect $\eff = (s,i,op,v)$ is said to precede
another effect $\eff' = (s',i',op',v')$ in session order if and only
if $s'=s$ and $i'\ge i$. An effect generated at a replica becomes
visible at rest of the replicas eventually. The local state of a
replica $r$ (i.e., $\Theta(r)$) is a set of effects that are currently
visible at $r$. An operation $op$ performed at replica $r$ can only
witness the set of effects ($\Theta(r)$) visible at $r$.  Note that
in presence of multiple replicated objects, $\visZ$ only relates
effects of operations performed on a single object. For instance, a
{\sf withdraw} operation can witness (via $\visZ$) a {\sf deposit}
operation on the same bank account, but not a {\sf deposit} on a
different account. Often, in our formalization, we use $\visZ$ and
$\soZ$ binary relations to obtain a set of effects visible to a given
effect $\eff$, or set of effects that precede a given effect $\eff$ in
the session order. As a syntactic convenience, whenever $R$ is a
binary relation, we write $R(\eff)$ to denote the set of all $\eff'$
such that $(\eff,\eff') \in R$.  Conversely, we write $R^{-1}(\eff)$
to denote the set of all $\eff'$ such that $(\eff',\eff) \in R$.  

Rule $\rulelabel{Comm}$ captures inter-replica communication. The
basic guarantee provided by the store is causal visibility. That is,
an effect ($\eff$) is made visible at a replica $r$ only after all
the effects that causally precede $\eff$ are made visible at $r$.
Observe that this condition is applicable even for the effects that
originate at replica $r$. 
% It is helpful to think of effects being
% stored in a \emph{write buffer} until causal consistency condition is
% met, at which point they can be merged with the local state
% ($\Theta(r)$). 
It is important to note that that enforcing causal visibility does not
require any inter-replica coordination.

Rule $\rulelabel{Oper}$ is an auxiliary reduction of the form

\begin{smathpar}
\auxred{\Theta} {(\E,\langle s,i,op(v) \rangle)} {r} {(\E',\eff)}
\end{smathpar}

\noindent Under the store configuration $\Theta$, the rule captures
the progress in execution (from $\E$ to $\E'$) due to the application
of operation $op$ to replica $r$ resulting in a new effect $\eff$.
The rule first constructs a \emph{context} for the application from
the local state ($\Theta(r)$) of the replica, by
projecting\footnote{{\textsf{ctxt*}} is auxiliary function
\textsf{ctxt} extended straightforwardly to set of effects} relevant
information from effects in $\Theta(r)$. It then substitutes the
definition ($\Ops(op)$) of the operation for its label ($op$), and
relies on the reduction relation ($\rdtredsto$) of the server-side
language to reduce the application $\Ops(op)(v,~ctxt)$ to a value
$v'$.  Subsequently, the the attributes of execution state, namely
$\EffSoup$, $\visZ$ and $\soZ$, are extended to incorporate
the new effect ($\eff$). 

If the operation $op$ is highly-available (rule $\rulelabel{HA}$), we
simply apply the operation to any replica $r$. Since the store
provides causal visibility, highly available operations are
satisfiable under any replica. If the operation is sticky-available
(rule $\rulelabel{SA}$), the operation can be applied to a replica $r$
only if it already contains the effects of all the previous operations
from the same session. This guarantee can be satisfied by applying all
operations from the same session to the same logical copy of the
database. If such a logical copy is untenable, then the operation
might block. Since the store is assumed to converge eventually, the
blocked sticky-available operation guaranteed to unblock eventually.

An unavailable operation expects strong (sequential) consistency.
That is, universe of all effects ($\EffSoup$) in an execution ($\E$)
must be partitionable into a set of effects that \emph{happened
before} $\eff$ and another set that \emph{happened after} $\eff$,
where $\eff$ is the effect generated by an unavailable operation. The
rule $\rulelabel{UA}$ enforces this sequencing in two steps; firstly,
it insists that the the unavailable operation ($op$) witness effects
of all operations executed so far by requiring the global set of
effects $\EffSoup$ to be a subset of local state ($\Theta(r)$) of the
replica ($r$) executing $op$. Secondly, the rule requires the effect
($\eff$) generated by $op$ to be added to the local state of every
other replica in the store, so that further operations on these
replicas can witness the effect of $op$. Since both these steps
require global coordination among replicas, the store is
\emph{unavailable} during the time it is executing $op$. However, it
must be noted that executing unavailable operation does not entail the
convergence of local states of all replicas; it simply means that
there is one replica ($r$) whose local state is the upper bound of
local states of all replicas, and there exists a set containing
atleast one effect ($\eff$), which is their lower bound.

\subsection{Metatheory}

To help us state our main meta-theoretic result, we first define a
well-formedness condition over execution states:

\begin{definition}[Well-formed execution]
An execution state $\E = \Exec$ corresponding to an execution of
operations over a replicated datatype with specification $\rdtspec =
(\tau,\Ops,\Ctrts)$ is said to be well-formed if and only if 
$\forall \eff \in \EffSoup. ~\E \models \De{\cv[\eff/\cureff]}$, where $\cv =
\Ctrts(\operZ(\eff))$.
\end{definition}

Symbol $\models$ denotes the standard \emph{models} relation of
first-order logic.  A valid model for a first-order formula defines
the universe of discourse, and provides interpretations for free
symbols in the formula such that the formula is true under given
interpretations.  An execution state ($\E$) provides interpretations
for $\Rvis$ and $\Rso$ relations, while fixing the universe of
discourse to be the set of all generated effects ($\EffSoup$).
Intuitively, the well-formedness definition captures the idea that for
every generated effect, the contract on the corresponding operation is
satisfied under the given execution state. 

Our main meta-theoretic result guarantees the soundness of our
operational semantics in enforcing all contractual obligations by
ensuring that every reduction step preserves the well-formedness of
execution state.

\begin{theorem}[Well-formedness preservation]
\label{lem:core-preservation}
If $\E$ is well-formed and $\E; \Theta; \Sigma \goesto \E'; \Theta';
\Sigma'$, then $\E'$ is well-formed.
\end{theorem}
% Lemmas
%\begin{theorem}[Well-formedness preservation]
  \label{lem:core-preservation}
  If $\E$ is well-formed and $\E; \Theta; \Sigma \goesto \E'; \Theta';
  \Sigma'$, then $\E'$ is well-formed.
  \end{theorem}
\begin{proof}
  Let $\E = (\EffSoup,\visZ,\soZ)$ and $\E'=(\EffSoup',\Rvis',\Rso')$.
  We prove by case analysis on the derivation $\E; \Theta; \Sigma
    \goesto \E'; \Theta'; \Sigma'$. Cases:
    \begin{itemize}
      \item Case \rulelabel{Comm}: The rule does not change the
      execution state (i.e., $\E' = \E$); so, well-formedness is
      trivially preserved.

      \item Case \rulelabel{HA}: $\Sigma = \tuplee{s,~op(v); \sigma} \pll
      \Sigma_0$, and $\Sigma' = \tuplee{s,\sigma} \pll \Sigma_0$,
      where $\Sigma_0$ represents reset of the session soup. From the
      definition of well-formedness, we have:
      \begin{smathpar}
      \begin{array}{cr}
        \forall a\in\EffSoup. ~\msentails{E}{\Ctrts(a)} & H\npp\\
      \end{array}
      \end{smathpar}
      Let us call $\Ctrts(op)$ as $\cv$. We have the following hypotheses:
      \begin{smathpar}
      \begin{array}{cr}
        {\sf HighlyAvailable}(\cv) & H\npp \\
        \E;\Theta;\tuplee{s,op(v); \sigma}
          \;\xhookrightarrow{\eff,r}\;
        \E';\Theta';\tuplee{s,\sigma} & H\npp\\
      \end{array}
      \end{smathpar}
      The last hypothesis indicates that $op(v)$ is executed on
      replica $r$ yielding effect $\eff$. Inversion on $H2$ gives:
      \begin{smathpar}
      \begin{array}{cr}
        \EffSoup' = \EffSoup \cup \{\eff\} & H\npp\\
        \visZ' = \Theta(r)\times\eff ~\cup~ \visZ & H\npp\\
        \Rso' = \EffSoup_{{\sf SessID}=s,\,{\sf SeqNo}<i}\times\eff ~\cup~ \Rso & H\npp\\
      \end{array}
      \end{smathpar}
      From Hypotheses $H0$ and $H3$, we deduce that in order to prove:
      \begin{smathpar}
      \begin{array}{cr}
        \forall a\in\EffSoup'. ~\msentails{E'}{\Ctrts(a)} & \\
      \end{array}
      \end{smathpar}
      It suffices to prove the following goal:
      \begin{smathpar}
      \begin{array}{cr}
        \msentails{E'}{\cv} & G\mpp\\
      \end{array}
      \end{smathpar}
      By destructing $H1$, we get:
      \begin{smathpar}
      \begin{array}{cr}
        \hasTyp{}{\De{\cvc \Rightarrow \cv}} & H\npp\\
      \end{array}
      \end{smathpar}
      Applying (\emph{eapply}) lemma~\ref{lem:1} using $H6$ with
      $E:=E'$, we deduce that to prove $G0$, it is sufficient to
      prove:
      \begin{smathpar}
      \begin{array}{cr}
        \msentails{E'}{\De{\cvc}} & G\mpp\\
      \end{array}
      \end{smathpar}
      $G1$ becomes our new goal. Expanding the definition of $\cvc$:
      \begin{smathpar}
      \begin{array}{cr}
        \msentails{E'}{\De{\forall a,b. \hbo{a}{b} \wedge \vis{b}{\eff} \Rightarrow
          \vis{a}{\eff}}} & \\
      \end{array}
      \end{smathpar}
      By definition $\De{\forall a.\psi} = \forall a.
      \De{\psi}$ and  $\De{\psi_1 \Rightarrow \psi_2} = \De{\psi_1}
      \Rightarrow \De{\psi_2}$. Further, the model $\E'$ defines
      $\EffSoup'$ as the universe of values. Therefore, the goal can
      be rewritten:
      \begin{smathpar}
      \begin{array}{cr}
        \forall (a,b\in\EffSoup').
        \msentails{E'}{\De{\hbo{a}{b} \wedge
        \vis{b}{\eff}} \Rightarrow
          \De{\vis{a}{\eff}}} & G\mpp\\
      \end{array}
      \end{smathpar}
      New hypotheses after \emph{intros}:
      \begin{smathpar}
      \begin{array}{cr}
        a \in \EffSoup' & H\npp\\
        b \in \EffSoup' & H\npp\\
      \end{array}
      \end{smathpar}
      And new goal:
      \begin{smathpar}
      \begin{array}{cr}
        \msentails{E'}{\De{\hbo{a}{b} \wedge
        \vis{b}{\eff}} \Rightarrow
          \De{\vis{a}{\eff}}} & G\mpp\\
      \end{array}
      \end{smathpar}
      From the definition of \emph{models} relation of
      first-order logic (i.e., semantics of first-order logic),
      \emph{inversion} on $G3$ adds following hypothesis:
      \begin{smathpar}
      \begin{array}{cr}
        \msentails{\E'}{\De{ \hbo{a}{b} \wedge
        \vis{b}{\eff}}}  & H\npp\\
      \end{array}
      \end{smathpar}
      New goal following the inversion:
      \begin{smathpar}
      \begin{array}{cr}
          \msentails{\E'}{\De{\vis{a}{\eff}}}& G\mpp\\
      \end{array}
      \end{smathpar}
      From the definition of $\De{.}$, $H7$ expands to:
      \begin{smathpar}
      \begin{array}{cr}
          (\msentails{E'}{\hbo{a}{b}}) \wedge
          (\msentails{E'}{\Rvis'(b,\eff)}) & H\npp\\
      \end{array}
      \end{smathpar}
      \E' provides interpretation for $\Rvis$ as $\Rvis'$, and $\Rso$
      as $\Rso'$. Therefore, inverting $H10$ gives:
      \begin{smathpar}
      \begin{array}{cr}
          {{\sf hbo'}(a,b)} \wedge \Rvis'(b,\eff) & H\npp\\
      \end{array}
      \end{smathpar}
      In $H11$, ${\sf hbo'} = (\Rso' \cup \Rvis')^{+}$. Similarly,
      inverting the goal $G4$:
      \begin{smathpar}
      \begin{array}{cr}
        \Rvis'(a,\eff) & G\mpp\\
      \end{array}
      \end{smathpar}
      Inversion on $H11$:
      \begin{smathpar}
      \begin{array}{cr}
          {{\sf hbo'}(a,b)} & H\npp\\
          \Rvis'(b,\eff) & H\npp\\
      \end{array}
      \end{smathpar}
      Since $\eff$ is unique, from $H13$ and $H4$ we have the following:
      \begin{smathpar}
      \begin{array}{cr}
            b\in\Theta(r)& H\npp\\
      \end{array}
      \end{smathpar}
      Applying lemma~\ref{lem:local-state-is-cc} using $H12$ and $H14$
      with $\Rvis=\Rvis'$ and $\Rso=\Rso'$, we derive:
      \begin{smathpar}
      \begin{array}{cr}
        a \in \Theta(r) & H\npp\\
      \end{array}
      \end{smathpar}
      Now, from $H4$ and $H15$, we deduce:
      \begin{smathpar}
      \begin{array}{cr}
        (a,\eff) \in \Rvis'& \\
      \end{array}
      \end{smathpar}
      which is what needs to be proven ($G5$).
      \item Case \rulelabel{SA}: $\Sigma = \tuplee{s,~op(v); \sigma} \pll
      \Sigma_0$, and $\Sigma' = \tuplee{s,\sigma} \pll \Sigma_0$. From the definition of
      well-formedness, we have:
      \begin{smathpar}
      \begin{array}{cr}
        \forall a\in\EffSoup. ~\msentails{E}{\Ctrts(a)} & H\npp\\
      \end{array}
      \end{smathpar}
      Let us call $\Ctrts(op)$ as $\cv$. We have the following hypotheses:
      \begin{smathpar}
      \begin{array}{cr}
        {\sf StickyAvailable}(\cv) & H\npp \\
        \E;\Theta;\tuplee{s,op(v); \sigma}
          \;\xhookrightarrow{\eff,r}\;
        \E';\Theta';\tuplee{s,\sigma} & H\npp\\
        \E'.\soZ^{-1}(\eff) \subseteq \Theta(r) & H\npp\\
      \end{array}
      \end{smathpar}
      $H18$ indicates that $op(v)$ is executed on
      replica $r$ yielding effect $\eff$. Inversion on $H18$ gives:
      \begin{smathpar}
      \begin{array}{cr}
        \EffSoup' = \EffSoup \cup \{\eff\} & H\npp\\
        \visZ' = \Theta(r)\times\eff ~\cup~ \visZ & H\npp\\
        \Rso' = \EffSoup_{{\sf SessID}=s,\,{\sf SeqNo}<i}\times\eff ~\cup~ \Rso & H\npp\\
      \end{array}
      \end{smathpar}
      From $H19$ and $H22$, we have:
      \begin{smathpar}
      \begin{array}{cr}
        \EffSoup_{{\sf SessID}=s,\,{\sf SeqNo}<i} \in \Theta(r) &
        H\npp \\
      \end{array}
      \end{smathpar}
      From $H16$ and $H20$, we know that in order to prove:
      \begin{smathpar}
      \begin{array}{cr}
        \forall a\in\EffSoup'. ~\msentails{E'}{\Ctrts(a)} & \\
      \end{array}
      \end{smathpar}
      It suffices to prove the following goal:
      \begin{smathpar}
      \begin{array}{cr}
        \msentails{E'}{\cv} & G\mpp\\
      \end{array}
      \end{smathpar}
      By destructing $H17$, we get:
      \begin{smathpar}
      \begin{array}{cr}
        \hasTyp{}{\De{\ccc \Rightarrow \cv}} & H\npp\\
      \end{array}
      \end{smathpar}
      Applying (\emph{eapply}) lemma~\ref{lem:1} using $H24$ with
      $E:=E'$, we deduce that to prove $G0$, it is sufficient to
      prove:
      \begin{smathpar}
      \begin{array}{cr}
        \msentails{E'}{\De{\ccc}} & G\mpp\\
      \end{array}
      \end{smathpar}
      $G7$ becomes our new goal. Expanding the definition of $\ccc$:
      \begin{smathpar}
      \begin{array}{cr}
        \msentails{E'}{\De{\forall a. \hbo{a}{\eff} \Rightarrow
        \vis{a}{\eff}}} & \\
      \end{array}
      \end{smathpar}
      By definition $\De{\forall a.\psi} = \forall a.
      \De{\psi}$ and  $\De{\psi_1 \Rightarrow \psi_2} = \De{\psi_1}
      \Rightarrow \De{\psi_2}$. Further, the model $\E'$ defines
      $\EffSoup'$ as the universe of values. Therefore, the goal can
      be rewritten:
      \begin{smathpar}
      \begin{array}{cr}
        \forall (a,b\in\EffSoup').
        \msentails{E'}{\De{\hbo{a}{\eff}} \Rightarrow
          \De{\vis{a}{\eff}}} & G\mpp\\
      \end{array}
      \end{smathpar}
      New hypotheses after \emph{intros}:
      \begin{smathpar}
      \begin{array}{cr}
        a \in \EffSoup' & H\npp\\
      \end{array}
      \end{smathpar}
      And new goal:
      \begin{smathpar}
      \begin{array}{cr}
        \msentails{E'}{\De{\hbo{a}{\eff}} \Rightarrow
          \De{\vis{a}{\eff}}} & G\mpp\\
      \end{array}
      \end{smathpar}
      From the definition of \emph{models} relation of
      first-order logic (i.e., semantics of first-order logic),
      \emph{inversion} on $G9$ adds following hypothesis:
      \begin{smathpar}
      \begin{array}{cr}
        \msentails{\E'}{\De{ \hbo{a}{\eff}}}  & H\npp\\
      \end{array}
      \end{smathpar}
      New goal following the inversion:
      \begin{smathpar}
      \begin{array}{cr}
          \msentails{\E'}{\De{\vis{a}{\eff}}}& G\mpp\\
      \end{array}
      \end{smathpar}
      By inversion on $H26$:
      \begin{smathpar}
      \begin{array}{cr}
          {{\sf hbo'}(a,\eff)}& H\npp\\
      \end{array}
      \end{smathpar}
      where ${\sf hbo'} = (\Rso' \cup \Rvis')^{+}$. Similarly,
      inverting the goal $G4$, we get the current goal:
      \begin{smathpar}
      \begin{array}{cr}
        \Rvis'(a,\eff) & G\mpp\\
      \end{array}
      \end{smathpar}
      Inverting $H27$, we get two cases:
      \begin{itemize}
        \item SCase 1: Hypotheses:
        \begin{smathpar}
        \begin{array}{cr}
          (\Rso' \cup \Rvis')(a,\eff) & H\npp\\
        \end{array}
        \end{smathpar}
        Inversion on $H28$ leads to two subcases. In one case, we
        assume $\Rvis'(a,\eff)$ and try to prove the goal $G11$. The
        proof for this case mimics the proof for Case \rulelabel{HA}.
        Alternatively, in second case, we assume:
        \begin{smathpar}
        \begin{array}{cr}
          \Rso'(a,\eff) & H\npp\\
        \end{array}
        \end{smathpar}
        and prove $G11$. From $H29$ and $H19$, we infer:
        \begin{smathpar}
        \begin{array}{cr}
          a \in \Theta(r) & H\npp\\
        \end{array}
        \end{smathpar}
        Now, from $H30$ and $H21$ we know:
        \begin{smathpar}
        \begin{array}{cr}
          (a,\eff) \in \Rvis' & \\
        \end{array}
        \end{smathpar}
        from which the goal ($G11$) follows.

        \item SCase 2: Hypotheses (after abbreviating the occurance
        of$(\Rso' \cup \Rvis')^{+}$ as {\sf hbo'}):
        \begin{smathpar}
        \begin{array}{cr}
          \exists(c\in\EffSoup').{\sf hbo'}(a,c) \wedge (\Rso' \cup
          \Rvis')(c,\eff) & H\npp\\
        \end{array}
        \end{smathpar}
        Inverting $H31$, followed by expanding $\Rso' \cup \Rvis'$:
        \begin{smathpar}
        \begin{array}{cr}
          c\in\EffSoup' & H\npp\\
           {\sf hbo'}(a,c) \wedge (\Rso'(c,\eff) \vee \Rvis'(c,\eff))& H\npp\\
        \end{array}
        \end{smathpar}
        Inverting the disjunction in $H33$, we get two cases:
        \begin{itemize}
          \item SSCase R: Hypothesis is
          \begin{smathpar}
          \begin{array}{cr}
             {\sf hbo'}(a,c) \wedge \Rvis'(c,\eff)& H\npp\\
          \end{array}
          \end{smathpar}
          Observe that hypothesis $H34$ and current goal ($G11$) are
          same as hypothesis $H11$ and goal ($G5$) in Case
          \rulelabel{HA}. The proof for this SSCase is also the same.

          \item SSCase L: Hypothesis is
          \begin{smathpar}
          \begin{array}{cr}
             {\sf hbo'}(a,c) \wedge \Rso'(c,\eff)& H\npp\\
          \end{array}
          \end{smathpar}
          Inverting $H35$:
          \begin{smathpar}
          \begin{array}{cr}
             {\sf hbo'}(a,c)& H\npp\\
             \Rso'(c,\eff)& H\npp\\
          \end{array}
          \end{smathpar}
          From $H37$ and $H19$, we infer:
          \begin{smathpar}
          \begin{array}{cr}
            c \in \Theta(r) & H\npp \\
          \end{array}
          \end{smathpar}
          Applying lemma~\ref{lem:local-state-is-cc} using $H36$ and
          $H38$ with $\Rso=\Rso'$ and $\Rvis=\Rvis'$, we get:
          \begin{smathpar}
          \begin{array}{cr}
            a \in \Theta(r) & H\npp\\
          \end{array}
          \end{smathpar}
          Proof follows from $H39$ and $H21$.
        % End of SSCase
        \end{itemize}
      % End of SCases
      \end{itemize}

      \item Case \rulelabel{UA}: $\Sigma = \tuplee{s,~op(v); \sigma} \pll
      \Sigma_0$, and $\Sigma' = \tuplee{s,\sigma} \pll \Sigma_0$. From the definition of
      well-formedness, we have:
      \begin{smathpar}
      \begin{array}{cr}
        \forall a\in\EffSoup. ~\msentails{E}{\Ctrts(a)} & H\npp\\
      \end{array}
      \end{smathpar}
      Let us call $\Ctrts(op)$ as $\cv$. We have the following hypotheses:
      \begin{smathpar}
      \begin{array}{cr}
        {\sf Unavailable}(\cv) & H\npp \\
        \E;\Theta;\tuplee{s,op(v); \sigma}
          \;\xhookrightarrow{\eff,r}\;
        \E';\Theta_0;\tuplee{s,\sigma} & H\npp\\
        \E.A \subseteq \Theta(r) & H\npp\\
        \forall r'\in \dom(\Theta_0). \Theta'(r') = \Theta_0(r') \cup
          \{\eta\} & H\npp\\
      \end{array}
      \end{smathpar}
      Inversion on $H42$ gives:
      \begin{smathpar}
      \begin{array}{cr}
        \EffSoup' = \EffSoup \cup \{\eff\} & H\npp\\
        \visZ' = \Theta(r)\times\{\eff\} ~\cup~ \visZ & H\npp\\
        \Rso' = \EffSoup_{{\sf SessID}=s,\,{\sf SeqNo}<i}\times\eff ~\cup~ \Rso & H\npp\\
      \end{array}
      \end{smathpar}
      From Hypotheses $H0$ and $H3$, we deduce that in order to prove:
      \begin{smathpar}
      \begin{array}{cr}
        \forall a\in\EffSoup'. ~\msentails{E'}{\Ctrts(a)} & \\
      \end{array}
      \end{smathpar}
      It suffices to prove the following goal:
      \begin{smathpar}
      \begin{array}{cr}
        \msentails{E'}{\cv} & G\mpp\\
      \end{array}
      \end{smathpar}
      By destructing $H1$, we get:
      \begin{smathpar}
      \begin{array}{cr}
        \hasTyp{}{\De{\scc \Rightarrow \cv}} & H\npp\\
      \end{array}
      \end{smathpar}
      Applying (\emph{eapply}) lemma~\ref{lem:1} using $H6$ with
      $E:=E'$, we deduce that to prove $G0$, it is sufficient to
      prove:
      \begin{smathpar}
      \begin{array}{cr}
        \msentails{E'}{\De{\scc}} & G\mpp\\
      \end{array}
      \end{smathpar}
      $G1$ becomes our new goal. Expanding the definition of $\scc$:
      \begin{smathpar}
      \begin{array}{cr}
        \msentails{E'}{\forall a. \De{\phi_1 \wedge \phi_2 \wedge \phi_3}} &
        G\mpp\\
      \end{array}
      \end{smathpar}
      where $\phi_1, \phi_2$ and $\phi_3$ are defined as:
      \begin{smathpar}
      \begin{array}{cr}
        \phi_1 =  (\hbo{a}{\eff} \vee \hbo{\eff}{a} \vee a =
        \eff) & H\npp\\
        \phi_2 = (\hbo{a}{\eff} \Rightarrow \vis{a}{\eff}) & H\npp\\
        \phi_3 = (\hbo{\eff}{a} \Rightarrow \vis{\eff}{a})& H\npp\\
      \end{array}
      \end{smathpar}
      After inversion followed by \emph{intros}, new hypothesis:
      \begin{smathpar}
      \begin{array}{cr}
        a\in\EffSoup' & H\npp\\
      \end{array}
      \end{smathpar}
      New goal:
      \begin{smathpar}
      \begin{array}{cr}
        \msentails{E'}{\De{\phi_1 \wedge \phi_2 \wedge \phi_3}} & G\mpp\\
      \end{array}
      \end{smathpar}
      Inverting \emph{models} relation in $G15$ followed by inverting
      conjunction, we get three goals:
      \begin{smathpar}
      \begin{array}{cr}
        \msentails{E'}{\De{\phi_1 }} & G\mpp\\
        \msentails{E'}{\De{\phi_2 }} & G\mpp\\
        \msentails{E'}{\De{\phi_3 }} & G\mpp\\
      \end{array}
      \end{smathpar}
      Substituting definitions of $\phi_1$, $\phi_2$ and $\phi_3$ in
      $G16-18$ and expanding:
      \begin{smathpar}
      \begin{array}{cr}
         {\sf hbo'}(a,\eff) \vee {\sf hbo'}(\eff,a)  \vee a = \eff & G\mpp\\
         {\sf hbo'}(a,\eff) \Rightarrow \Rvis'(a,\eff) & G\mpp\\
         {\sf hbo'}(\eff,a) \Rightarrow \Rvis'(\eff,a) & G\mpp\\
      \end{array}
      \end{smathpar}
      From $H45$ and $H52$, we know that either $a\in\EffSoup$ or
      $a=\eff$. When $a=\eff$:
      \begin{itemize}
        \item $G19$ follows trivially.
        \item Since ${\sf hbo'}$ is irreflexive, ${\sf
        hbo'}(\eff,\eff) = false$. Therefore, $G20$ and $G21$ are valid vacuosly.
      \end{itemize}
      When $a \in A$:
      \begin{itemize}
        \item From $H43$ we know that $a \in \Theta(r)$. Using $H46$, we
        derive:
        \begin{smathpar}
        \begin{array}{cr}
          \Rvis'(a,\eff) & H\npp\\
        \end{array}
        \end{smathpar}
        which proves $G20$

        \item We prove $G19$ by proving:
        \begin{smathpar}
        \begin{array}{cr}
          {\sf hbo'}(a,\eff)& G\mpp\\
        \end{array}
        \end{smathpar}
        From previous case, we know that:
        \begin{smathpar}
        \begin{array}{cr}
          \Rvis'(a,\eff) & H\npp\\
        \end{array}
        \end{smathpar}
        Introducting disjunction:
        \begin{smathpar}
        \begin{array}{cr}
          (\Rvis' \cup \Rso')(a,\eff) & H\npp\\
        \end{array}
        \end{smathpar}
        Now, since ${\sf hbo'} = (\Rvis' \cup \Rso')^{+}$, proof
        follows from $H54$.

        %% AXIOMS ON VIS, SO & HBO NEED TO BE PROVEN FOR VIS', SO' &
        %% HBO'.
        \item From previous case, we know that ${\sf hbo'}(a,\eff)$.
        Since ${\sf hbo'}$ is asymmetric, ${\sf hbo'}(\eff,a) =
        false$. Hence, $G21$ follows vacuosly.
      \end{itemize}
    % End of main cases
    \end{itemize}
  \hfill \qed
\end{proof}

\begin{lemma}[Encoding Implication]
  \label{lem:1}
  Forall contracts $\psi_1$ and $\psi_2$, if $\De{\psi_1 \Rightarrow
  \psi_2}$ is a valid formula in first-order logic (i.e.,
  $\hasTyp{\cdot} {\De{\psi_1 \Rightarrow \psi_2}}$), then any
  execution state ($\E$) that is a model of $\De{\psi_1}$ is also the
  model of $\De{\psi_2}$ (i.e., $(\msentails{\E}{\De{\psi_1}})
  \Rightarrow (\msentails{\E}{\De{\psi_2}})$).
\end{lemma}
\begin{proof}
  By definition, $\De{\psi_1 \Rightarrow \psi_2} = \De{\psi_1}
  \Rightarrow \De{\psi_2}$. Since $\De{\phi_1}$ and $\De{\phi_2}$ are
  formulas in first-order logic, the proof now follows directly from
  the soundness result of first-order logic.
\hfill \qed
\end{proof}

\begin{lemma}[Local State is Causally Consistent]
  \label{lem:local-state-is-cc}
  In a replicated data store $\E; \Theta; \Sigma$, local state of
  every replica $r$ in the store (i.e., $\Theta(r)$) is causally
  consistent. That is:\\
  $\forall (r\in{\sf dom}(\Theta)). \forall (\eff \in \Theta(r)). \\
  \hspace*{0.8in}\forall (a\in\E.A). \hbo{a}{\eff} \Rightarrow a \in \Theta(r)$\\
  where ${\sf hbo} = (\E.\Rvis \cup \E.\Rso)^{+}$
\end{lemma}
\begin{proof}
  By induction on $(\E; \Theta; \Sigma)$. Cases:
  \begin{itemize}
    \item Case \rulelabel{Comm}: Hypotheses:
    \begin{smathpar}
    \begin{array}{cr}
      r \in {\sf ReplID} & H\npp \\
      \eff \in A & H\npp \\
      \eff \notin \Theta(r)& H\npp \\
      \E.\visZ^{-1}(\eff) \cup \E.\soZ^{-1}(\eff) \subseteq \Theta(r)
      & H\npp \\
      \Theta = \Theta_0[r \mapsto \{\eff\} \cup \Theta_0(r)] & H\npp \\
    \end{array}
    \end{smathpar}
    Inductive Hypothesis:
    \begin{smathpar}
    \begin{array}{cr}
      \hspace*{-0.5in}\forall (r\in{\sf dom}(\Theta_0)). \forall (\eff \in
        \Theta_0(r)). & \\
      \hspace*{0.3in}\forall (a\in\E.A). \hbo{a}{\eff} \Rightarrow a \in \Theta_0(r) & IH\npp \\
    \end{array}
    \end{smathpar}
    From $H4$ and $IH5$, it suffices to prove:
    \begin{smathpar}
    \begin{array}{cr}
      \forall (a\in\E.A). \hbo{a}{\eff} \Rightarrow a \in \Theta(r) & G\mpp\\
    \end{array}
    \end{smathpar}
    After \emph{intros}, hypotheses:
    \begin{smathpar}
    \begin{array}{cr}
      a\in\E.A & H\npp\\
      \hbo{a}{\eff} & H\npp \\
    \end{array}
    \end{smathpar}
    Goal:
    \begin{smathpar}
    \begin{array}{cr}
      a \in \Theta(r) & G\mpp \\
    \end{array}
    \end{smathpar}
    Inversion on $H7$ leads to two cases:
    \begin{itemize}
      \item SCase \emph{$a$ directly precedes $\eff$ }: Hypothesis:
      \begin{smathpar}
      \begin{array}{cr}
        (\E.\Rvis \cup \E.\Rso)(a,\eff) & H\npp\\
      \end{array}
      \end{smathpar}
      From $H3$ and $H8$, we conclude that $a \in \Theta(r)$.

      \item SCase \emph{$a$ transitively precedes $\eff$}: Hypothesis:
      \begin{smathpar}
      \begin{array}{cr}
        \exists (c \in \E.A). \hbo{a}{c} \wedge (\E.\Rvis \cup
        \E.\Rso)(c,\eff) & H\npp \\
      \end{array}
      \end{smathpar}
      Inverting $H9$:
      \begin{smathpar}
      \begin{array}{cr}
        c \in \E.A & H\npp \\
        \hbo{a}{c} & H\npp \\
        (\E.\Rvis \cup \E.\Rso)(c,\eff) & H\npp \\
      \end{array}
      \end{smathpar}
      From $H3$ and $H12$, we have:
      \begin{smathpar}
      \begin{array}{cr}
        c \in \Theta(r) & H\npp\\
      \end{array}
      \end{smathpar}
      From $IH5$, $H13$ and $H11$, we conclude that $a \in \Theta(r)$.
    \end{itemize}

    \item Cases \rulelabel{HA} and \rulelabel{SA}: Local state is
    unmodified. That is, $\Theta$ remains unchanged. Hence, proof
    follows directly from inductive hypothesis for these cases.

    \item Case \rulelabel{UA}: $\Sigma = \tuplee{s,\sigma} \pll
    \Sigma_1$. Hypotheses:
    \begin{smathpar}
    \begin{array}{cr}
      r \in {\sf ReplID} & H\npp \\
      {\sf Unavailable}(\cv) & H\npp \\
      \E_1;\Theta_1;\tuplee{s,op(v); \sigma}
        \;\xhookrightarrow{\eff,r}\;
      \E;\Theta_2;\tuplee{s,\sigma} & H\npp\\
      \E_1.A \subseteq \Theta_1(r) & H\npp\\
      \dom(\Theta) = \dom(\Theta_2) & H\npp\\
      \forall r'\in \dom(\Theta). \Theta(r') = \Theta_2(r') \cup
      \{\eff\} & H\npp\\
    \end{array}
    \end{smathpar}
    Inductive hypothesis:
    \begin{smathpar}
    \begin{array}{cr}
      \hspace*{-0.5in}\forall (r'\in{\sf dom}(\Theta_1)). \forall (\eff' \in
        \Theta_1(r')). & \\
      \hspace*{0.3in}\forall (a\in\E_1.A). \hbo{a}{\eff'} \Rightarrow a
        \in \Theta_1(r') & IH\npp \\
    \end{array}
    \end{smathpar}
    The goal:
    \begin{smathpar}
    \begin{array}{cr}
      \hspace*{-0.5in}\forall (r'\in{\sf dom}(\Theta)). \forall (\eff' \in
        \Theta(r')). & \\
      \hspace*{0.3in}\forall (a\in\E.A). \hbo{a}{\eff'} \Rightarrow a
        \in \Theta(r') & G\mpp \\
    \end{array}
    \end{smathpar}
    Inverting $H16$:
    \begin{smathpar}
    \begin{array}{cr}
        \E.\EffSoup = \E_1.\EffSoup \cup \{\eff\} & H\npp\\
        \E.\visZ = \Theta_1(r)\times \{\eff\} ~\cup~ \E_1.\visZ & H\npp\\
        \E.\Rso = \E_1.\EffSoup_{{\sf SessID}=s,\,{\sf SeqNo}<i}\times
          \{\eff\} ~\cup~ \E_1.\Rso & H\npp\\
        \Theta_2 = \Theta_1 & H\npp\\
    \end{array}
    \end{smathpar}
    Using $H24$ to rewrite $H19$:
    \begin{smathpar}
    \begin{array}{cr}
      \forall r'\in \dom(\Theta). \Theta(r') = \Theta_1(r') \cup
      \{\eff\} & H\npp\\
    \end{array}
    \end{smathpar}
    Using $H25$ and $IH20$, we can reduce the goal ($G2$) to:
    \begin{smathpar}
    \begin{array}{cr}
      \forall (r'\in{\sf dom}(\Theta)).
      \forall (a\in\E.\EffSoup). \hbo{a}{\eff} \Rightarrow a
        \in \Theta(r') & G\mpp \\
    \end{array}
    \end{smathpar}
    After \emph{intros}, hypotheses:
    \begin{smathpar}
    \begin{array}{cr}
      r' \in \dom(\Theta) & H\npp \\
      a \in \E.\EffSoup & H\npp\\
      \hbo{a}{\eff} & H\npp\\
    \end{array}
    \end{smathpar}
    Goal:
    \begin{smathpar}
    \begin{array}{cr}
      a \in \Theta(r) & G\mpp \\
    \end{array}
    \end{smathpar}
    From $H27$ and $H21$, we know that either $a \in \E_1.\EffSoup$ or
    $a = \eff$.
    \begin{itemize}
      \item If $a \in \E_1.\EffSoup$, then from $H17$, we know that $a
      \in \Theta_1(r)$. However, from $H25$ we know that $\Theta_1(r)
      \subset \Theta(r)$, which lets us conclude that $a \in
      \Theta(r)$.

			\item If $a = \eta$, then $H28$ is $\hbo{\eff}{\eff}$, which is $false$
			since {\sf hbo} is irreflexive. Proof follows from contradition.
    \end{itemize}
  % End of main cases
  \end{itemize}
\hfill \qed
\end{proof}



The proof of the theorem, which we could not include here owing to
space constraints, can be found in our tech report~\cite{techrep}.

% It should be noted that Theorem ~\ref{lem:core-preservation} only
% proves the safety of our system. Proving its liveness requires us to
% axiomatize eventual consistency.
