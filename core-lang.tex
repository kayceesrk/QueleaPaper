\section{Meta-theory}
\label{sec:core-opsem}

In this section, we present the meta-theoretic result that certifies
the soundness of our classification-based contract enforcement
strategy.
% We show that any data store can correctly enforce $\scc$,
% $\ccc$, and $\ecc$ guarantees can satisfy application-level
% consistency constraints of all \name contracts.

Fig.~\ref{sem:oper} summarizes the artifacts needed to state our
meta-theoretic result. We formalize an abstract model of a replicated
data store as a tuple $\E;\Sigma$, where $\E$ denotes the execution
state, and $\Sigma$, the session soup, is the set of concurrent client
sessions interacting with the store. The execution state ($\E$) of a
data store is modeled as a tuple $\Exec$ where $\EffSoup$, called the
\emph{effect soup}, is a set of effects generated during the
execution, and $\visZ,\soZ \subseteq \EffSoup \times \EffSoup$ are
visibility and session order relations, respectively, witnessed over
generated effects. A session is modeled simply as a sequence of
replicated data type operations ($op$), tagged with the consistency
class ($\tau$) of their contracts (as determined by the contract
classification scheme). Our contract language and classification
scheme is independent of store semantics. Accordingly, we abstain from
making any specific assumptions about semantics of the store. However,
we do assume the existence of a reduction relation of form:

\begin{smathpar}
  \auxred{} {\E;\langle (op,\tau);\sigma \rangle \pll \Sigma} {\eff} 
    {\E';\langle \sigma \rangle \pll \Sigma'}
\end{smathpar}

\noindent The relation captures store's progress in execution (from
$\E$ to $\E'$) due to the application of an operation ($op$) from one
of the sessions in $\Sigma$ to a replica, generating a new effect
($\eff$).

Note that an execution state ($\E$) provides interpretations for
primitive relations (eg: $\visZ$) that occur free in contract
formulas, and also fixes the domain of quantification to set of all
effects ($\EffSoup$) observed during the execution. Therefore, $\E$ is
a potential model for any contract formula ($\cv$). If $\E$ is indeed
a model of $\cv$ (i.e., $\E \models \cv$), we say that the store
enforced the contract $\cv$ in execution $\E$.

% Core Operational Semantics
% --------------------------
\twocolumn
\begin{figure*}[t]
\textbf{RDT Specification Language}\\
\begin{minipage}{\columnwidth}
\begin{smathpar}
\stretcharraybig
\begin{array}{lclcl}
{\delta} & \in & \mathtt{Replicated Datatype}& &\\
{v} & \in & \mathtt{Value} & & \\
{e} & \in & \mathtt{Expression} & &\\
{op} & \in & \mathtt{Operation}& & \\
\end{array}
\end{smathpar}
\end{minipage}
\begin{minipage}{\columnwidth}
\begin{smathpar}
\stretcharraybig
\begin{array}{lclcl}
\Ops & \in & \mathtt{Operation Def.}   & \coloneqq & op \mapsto e \\
\cv & \in & {\sf Contract} & &\\
\Ctrts & \in & \mathtt{Operation Contract} & \coloneqq & op \mapsto \cv \\
   &   & \mathtt{RDT Specification} & \coloneqq & (\delta,\Ops,\Ctrts)\\
\end{array}
\end{smathpar}
\end{minipage}

\vspace{5mm}
\begin{minipage}{\columnwidth}
\textbf{System Model}\\
\begin{smathpar}
\stretcharraybig
\begin{array}{lclcl}
\multicolumn{5}{l}{
{s} \in \SessID \qquad
{i} \in \EffID \qquad
{r} \in \ReplID } \\
\eff & \in & \mathtt{Effect} & \coloneqq &  (s,~i,~op,~v)\\
\EffSoup & \in & \mathtt{EffSoup}	  & \coloneqq & \set{\eff} \\
\visZ, \soZ, \sameobjZ &	\in & \mathtt{Relations} & \coloneqq & \EffSoup \times \EffSoup \\
{\E} 		& \in & \mathtt{ExecState}  & \coloneqq & \Exec \\
\end{array}
\end{smathpar}
\end{minipage}
\begin{minipage}{\columnwidth}
\begin{smathpar}
\stretcharraybig
\begin{array}{lclcl}
\Theta  & \in & \mathtt{Store}      & \coloneqq & r \mapsto \set{\eff} \\
{\tau}		& \in & \mathtt{Consistency Class} 	& \coloneqq & {\sf ec},{\sf cc},{\sf sc} \\
{\sigma} 	& \in & \mathtt{Session} 					 	& \coloneqq & \cdot \ALT \langle op,\tau \rangle; \sigma \\
\Sigma 		& \in & \mathtt{Session\;Soup}   	 	& \coloneqq &
      \langle s, i, \sigma \rangle \pll \Sigma \ALT \emptyset \\
					&			&	\mathtt{Config}		  			 	& \coloneqq & \E;\Theta;\Sigma \\
\end{array}
\end{smathpar}
\end{minipage}

\vspace{5mm}
\textbf{Auxiliary Definitions}\\
\begin{minipage}{\columnwidth}
\begin{smathpar}
\stretcharraybig
\begin{array}{lcl}
\operZ(s,~i,~op,~v) & = & op \\
\end{array}
\end{smathpar}
\end{minipage}
\begin{minipage}{\columnwidth}
\begin{smathpar}
\stretcharraybig
\begin{array}{lcl}
\ctxtFn(s,~i,~op,~v) & = & (op,~v) \\
\end{array}
\end{smathpar}
\end{minipage}


\vspace{5mm}
\textbf{Auxiliary Reductions} \;
  \fbox{\(\auxred{\Theta} {(\E,\langle s,i,op \rangle)} {r} {(\E',\eff)}\)}\\

\begin{minipage}{\textwidth}
\rulelabel{Oper}
\begin{smathpar}
\stretcharraybig
\begin{array}{l}
\RuleTwo
{
r \in \dom(\Theta) \qquad
ctxt = {\ctxtFn}^{*}(\Theta(r)) \qquad
\Ops(op)(ctxt) {\rdtredsto}^{*} v' \qquad
\eff = (s,~i,~op,~v')\\
\EffSoup' = \{\eff\} \cup \EffSoup \qquad
\visZ' = \Theta(r)\times\{\eff\} \cup \visZ \qquad
\soZ' = \EffSoup_{({\sf SessID}=s,\,{\sf SeqNo}<i)}\times\{\eff\} \cup \soZ
\qquad
\sameobjZ' = \EffSoup' \times \EffSoup'
}
{
  \auxred {\Theta} {((\EffSoup,\visZ,\soZ,\sameobjZ), \langle s,i,op \rangle}
  {r} {((\EffSoup',\visZ',\soZ',\sameobjZ'),\eff)}
}
\end{array}
\end{smathpar}
\end{minipage}


\vspace{5mm}
\textbf{Operational Semantics} \;
  \fbox{\(\E;\Theta;\Sigma \;\xrightarrow{\eff}\; \E';\Theta';\Sigma'\)}\\

\begin{minipage}{\columnwidth}
\rulelabel{EffVis}
\begin{smathpar}
\stretcharraybig
\begin{array}{l}
\RuleTwo
{
  \eff \in \EffSoup \quad \eff \notin \Theta(r) \\
  \E.\visZ^{-1}(\eff) \cup \E.\soZ^{-1}(\eff) \subseteq \Theta(r)
  \qquad \Theta' = \Theta[r \mapsto \{\eff\} \cup \Theta(r)]
}
{
  \E;\Theta;\Sigma \;\xrightarrow{\eff}\; \E;\Theta';\Sigma
}
\end{array}
\end{smathpar}
\end{minipage}
\begin{minipage}{\columnwidth}
\rulelabel{EC}
\begin{smathpar}
\stretcharraybig
\begin{array}{l}
\RuleTwo
{
  \tau = {\sf EventuallyConsistent} \\
  \auxred{\Theta} {(\E,\langle s,i,op \rangle)} {r} {(\E',\eff)}
}
{
  \E;\Theta;\langle s,i,\langle op,\tau \rangle\; \sigma \rangle \pll \Sigma \;\xrightarrow{\eff}\;
  \E';\Theta; \langle s,i+1,\sigma \rangle\pll \Sigma
}
\end{array}
\end{smathpar}
\end{minipage}

\vspace{5mm}
\begin{minipage}{\columnwidth}
\rulelabel{CC}
\begin{smathpar}
\stretcharraybig
\begin{array}{l}
\RuleTwo
{
  \\
  \tau = {\sf CausallyConsistent} \\
  \auxred{\Theta} {(\E,\langle s,i,op \rangle)} {r} {(\E',\eff)} \qquad
  \E'.\soZ^{-1}(\eff) \subseteq \Theta(r)
}
{\E;\Theta;\langle s,i,\langle op,\tau \rangle; \sigma \rangle \pll \Sigma \;\xrightarrow{\eff}\;
\E';\Theta;\langle s,i+1,\sigma \rangle \pll \Sigma }
\end{array}
\end{smathpar}
\end{minipage}
\begin{minipage}{\columnwidth}
\rulelabel{SC}
\begin{smathpar}
\stretcharraybig
\begin{array}{l}
\RuleTwo
{
  \tau = {\sf StronglyConsistent} \\
  \auxred{\Theta} {(\E,\langle s,i,op \rangle)} {r}
  {(\E',\eff)} \qquad \E.A \subseteq \Theta(r) \\
  \dom(\Theta') = \dom(\Theta) \qquad
  \forall r'\in \dom(\Theta'). \Theta'(r') = \Theta(r') \cup \{\eff\}
}
{
  \E;\Theta;\langle s,i,\langle op,\tau \rangle; \sigma \rangle \pll \Sigma
  \;\xrightarrow{\eff}\; \E;\Theta';\langle s,i+1,\sigma \rangle \pll \Sigma
}
\end{array}
\end{smathpar}
\end{minipage}


\caption{Operational semantics of a replicated data store.}
\label{sem:oper}
\end{figure*}
\onecolumn




We now state the soundness of contract enforcement as following:

\begin{theorem}[Soundness of Contract Enforcement]
\label{lem:core-preservation}
Let $\cv$ be the contract of a replicated data type operation $op$,
and let $\tau$ denote the consistency class of $\cv$
as determined by the contract classification scheme. Forall execution
states $\E$, $\E'$ such that
$\auxred{} {\E;\langle (op,\tau);\,\sigma \rangle \pll \Sigma} {\eff}
 {\E';\langle \sigma \rangle \pll \Sigma}$
if $\E' \models [\eff/\cureff]\,\cv(\tau)$, then $\E' \models [\eff/\cureff]\,\cv$
\end{theorem}

The theorem states that if a data store correctly enforces $\scc$,
$\ccc$, and $\ecc$ guarantees in all executions, then contract
classification scheme can be used to satisfy any \name contract on
that data store. The proof of the theorem has been included in our
tech report~\cite{techrep}. Tech report also contains operational
semantics of our implementation of the data store, which concretizes
the reduction relation ($\xrightarrow{}$), along with the proof
that it correctly enforces all \name contracts.
