\section{Soundness}
\label{sec:core-opsem}

In this section, we present a soundness result that certifies
the correctness of our classification-based contract enforcement
strategy.
% We show that any data store can correctly enforce $\scc$,
% $\ccc$, and $\ecc$ guarantees can satisfy application-level
% consistency constraints of all \name contracts.

Fig.~\ref{sem:oper} summarizes the artifacts needed. We formalize an
abstract model of a replicated data store as a tuple $\E;\Sigma$, where $\E$
denotes the execution state, and $\Sigma$, the session soup, is the set of
concurrent client sessions interacting with the store. The execution state
($\E$) of a data store is modeled as a tuple $\Exec$ where $\EffSoup$,
called the \emph{effect soup}, is a set of effects generated during the
execution, and $\visZ,\soZ \subseteq \EffSoup \times \EffSoup$ are
visibility and session order relations, respectively, witnessed over
generated effects. A session is modeled simply as a sequence of replicated
data type operations ($op$), tagged with the consistency class ($\tau$) of
their contracts (as determined by the contract classification scheme). Our
contract language and classification scheme is independent of store
semantics. Accordingly, we avoid making any specific assumptions about the
semantics of the store. However, we do assume the existence of a reduction
relation of the form:

\begin{smathpar}
  \auxred{} {\E;\langle (op,\tau);\sigma \rangle \pll \Sigma} {\eff} 
    {\E';\langle \sigma \rangle \pll \Sigma'}
\end{smathpar}

\noindent The relation captures the store's execution progress (from
$\E$ to $\E'$) due to the application of an operation ($op$) from one
of the sessions in $\Sigma$ to a replica, generating a new effect
($\eff$).

Note that an execution state ($\E$) provides interpretations for
primitive relations (eg: $\visZ$) that occur free in contract
formulas, and also fixes the domain of quantification to set of all
effects ($\EffSoup$) observed during the execution. Therefore, $\E$ is
a potential model for any contract formula ($\cv$). If $\E$ is indeed
a model of $\cv$ (i.e., $\E \models \cv$), we say that the store
enforced the contract $\cv$ in execution $\E$.

% Core Operational Semantics
% --------------------------
\begin{figure}
\begin{smathpar}
\begin{array}{lclcl}
\multicolumn{5}{l}{
  {\eff} \in \mathtt{Effect} \qquad
  {op} \in \mathtt{Operation} \qquad
  {\cv} \in \mathtt{Contract}
}\\
\multicolumn{5}{l}{
  \set{\eff} \in \mathtt{Set\; of\; Effects} \qquad
  \tau \in \mathtt{Consistency\; Class}
}\\
%{op} & \in & {\sf Operation}& & \\
%\cv & \in & {\sf Contract} & &\\
\cv(\tau) & \in & \mathtt{Store\; Contract\; of \; \tau} & \coloneqq & \scc,
  \ccc, \ecc\\
%{a_s} & \in & \mathtt{Action} & \coloneqq &  \\
{\sigma} & \in & \mathtt{Session} & \coloneqq & \cdot \ALT (op,\tau); \sigma \\
%\eff & \in & \mathtt{Effect} & \coloneqq &  (s,~i,~op,~v)\\
\EffSoup & \in & \mathtt{EffSoup}	  & \coloneqq & \set{\eff} \\
\visZ, \soZ, &	\in & \mathtt{Relations} & \coloneqq &
  \set{\eff}\times\set{\eff} \\
%	\sameobjZ		&     &  & \\
{\E} 		& \in & \mathtt{ExecState}  & \coloneqq & (\EffSoup,\visZ,\soZ)\\
\Sigma 	& \in & \mathtt{Session\;Soup}   & \coloneqq &
  \langle s,{\sigma} \rangle \pll \Sigma \ALT \emptyset \\
				&			&	\mathtt{Store}		  & \coloneqq & \E;\Sigma \\
\end{array}
\end{smathpar}

\caption{Abstract model of a data store}
\label{sem:oper}
\end{figure}




\begin{theorem}[Soundness of Contract Enforcement]
\label{lem:core-preservation}
Let $\cv$ be the contract of a replicated data type operation $op$,
and let $\tau$ denote the consistency class of $\cv$
as determined by the contract classification scheme. Forall execution
states $\E$, $\E'$ such that
$\auxred{} {\E;\langle (op,\tau);\,\sigma \rangle \pll \Sigma} {\eff}
 {\E';\langle \sigma \rangle \pll \Sigma}$
if $\E' \models [\eff/\cureff]\,\cv(\tau)$, then $\E' \models [\eff/\cureff]\,\cv$
\end{theorem}

The theorem states that if a data store correctly enforces $\scc$,
$\ccc$, and $\ecc$ guarantees in all executions, then contract
classification scheme can be used to satisfy any \name contract on
that data store. The proof of the theorem is available in the supplementary
material, which also contains an operational
semantics of our implementation of the data store that concretizes
the reduction relation ($\xrightarrow{}$), along with the proof
that it correctly enforces all \name contracts.
