\begin{theorem}[Well-formedness preservation]
  \label{lem:core-preservation}
  If $\E$ is well-formed and $\E; \Theta; \Sigma \goesto \E'; \Theta';
  \Sigma'$, then $\E'$ is well-formed.
  \end{theorem}
\begin{proof}
  Let $\E = (\EffSoup,\visZ,\soZ)$ and $\E'=(\EffSoup',\Rvis',\Rso')$.
  We prove by case analysis on the derivation $\E; \Theta; \Sigma
    \goesto \E'; \Theta'; \Sigma'$. Cases:
    \begin{itemize}
      \item Case \rulelabel{Comm}: The rule does not change the
      execution state (i.e., $\E' = \E$); so, well-formedness is
      trivially preserved.

      \item Case \rulelabel{HA}: $\Sigma = \tuplee{s,~op(v); \sigma} \pll
      \Sigma_0$, and $\Sigma' = \tuplee{s,\sigma} \pll \Sigma_0$,
      where $\Sigma_0$ represents reset of the session soup. From the
      definition of well-formedness, we have:
      \begin{smathpar}
      \begin{array}{cr}
        \forall a\in\EffSoup. ~\msentails{E}{\Ctrts(a)} & H\npp\\
      \end{array}
      \end{smathpar}
      Let us call $\Ctrts(op)$ as $\cv$. We have the following hypotheses:
      \begin{smathpar}
      \begin{array}{cr}
        {\sf HighlyAvailable}(\cv) & H\npp \\
        \E;\Theta;\tuplee{s,op(v); \sigma}
          \;\xhookrightarrow{\eff,r}\;
        \E';\Theta';\tuplee{s,\sigma} & H\npp\\
      \end{array}
      \end{smathpar}
      The last hypothesis indicates that $op(v)$ is executed on
      replica $r$ yielding effect $\eff$. Inversion on $H2$ gives:
      \begin{smathpar}
      \begin{array}{cr}
        \EffSoup' = \EffSoup \cup \{\eff\} & H\npp\\
        \visZ' = \Theta(r)\times\eff ~\cup~ \visZ & H\npp\\
        \Rso' = \EffSoup_{{\sf SessID}=s,\,{\sf SeqNo}<i}\times\eff ~\cup~ \Rso & H\npp\\
      \end{array}
      \end{smathpar}
      From Hypotheses $H0$ and $H3$, we deduce that in order to prove:
      \begin{smathpar}
      \begin{array}{cr}
        \forall a\in\EffSoup'. ~\msentails{E'}{\Ctrts(a)} & \\
      \end{array}
      \end{smathpar}
      It suffices to prove the following goal:
      \begin{smathpar}
      \begin{array}{cr}
        \msentails{E'}{\cv} & G\mpp\\
      \end{array}
      \end{smathpar}
      By destructing $H1$, we get:
      \begin{smathpar}
      \begin{array}{cr}
        \hasTyp{}{\De{\cvc \Rightarrow \cv}} & H\npp\\
      \end{array}
      \end{smathpar}
      Applying (\emph{eapply}) lemma~\ref{lem:1} using $H6$ with
      $E:=E'$, we deduce that to prove $G0$, it is sufficient to
      prove:
      \begin{smathpar}
      \begin{array}{cr}
        \msentails{E'}{\De{\cvc}} & G\mpp\\
      \end{array}
      \end{smathpar}
      $G1$ becomes our new goal. Expanding the definition of $\cvc$:
      \begin{smathpar}
      \begin{array}{cr}
        \msentails{E'}{\De{\forall a,b. \hbo{a}{b} \wedge \vis{b}{\eff} \Rightarrow
          \vis{a}{\eff}}} & \\
      \end{array}
      \end{smathpar}
      By definition $\De{\forall a.\psi} = \forall a.
      \De{\psi}$ and  $\De{\psi_1 \Rightarrow \psi_2} = \De{\psi_1}
      \Rightarrow \De{\psi_2}$. Further, the model $\E'$ defines
      $\EffSoup'$ as the universe of values. Therefore, the goal can
      be rewritten:
      \begin{smathpar}
      \begin{array}{cr}
        \forall (a,b\in\EffSoup').
        \msentails{E'}{\De{\hbo{a}{b} \wedge
        \vis{b}{\eff}} \Rightarrow
          \De{\vis{a}{\eff}}} & G\mpp\\
      \end{array}
      \end{smathpar}
      New hypotheses after \emph{intros}:
      \begin{smathpar}
      \begin{array}{cr}
        a \in \EffSoup' & H\npp\\
        b \in \EffSoup' & H\npp\\
      \end{array}
      \end{smathpar}
      And new goal:
      \begin{smathpar}
      \begin{array}{cr}
        \msentails{E'}{\De{\hbo{a}{b} \wedge
        \vis{b}{\eff}} \Rightarrow
          \De{\vis{a}{\eff}}} & G\mpp\\
      \end{array}
      \end{smathpar}
      From the definition of \emph{models} relation of
      first-order logic (i.e., semantics of first-order logic),
      \emph{inversion} on $G3$ adds following hypothesis:
      \begin{smathpar}
      \begin{array}{cr}
        \msentails{\E'}{\De{ \hbo{a}{b} \wedge
        \vis{b}{\eff}}}  & H\npp\\
      \end{array}
      \end{smathpar}
      New goal following the inversion:
      \begin{smathpar}
      \begin{array}{cr}
          \msentails{\E'}{\De{\vis{a}{\eff}}}& G\mpp\\
      \end{array}
      \end{smathpar}
      From the definition of $\De{.}$, $H7$ expands to:
      \begin{smathpar}
      \begin{array}{cr}
          (\msentails{E'}{\hbo{a}{b}}) \wedge
          (\msentails{E'}{\Rvis'(b,\eff)}) & H\npp\\
      \end{array}
      \end{smathpar}
      \E' provides interpretation for $\Rvis$ as $\Rvis'$, and $\Rso$
      as $\Rso'$. Therefore, inverting $H10$ gives:
      \begin{smathpar}
      \begin{array}{cr}
          {{\sf hbo'}(a,b)} \wedge \Rvis'(b,\eff) & H\npp\\
      \end{array}
      \end{smathpar}
      In $H11$, ${\sf hbo'} = (\Rso' \cup \Rvis')^{+}$. Similarly,
      inverting the goal $G4$:
      \begin{smathpar}
      \begin{array}{cr}
        \Rvis'(a,\eff) & G\mpp\\
      \end{array}
      \end{smathpar}
      Inversion on $H11$:
      \begin{smathpar}
      \begin{array}{cr}
          {{\sf hbo'}(a,b)} & H\npp\\
          \Rvis'(b,\eff) & H\npp\\
      \end{array}
      \end{smathpar}
      Since $\eff$ is unique, from $H13$ and $H4$ we have the following:
      \begin{smathpar}
      \begin{array}{cr}
            b\in\Theta(r)& H\npp\\
      \end{array}
      \end{smathpar}
      Applying lemma~\ref{lem:local-state-is-cc} using $H12$ and $H14$
      with $\Rvis=\Rvis'$ and $\Rso=\Rso'$, we derive:
      \begin{smathpar}
      \begin{array}{cr}
        a \in \Theta(r) & H\npp\\
      \end{array}
      \end{smathpar}
      Now, from $H4$ and $H15$, we deduce:
      \begin{smathpar}
      \begin{array}{cr}
        (a,\eff) \in \Rvis'& \\
      \end{array}
      \end{smathpar}
      which is what needs to be proven ($G5$).
      \item Case \rulelabel{SA}: $\Sigma = \tuplee{s,~op(v); \sigma} \pll
      \Sigma_0$, and $\Sigma' = \tuplee{s,\sigma} \pll \Sigma_0$. From the definition of
      well-formedness, we have:
      \begin{smathpar}
      \begin{array}{cr}
        \forall a\in\EffSoup. ~\msentails{E}{\Ctrts(a)} & H\npp\\
      \end{array}
      \end{smathpar}
      Let us call $\Ctrts(op)$ as $\cv$. We have the following hypotheses:
      \begin{smathpar}
      \begin{array}{cr}
        {\sf StickyAvailable}(\cv) & H\npp \\
        \E;\Theta;\tuplee{s,op(v); \sigma}
          \;\xhookrightarrow{\eff,r}\;
        \E';\Theta';\tuplee{s,\sigma} & H\npp\\
        \E'.\soZ^{-1}(\eff) \subseteq \Theta(r) & H\npp\\
      \end{array}
      \end{smathpar}
      $H18$ indicates that $op(v)$ is executed on
      replica $r$ yielding effect $\eff$. Inversion on $H18$ gives:
      \begin{smathpar}
      \begin{array}{cr}
        \EffSoup' = \EffSoup \cup \{\eff\} & H\npp\\
        \visZ' = \Theta(r)\times\eff ~\cup~ \visZ & H\npp\\
        \Rso' = \EffSoup_{{\sf SessID}=s,\,{\sf SeqNo}<i}\times\eff ~\cup~ \Rso & H\npp\\
      \end{array}
      \end{smathpar}
      From $H19$ and $H22$, we have:
      \begin{smathpar}
      \begin{array}{cr}
        \EffSoup_{{\sf SessID}=s,\,{\sf SeqNo}<i} \in \Theta(r) &
        H\npp \\
      \end{array}
      \end{smathpar}
      From $H16$ and $H20$, we know that in order to prove:
      \begin{smathpar}
      \begin{array}{cr}
        \forall a\in\EffSoup'. ~\msentails{E'}{\Ctrts(a)} & \\
      \end{array}
      \end{smathpar}
      It suffices to prove the following goal:
      \begin{smathpar}
      \begin{array}{cr}
        \msentails{E'}{\cv} & G\mpp\\
      \end{array}
      \end{smathpar}
      By destructing $H17$, we get:
      \begin{smathpar}
      \begin{array}{cr}
        \hasTyp{}{\De{\ccc \Rightarrow \cv}} & H\npp\\
      \end{array}
      \end{smathpar}
      Applying (\emph{eapply}) lemma~\ref{lem:1} using $H24$ with
      $E:=E'$, we deduce that to prove $G0$, it is sufficient to
      prove:
      \begin{smathpar}
      \begin{array}{cr}
        \msentails{E'}{\De{\ccc}} & G\mpp\\
      \end{array}
      \end{smathpar}
      $G7$ becomes our new goal. Expanding the definition of $\ccc$:
      \begin{smathpar}
      \begin{array}{cr}
        \msentails{E'}{\De{\forall a. \hbo{a}{\eff} \Rightarrow
        \vis{a}{\eff}}} & \\
      \end{array}
      \end{smathpar}
      By definition $\De{\forall a.\psi} = \forall a.
      \De{\psi}$ and  $\De{\psi_1 \Rightarrow \psi_2} = \De{\psi_1}
      \Rightarrow \De{\psi_2}$. Further, the model $\E'$ defines
      $\EffSoup'$ as the universe of values. Therefore, the goal can
      be rewritten:
      \begin{smathpar}
      \begin{array}{cr}
        \forall (a,b\in\EffSoup').
        \msentails{E'}{\De{\hbo{a}{\eff}} \Rightarrow
          \De{\vis{a}{\eff}}} & G\mpp\\
      \end{array}
      \end{smathpar}
      New hypotheses after \emph{intros}:
      \begin{smathpar}
      \begin{array}{cr}
        a \in \EffSoup' & H\npp\\
      \end{array}
      \end{smathpar}
      And new goal:
      \begin{smathpar}
      \begin{array}{cr}
        \msentails{E'}{\De{\hbo{a}{\eff}} \Rightarrow
          \De{\vis{a}{\eff}}} & G\mpp\\
      \end{array}
      \end{smathpar}
      From the definition of \emph{models} relation of
      first-order logic (i.e., semantics of first-order logic),
      \emph{inversion} on $G9$ adds following hypothesis:
      \begin{smathpar}
      \begin{array}{cr}
        \msentails{\E'}{\De{ \hbo{a}{\eff}}}  & H\npp\\
      \end{array}
      \end{smathpar}
      New goal following the inversion:
      \begin{smathpar}
      \begin{array}{cr}
          \msentails{\E'}{\De{\vis{a}{\eff}}}& G\mpp\\
      \end{array}
      \end{smathpar}
      By inversion on $H26$:
      \begin{smathpar}
      \begin{array}{cr}
          {{\sf hbo'}(a,\eff)}& H\npp\\
      \end{array}
      \end{smathpar}
      where ${\sf hbo'} = (\Rso' \cup \Rvis')^{+}$. Similarly,
      inverting the goal $G4$, we get the current goal:
      \begin{smathpar}
      \begin{array}{cr}
        \Rvis'(a,\eff) & G\mpp\\
      \end{array}
      \end{smathpar}
      Inverting $H27$, we get two cases:
      \begin{itemize}
        \item SCase 1: Hypotheses:
        \begin{smathpar}
        \begin{array}{cr}
          (\Rso' \cup \Rvis')(a,\eff) & H\npp\\
        \end{array}
        \end{smathpar}
        Inversion on $H28$ leads to two subcases. In one case, we
        assume $\Rvis'(a,\eff)$ and try to prove the goal $G11$. The
        proof for this case mimics the proof for Case \rulelabel{HA}.
        Alternatively, in second case, we assume:
        \begin{smathpar}
        \begin{array}{cr}
          \Rso'(a,\eff) & H\npp\\
        \end{array}
        \end{smathpar}
        and prove $G11$. From $H29$ and $H19$, we infer:
        \begin{smathpar}
        \begin{array}{cr}
          a \in \Theta(r) & H\npp\\
        \end{array}
        \end{smathpar}
        Now, from $H30$ and $H21$ we know:
        \begin{smathpar}
        \begin{array}{cr}
          (a,\eff) \in \Rvis' & \\
        \end{array}
        \end{smathpar}
        from which the goal ($G11$) follows.

        \item SCase 2: Hypotheses (after abbreviating the occurance
        of$(\Rso' \cup \Rvis')^{+}$ as {\sf hbo'}):
        \begin{smathpar}
        \begin{array}{cr}
          \exists(c\in\EffSoup').{\sf hbo'}(a,c) \wedge (\Rso' \cup
          \Rvis')(c,\eff) & H\npp\\
        \end{array}
        \end{smathpar}
        Inverting $H31$, followed by expanding $\Rso' \cup \Rvis'$:
        \begin{smathpar}
        \begin{array}{cr}
          c\in\EffSoup' & H\npp\\
           {\sf hbo'}(a,c) \wedge (\Rso'(c,\eff) \vee \Rvis'(c,\eff))& H\npp\\
        \end{array}
        \end{smathpar}
        Inverting the disjunction in $H33$, we get two cases:
        \begin{itemize}
          \item SSCase R: Hypothesis is
          \begin{smathpar}
          \begin{array}{cr}
             {\sf hbo'}(a,c) \wedge \Rvis'(c,\eff)& H\npp\\
          \end{array}
          \end{smathpar}
          Observe that hypothesis $H34$ and current goal ($G11$) are
          same as hypothesis $H11$ and goal ($G5$) in Case
          \rulelabel{HA}. The proof for this SSCase is also the same.

          \item SSCase L: Hypothesis is
          \begin{smathpar}
          \begin{array}{cr}
             {\sf hbo'}(a,c) \wedge \Rso'(c,\eff)& H\npp\\
          \end{array}
          \end{smathpar}
          Inverting $H35$:
          \begin{smathpar}
          \begin{array}{cr}
             {\sf hbo'}(a,c)& H\npp\\
             \Rso'(c,\eff)& H\npp\\
          \end{array}
          \end{smathpar}
          From $H37$ and $H19$, we infer:
          \begin{smathpar}
          \begin{array}{cr}
            c \in \Theta(r) & H\npp \\
          \end{array}
          \end{smathpar}
          Applying lemma~\ref{lem:local-state-is-cc} using $H36$ and
          $H38$ with $\Rso=\Rso'$ and $\Rvis=\Rvis'$, we get:
          \begin{smathpar}
          \begin{array}{cr}
            a \in \Theta(r) & H\npp\\
          \end{array}
          \end{smathpar}
          Proof follows from $H39$ and $H21$.
        % End of SSCase
        \end{itemize}
      % End of SCases
      \end{itemize}

      \item Case \rulelabel{UA}: $\Sigma = \tuplee{s,~op(v); \sigma} \pll
      \Sigma_0$, and $\Sigma' = \tuplee{s,\sigma} \pll \Sigma_0$. From the definition of
      well-formedness, we have:
      \begin{smathpar}
      \begin{array}{cr}
        \forall a\in\EffSoup. ~\msentails{E}{\Ctrts(a)} & H\npp\\
      \end{array}
      \end{smathpar}
      Let us call $\Ctrts(op)$ as $\cv$. We have the following hypotheses:
      \begin{smathpar}
      \begin{array}{cr}
        {\sf Unavailable}(\cv) & H\npp \\
        \E;\Theta;\tuplee{s,op(v); \sigma}
          \;\xhookrightarrow{\eff,r}\;
        \E';\Theta_0;\tuplee{s,\sigma} & H\npp\\
        \E.A \subseteq \Theta(r) & H\npp\\
        \forall r'\in \dom(\Theta_0). \Theta'(r') = \Theta_0(r') \cup
          \{\eta\} & H\npp\\
      \end{array}
      \end{smathpar}
      Inversion on $H42$ gives:
      \begin{smathpar}
      \begin{array}{cr}
        \EffSoup' = \EffSoup \cup \{\eff\} & H\npp\\
        \visZ' = \Theta(r)\times\{\eff\} ~\cup~ \visZ & H\npp\\
        \Rso' = \EffSoup_{{\sf SessID}=s,\,{\sf SeqNo}<i}\times\eff ~\cup~ \Rso & H\npp\\
      \end{array}
      \end{smathpar}
      From Hypotheses $H0$ and $H3$, we deduce that in order to prove:
      \begin{smathpar}
      \begin{array}{cr}
        \forall a\in\EffSoup'. ~\msentails{E'}{\Ctrts(a)} & \\
      \end{array}
      \end{smathpar}
      It suffices to prove the following goal:
      \begin{smathpar}
      \begin{array}{cr}
        \msentails{E'}{\cv} & G\mpp\\
      \end{array}
      \end{smathpar}
      By destructing $H1$, we get:
      \begin{smathpar}
      \begin{array}{cr}
        \hasTyp{}{\De{\scc \Rightarrow \cv}} & H\npp\\
      \end{array}
      \end{smathpar}
      Applying (\emph{eapply}) lemma~\ref{lem:1} using $H6$ with
      $E:=E'$, we deduce that to prove $G0$, it is sufficient to
      prove:
      \begin{smathpar}
      \begin{array}{cr}
        \msentails{E'}{\De{\scc}} & G\mpp\\
      \end{array}
      \end{smathpar}
      $G1$ becomes our new goal. Expanding the definition of $\scc$:
      \begin{smathpar}
      \begin{array}{cr}
        \msentails{E'}{\forall a. \De{\phi_1 \wedge \phi_2 \wedge \phi_3}} &
        G\mpp\\
      \end{array}
      \end{smathpar}
      where $\phi_1, \phi_2$ and $\phi_3$ are defined as:
      \begin{smathpar}
      \begin{array}{cr}
        \phi_1 =  (\hbo{a}{\eff} \vee \hbo{\eff}{a} \vee a =
        \eff) & H\npp\\
        \phi_2 = (\hbo{a}{\eff} \Rightarrow \vis{a}{\eff}) & H\npp\\
        \phi_3 = (\hbo{\eff}{a} \Rightarrow \vis{\eff}{a})& H\npp\\
      \end{array}
      \end{smathpar}
      After inversion followed by \emph{intros}, new hypothesis:
      \begin{smathpar}
      \begin{array}{cr}
        a\in\EffSoup' & H\npp\\
      \end{array}
      \end{smathpar}
      New goal:
      \begin{smathpar}
      \begin{array}{cr}
        \msentails{E'}{\De{\phi_1 \wedge \phi_2 \wedge \phi_3}} & G\mpp\\
      \end{array}
      \end{smathpar}
      Inverting \emph{models} relation in $G15$ followed by inverting
      conjunction, we get three goals:
      \begin{smathpar}
      \begin{array}{cr}
        \msentails{E'}{\De{\phi_1 }} & G\mpp\\
        \msentails{E'}{\De{\phi_2 }} & G\mpp\\
        \msentails{E'}{\De{\phi_3 }} & G\mpp\\
      \end{array}
      \end{smathpar}
      Substituting definitions of $\phi_1$, $\phi_2$ and $\phi_3$ in
      $G16-18$ and expanding:
      \begin{smathpar}
      \begin{array}{cr}
         {\sf hbo'}(a,\eff) \vee {\sf hbo'}(\eff,a)  \vee a = \eff & G\mpp\\
         {\sf hbo'}(a,\eff) \Rightarrow \Rvis'(a,\eff) & G\mpp\\
         {\sf hbo'}(\eff,a) \Rightarrow \Rvis'(\eff,a) & G\mpp\\
      \end{array}
      \end{smathpar}
      From $H45$ and $H52$, we know that either $a\in\EffSoup$ or
      $a=\eff$. When $a=\eff$:
      \begin{itemize}
        \item $G19$ follows trivially.
        \item Since ${\sf hbo'}$ is irreflexive, ${\sf
        hbo'}(\eff,\eff) = false$. Therefore, $G20$ and $G21$ are valid vacuosly.
      \end{itemize}
      When $a \in A$:
      \begin{itemize}
        \item From $H43$ we know that $a \in \Theta(r)$. Using $H46$, we
        derive:
        \begin{smathpar}
        \begin{array}{cr}
          \Rvis'(a,\eff) & H\npp\\
        \end{array}
        \end{smathpar}
        which proves $G20$

        \item We prove $G19$ by proving:
        \begin{smathpar}
        \begin{array}{cr}
          {\sf hbo'}(a,\eff)& G\mpp\\
        \end{array}
        \end{smathpar}
        From previous case, we know that:
        \begin{smathpar}
        \begin{array}{cr}
          \Rvis'(a,\eff) & H\npp\\
        \end{array}
        \end{smathpar}
        Introducting disjunction:
        \begin{smathpar}
        \begin{array}{cr}
          (\Rvis' \cup \Rso')(a,\eff) & H\npp\\
        \end{array}
        \end{smathpar}
        Now, since ${\sf hbo'} = (\Rvis' \cup \Rso')^{+}$, proof
        follows from $H54$.

        %% AXIOMS ON VIS, SO & HBO NEED TO BE PROVEN FOR VIS', SO' &
        %% HBO'.
        \item From previous case, we know that ${\sf hbo'}(a,\eff)$.
        Since ${\sf hbo'}$ is asymmetric, ${\sf hbo'}(\eff,a) =
        false$. Hence, $G21$ follows vacuosly.
      \end{itemize}
    % End of main cases
    \end{itemize}
  \hfill \qed
\end{proof}

\begin{lemma}[Encoding Implication]
  \label{lem:1}
  Forall contracts $\psi_1$ and $\psi_2$, if $\De{\psi_1 \Rightarrow
  \psi_2}$ is a valid formula in first-order logic (i.e.,
  $\hasTyp{\cdot} {\De{\psi_1 \Rightarrow \psi_2}}$), then any
  execution state ($\E$) that is a model of $\De{\psi_1}$ is also the
  model of $\De{\psi_2}$ (i.e., $(\msentails{\E}{\De{\psi_1}})
  \Rightarrow (\msentails{\E}{\De{\psi_2}})$).
\end{lemma}
\begin{proof}
  By definition, $\De{\psi_1 \Rightarrow \psi_2} = \De{\psi_1}
  \Rightarrow \De{\psi_2}$. Since $\De{\phi_1}$ and $\De{\phi_2}$ are
  formulas in first-order logic, the proof now follows directly from
  the soundness result of first-order logic.
\hfill \qed
\end{proof}

\begin{lemma}[Local State is Causally Consistent]
  \label{lem:local-state-is-cc}
  In a replicated data store $\E; \Theta; \Sigma$, local state of
  every replica $r$ in the store (i.e., $\Theta(r)$) is causally
  consistent. That is:\vspace*{0.04in}\\
  $\forall (r\in{\sf dom}(\Theta)). \forall (\eff \in \Theta(r)). \\
  \hspace*{0.8in}\forall (a\in\E.A). \hbo{a}{\eff} \Rightarrow a \in \Theta(r)$\vspace*{0.04in}\\
  where ${\sf hbo} = (\E.\Rvis \cup \E.\Rso)^{+}$
\end{lemma}
\begin{proof}
  By induction on $(\E; \Theta; \Sigma)$. Cases:
  \begin{itemize}
    \item Case \rulelabel{Comm}: Hypotheses:
    \begin{smathpar}
    \begin{array}{cr}
      r \in {\sf ReplID} & H\npp \\
      \eff \in A & H\npp \\
      \eff \notin \Theta(r)& H\npp \\
      \E.\visZ^{-1}(\eff) \cup \E.\soZ^{-1}(\eff) \subseteq \Theta(r)
      & H\npp \\
      \Theta = \Theta_0[r \mapsto \{\eff\} \cup \Theta_0(r)] & H\npp \\
    \end{array}
    \end{smathpar}
    Inductive Hypothesis:
    \begin{smathpar}
    \begin{array}{cr}
      \hspace*{-0.5in}\forall (r\in{\sf dom}(\Theta_0)). \forall (\eff \in
        \Theta_0(r)). & \\
      \hspace*{0.3in}\forall (a\in\E.A). \hbo{a}{\eff} \Rightarrow a \in \Theta_0(r) & IH\npp \\
    \end{array}
    \end{smathpar}
    From $H4$ and $IH5$, it suffices to prove:
    \begin{smathpar}
    \begin{array}{cr}
      \forall (a\in\E.A). \hbo{a}{\eff} \Rightarrow a \in \Theta(r) & G\mpp\\
    \end{array}
    \end{smathpar}
    After \emph{intros}, hypotheses:
    \begin{smathpar}
    \begin{array}{cr}
      a\in\E.A & H\npp\\
      \hbo{a}{\eff} & H\npp \\
    \end{array}
    \end{smathpar}
    Goal:
    \begin{smathpar}
    \begin{array}{cr}
      a \in \Theta(r) & G\mpp \\
    \end{array}
    \end{smathpar}
    Inversion on $H7$ leads to two cases:
    \begin{itemize}
      \item SCase \emph{$a$ directly precedes $\eff$ }: Hypothesis:
      \begin{smathpar}
      \begin{array}{cr}
        (\E.\Rvis \cup \E.\Rso)(a,\eff) & H\npp\\
      \end{array}
      \end{smathpar}
      From $H3$ and $H8$, we conclude that $a \in \Theta(r)$.

      \item SCase \emph{$a$ transitively precedes $\eff$}: Hypothesis:
      \begin{smathpar}
      \begin{array}{cr}
        \exists (c \in \E.A). \hbo{a}{c} \wedge (\E.\Rvis \cup
        \E.\Rso)(c,\eff) & H\npp \\
      \end{array}
      \end{smathpar}
      Inverting $H9$:
      \begin{smathpar}
      \begin{array}{cr}
        c \in \E.A & H\npp \\
        \hbo{a}{c} & H\npp \\
        (\E.\Rvis \cup \E.\Rso)(c,\eff) & H\npp \\
      \end{array}
      \end{smathpar}
      From $H3$ and $H12$, we have:
      \begin{smathpar}
      \begin{array}{cr}
        c \in \Theta(r) & H\npp\\
      \end{array}
      \end{smathpar}
      From $IH5$, $H13$ and $H11$, we conclude that $a \in \Theta(r)$.
    \end{itemize}

    \item Cases \rulelabel{HA} and \rulelabel{SA}: Local state is
    unmodified. That is, $\Theta$ remains unchanged. Hence, proof
    follows directly from inductive hypothesis for these cases.

    \item Case \rulelabel{UA}: $\Sigma = \tuplee{s,\sigma} \pll
    \Sigma_1$. Hypotheses:
    \begin{smathpar}
    \begin{array}{cr}
      r \in {\sf ReplID} & H\npp \\
      {\sf Unavailable}(\cv) & H\npp \\
      \E_1;\Theta_1;\tuplee{s,op(v); \sigma}
        \;\xhookrightarrow{\eff,r}\;
      \E;\Theta_2;\tuplee{s,\sigma} & H\npp\\
      \E_1.A \subseteq \Theta_1(r) & H\npp\\
      \dom(\Theta) = \dom(\Theta_2) & H\npp\\
      \forall r'\in \dom(\Theta). \Theta(r') = \Theta_2(r') \cup
      \{\eff\} & H\npp\\
    \end{array}
    \end{smathpar}
    Inductive hypothesis:
    \begin{smathpar}
    \begin{array}{cr}
      \hspace*{-0.5in}\forall (r'\in{\sf dom}(\Theta_1)). \forall (\eff' \in
        \Theta_1(r')). & \\
      \hspace*{0.3in}\forall (a\in\E_1.A). \hbo{a}{\eff'} \Rightarrow a
        \in \Theta_1(r') & IH\npp \\
    \end{array}
    \end{smathpar}
    The goal:
    \begin{smathpar}
    \begin{array}{cr}
      \hspace*{-0.5in}\forall (r'\in{\sf dom}(\Theta)). \forall (\eff' \in
        \Theta(r')). & \\
      \hspace*{0.3in}\forall (a\in\E.A). \hbo{a}{\eff'} \Rightarrow a
        \in \Theta(r') & G\mpp \\
    \end{array}
    \end{smathpar}
    Inverting $H16$:
    \begin{smathpar}
    \begin{array}{cr}
        \E.\EffSoup = \E_1.\EffSoup \cup \{\eff\} & H\npp\\
        \E.\visZ = \Theta_1(r)\times \{\eff\} ~\cup~ \E_1.\visZ & H\npp\\
        \E.\Rso = \E_1.\EffSoup_{{\sf SessID}=s,\,{\sf SeqNo}<i}\times
          \{\eff\} ~\cup~ \E_1.\Rso & H\npp\\
        \Theta_2 = \Theta_1 & H\npp\\
    \end{array}
    \end{smathpar}
    Using $H24$ to rewrite $H19$:
    \begin{smathpar}
    \begin{array}{cr}
      \forall r'\in \dom(\Theta). \Theta(r') = \Theta_1(r') \cup
      \{\eff\} & H\npp\\
    \end{array}
    \end{smathpar}
    Using $H25$ and $IH20$, we can reduce the goal ($G2$) to:
    \begin{smathpar}
    \begin{array}{cr}
      \forall (r'\in{\sf dom}(\Theta)).
      \forall (a\in\E.\EffSoup). \hbo{a}{\eff} \Rightarrow a
        \in \Theta(r') & G\mpp \\
    \end{array}
    \end{smathpar}
    After \emph{intros}, hypotheses:
    \begin{smathpar}
    \begin{array}{cr}
      r' \in \dom(\Theta) & H\npp \\
      a \in \E.\EffSoup & H\npp\\
      \hbo{a}{\eff} & H\npp\\
    \end{array}
    \end{smathpar}
    Goal:
    \begin{smathpar}
    \begin{array}{cr}
      a \in \Theta(r) & G\mpp \\
    \end{array}
    \end{smathpar}
    From $H27$ and $H21$, we know that either $a \in \E_1.\EffSoup$ or
    $a = \eff$.
    \begin{itemize}
      \item If $a \in \E_1.\EffSoup$, then from $H17$, we know that $a
      \in \Theta_1(r)$. However, from $H25$ we know that $\Theta_1(r)
      \subset \Theta(r)$, which lets us conclude that $a \in
      \Theta(r)$.

			\item If $a = \eta$, then $H28$ is $\hbo{\eff}{\eff}$, which is $false$
			since {\sf hbo} is irreflexive. Proof follows from contradition.
    \end{itemize}
  % End of main cases
  \end{itemize}
\hfill \qed
\end{proof}

