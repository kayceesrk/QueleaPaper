\section{Transaction Contracts}

While contracts on individual operations offer the programmer object-level
declarative reasoning, real-world scenarios often involve operations that span
multiple objects. In order to address this problem, several recent
systems~\cite{COPS,BurckhardtESOP,BailisHAT} have proposed a variety of
transactions, with varying semantics, in order to compose operations on
multiple objects. However, given that classical transaction models such as
serializability~\cite{} and snapshot isolation~\cite{} require inter-replica
coordination, these systems espouse \emph{coordination-free transactions} that
remain available under network partitions, but only provide weaker isolation
guarantees. Coordination-free transactions have intricate consistency semantics
and widely varying runtime overheads. As with operation-level consistency, the
onus is on the programmer to pick the correct transaction kind, the choice of
which is complicated by the consistency requirement of the operations it
composes.

\subsection{Syntax Extension}

\name automates the choice of assigning the correct and most efficient
transaction isolation level. Similar to contracts on individual operations, the
programmer associates contracts with transactions, declaratively expressing its
consistency specification. We extend the contract language with a new term
under quantifier-free propositions ${\small \txnZ}~S_1~S_2$, where $S_1$ and
$S_2$ are sets of effects, and introduce a new primitive equivalence relation
$\small \sametxnZ$ that holds for effects from the same transaction. $\small
\txn{a,b}{c,d}$ is just syntactic sugar for $\small \sametxn{a}{b} ~\wedge~
\sametxn{c}{d} ~\wedge~ \neg\sametxn{a}{c}$, where $a$ and $b$ considered to
belong to the \emph{current} transaction. We assume that operations not part of
any transaction to belong to their own unique transaction. While the
transactions may have varying isolation guarantees, we make the standard
assumption that all transactions provide atomicity. Hence, we include the
following axiom in $\Delta$: $\small \forall a,b,c.~\txn{a}{b,c} ~\wedge~
\sameobj{a}{b} ~\wedge~ \sameobj{b}{c} ~\wedge~ \vis{b}{a} \Rightarrow
\vis{c}{a}$.

\subsection{Transactional Bank Account}

In order to illustrate the utility of declarative reasoning for transactions,
let us extend our running bank account example with use two accounts (objects)
-- current ($c$) and savings ($s$). Each account provides operations
\cf{withdraw}, \cf{deposit} and \cf{getBalance}, with the same contracts as
defined previously. We consider two transactions -- \cf{save(amt)}, which
transfers \cf{amt} from current to savings, and \cf{totalBalance} returns
the sum of the balances of individual accounts. The pseudo code for the
transactions is given below:

\noindent \begin{minipage}[t]{0.5\columnwidth}
\begin{codepython}
def save(amt):
  atomically psi1:
    b = c.withdraw(amt)
    if (b): s.deposit(amt)
\end{codepython}
\end{minipage}
\begin{minipage}[t]{0.5\columnwidth}
\begin{codepython}
def totalBalance():
  atomically psi2:
    b1 = c.getBalance()
    b2 = s.getBalance()
    return b1 + b2
\end{codepython}
\end{minipage}

\noindent where $\cv_1$ and $\cv_2$ are the contracts on the corresponding
transactions. Our goal is to ensure that \cf{totalBalance} returns the result
obtained from a consistent snapshot of the object states.

While making both transactions serializable would ensure correctness,
distributed stores rarely offer serializable transactions. Moreover,
serializability is unavailable and hinders scalability. As we will see, these
transactions can be satisfied with much weaker isolation guarantees. Despite
the atomicity offered by the transaction, anomalies are still possible.
