\section{Appendix}
\label{sec:appendix}

\subsection{Soundness of Contract Classification}

We restate theorem ~\ref{thm:classification-sound} and provide its proof
below:

\begin{theorem}[Soundness of Contract Enforcement]
\label{lem:classification-sound}
Let $\cv$ be a well-formed contract of a replicated data type
operation $\mathit{op}$, and let $\tau$ denote the consistency class
of $\cv$ as determined by the contract classification scheme. For all
well-formed execution states $\E$, $\E'$ such that $\auxred{}
{\E,\langle op,\tau \rangle;\sigma \pll \Sigma} {\eff} {\E', \sigma
\pll \Sigma}$, if $\E' \models \cv_\tau[\eff/\cureff]$, then $\E'
\models \cv[\eff/\cureff]$
\end{theorem}
\begin{proof}
  Hypothesis:
  \begin{smathpar}
  \begin{array}{cr}
    \auxred{} {\E,\langle op,\tau \rangle;\sigma \pll \Sigma} {\eff}
    {\E', \sigma \pll \Sigma} & H\npp\\
    \E' \models \cv_\tau[\eff/\cureff] & H\npp\\
  \end{array}
  \end{smathpar}
  Since $\tau$ is the contract class of $\cv$, by inversion, we have
  $\cv \le \cv_\tau$. By the definition of $\le$ relation:
  \begin{smathpar}
  \begin{array}{cr}
    \Delta \vdash \forall \cureff.\,\cv_\tau \Rightarrow \cv & H\npp\\
  \end{array}
  \end{smathpar}
   Since $\eff$ denotes new effect, it is a fresh variable that does
   not occur free in $\Delta$. From $H2$, after instantiating bound
   $\cureff$ with $\eff$, we have:
  \begin{smathpar}
  \begin{array}{cr}
    \Delta \vdash \cv_\tau[\eff/\cureff] \Rightarrow \cv[\eff/\cureff]
      & H\npp\\
  \end{array}
  \end{smathpar}
  Due to the soundness of natural deduction for first-order logic,
  $H3$ implies that for all models $\mathcal{M}$ such that
  $\mathcal{M} \models \Delta$, if $\mathcal{M} \models
  \cv_\tau[\eff/\cureff]$ then $\mathcal{M} \models
  \cv[\eff/\cureff]$. Since $\E'$ is well-formed, we have:
  \begin{smathpar}
  \begin{array}{cr}
    \E' \models \Delta & H\npp\\
  \end{array}
  \end{smathpar}
  Proof follows from $H1$ and $H4$.
  \hfill \qed
\end{proof}

\subsection{Operational Semantics}
\renewcommand{\auxred}[4]{#1 \vdash #2 \;\xhookrightarrow{#3}\; #4 }

We now describe operational semantics of a data store that implements strong,
causal and eventual consistency guarantees. The semantics also serves as an
elegant representation of our implementation of \name.

Figure ~\ref{sem:oper} presents our semantics as concretization of the rewrite
relation ($\xrightarrow{}$) over execution state. Since we now have a concrete
store, we extend our system model with $\Theta$, a representation of the store
as a map from replicas to their local states. The local state of a replica $r$
(i.e., $\Theta(r)$) is a set of effects that are currently visible at $r$.  An
operation $op$ performed at replica $r$ can only witness the set of effects
($\Theta(r)$) visible at $r$. For the sake of clarity, we only consider a
single replicated object of well-defined type (for eg: a replicated object of
type \cf{BankAccount}) in our formalization.  Our semantics are parametric over
the specification of this replicated data type. Figure~\ref{sem:oper}
formalizes replicated data type (RDT) specification as tuple
$(\delta,\Ops,\Ctrts)$, where $\delta$ is the data type, $\Ops$ maps labels
($op$) of operations on $\delta$ to their definitions, while $\Ctrts$ maps them
to their consistency contracts ($\cv$). The definition of an operation is
expected to be a lambda expression, although we do not enforce this in our
formalization. For technical reasons, we tag each session with a session
identifier ($s$) and the sequence number ($i$) of the next operation in the
session.

The state of an operational execution ($\E$) is a tuple $\Exec$ where
$\EffSoup$ is a set of effects, and $\visZ,\soZ, \sameobjZ \subseteq \EffSoup
\times \EffSoup$ are \emph{visibility}, \emph{session order}, and \emph{same
object} relations over effects, respectively. We concretize an effect ($\eff$)
as a tuple $(s,i,op,v)$, which records the fact that $i^{th}$ action in session
with {\sf SessID} $s$, which is an operation $op$ on the replicated object, has
been successfully executed on some replica yielding a return value $v$. Note
that the combination of $s$ and $i$ uniquely identifies the effect. Session
order relation ($\soZ$) relates effects generated by the same session. An
effect $\eff = (s,i,op,v)$ is said to precede another effect $\eff' =
(s',i',op',v')$ in session order if and only if $s'=s$ and $i'\ge i$. Since we
only consider one replicated object in our formalization, the $\sameobjZ$
relation relates every pair of effects in the effect soup ($\EffSoup$). An
effect generated at a replica becomes visible at rest of the replicas
eventually.  If we denote the effect generated by the operation $op$ as
$\eff_{op}$, then $\Theta(r) \times \{\eff_{op}\} ~\subseteq~ \visZ$. Often, in
our formalization, we use $\visZ$ and $\soZ$ binary relations to obtain a set
of effects visible to a given effect $\eff$, or set of effects that precede a
given effect $\eff$ in the session order. As a syntactic convenience, whenever
$R$ is a binary relation, we write $R(\eff)$ to denote the set of all $\eff'$
such that $(\eff,\eff') \in R$.  Conversely, we write $R^{-1}(\eff)$ to denote
the set of all $\eff'$ such that $(\eff',\eff) \in R$.

Basic gurantee provided by the store is causal visibility, which is
captured by the rule $\rulelabel{EffVis}$ as a condition for an effect
to be visible at a replica. The rule makes an effect ($\eff$) visible
at a replica $r$ only after all the effects that causally precede
$\eff$ are made visible at $r$.  It is important to note that that
enforcing causal visibility does not require any inter-replica
coordination. Any eventually consistent store can provide causal
visibility while being eventually consistent.  Therefore, we do not lose
any generality by assuming that the store provides causal visiblity.

% Semantics
% ---------
\begin{figure}
\begin{smathpar}
\begin{array}{lclcl}
\multicolumn{5}{l}{
  {\eff} \in \mathtt{Effect} \qquad
  {op} \in \mathtt{Operation} \qquad
  {\cv} \in \mathtt{Contract}
}\\
\multicolumn{5}{l}{
  \set{\eff} \in \mathtt{Set\; of\; Effects} \qquad
  \tau \in \mathtt{Consistency\; Class}
}\\
%{op} & \in & {\sf Operation}& & \\
%\cv & \in & {\sf Contract} & &\\
\cv(\tau) & \in & \mathtt{Store\; Contract\; of \; \tau} & \coloneqq & \scc,
  \ccc, \ecc\\
%{a_s} & \in & \mathtt{Action} & \coloneqq &  \\
{\sigma} & \in & \mathtt{Session} & \coloneqq & \cdot \ALT (op,\tau); \sigma \\
%\eff & \in & \mathtt{Effect} & \coloneqq &  (s,~i,~op,~v)\\
\EffSoup & \in & \mathtt{EffSoup}	  & \coloneqq & \set{\eff} \\
\visZ, \soZ, &	\in & \mathtt{Relations} & \coloneqq &
  \set{\eff}\times\set{\eff} \\
%	\sameobjZ		&     &  & \\
{\E} 		& \in & \mathtt{ExecState}  & \coloneqq & (\EffSoup,\visZ,\soZ)\\
\Sigma 	& \in & \mathtt{Session\;Soup}   & \coloneqq &
  \langle s,{\sigma} \rangle \pll \Sigma \ALT \emptyset \\
				&			&	\mathtt{Store}		  & \coloneqq & \E;\Sigma \\
\end{array}
\end{smathpar}

\caption{Abstract model of a data store}
\label{sem:oper}
\end{figure}




Rule $\rulelabel{Oper}$ is an auxiliary reduction of the
form:\vspace{-1.7mm}
\begin{smathpar}
\auxred{\Theta} {(\E,\langle s,i,op \rangle)} {r} {(\E',\eff)}
\vspace{-1.7mm}
\end{smathpar}
\noindent Under the store configuration $\Theta$, the rule captures
the progress in execution (from $\E$ to $\E'$) due to the application
of operation $op$ to replica $r$ resulting in a new effect $\eff$.
The rule first constructs a \emph{context} for the application from
the local state ($\Theta(r)$) of the replica, by
projecting\footnote{{\textsf{ctxt*}} is auxiliary function
\textsf{ctxt} extended straightforwardly to set of effects} relevant
information from effects in $\Theta(r)$. It then substitutes the
definition ($\Ops(op)$) of the operation for its label ($op$), and
relies on the reduction relation ($\rdtredsto$) of the server-side
language to reduce the application $\Ops(op)(ctxt)$ to a value
$v'$.  Subsequently, the the attributes of execution state, namely
$\EffSoup$, $\visZ$, $\soZ$, and $\sameobjZ$ are extended to
incorporate the new effect ($\eff$).

If the operation $op$ is {\sf EventuallyConsistent}, we simply apply
the operation to any replica $r$. Since the store provides causal
visibility, eventually consistent operations are satisfiable under any
replica. If the operation is {\sf CausallyConsistent}, the operation
can be applied to a replica $r$ only if it already contains the
effects of all the previous operations from the same session. This
guarantee can be satisfied by applying all operations from the same
session to the same logical copy of the database.  If such a logical
copy is untenable, then the operation might block. Since the store is
assumed to converge eventually, the blocked causally consistent
operation guaranteed to unblock eventually.

A {\sf StronglyConsistent} operation expects sequential consistency.
That is, universe of all effects ($\EffSoup$) in an execution ($\E$)
must be partitionable into a set of effects that \emph{happened
before} $\eff$ and another set that \emph{happened after} $\eff$,
where $\eff$ is the effect generated by an strongly consistent
operation. The rule $\rulelabel{UA}$ enforces this sequencing in two
steps; firstly, it insists that the the strongly consistent operation
($op$) witness effects of all operations executed so far by requiring
the global set of effects $\EffSoup$ to be a subset of local state
($\Theta(r)$) of the replica ($r$) executing $op$. Secondly, the rule
requires the effect ($\eff$) generated by $op$ to be added to the
local state of every other replica in the store, so that further
operations on these replicas can witness the effect of $op$. Since
both these steps require global coordination among replicas, the store
is \emph{unavailable} during the time it is executing $op$.
% However, it must be noted that executing strongly consistent
% operation does not entail the convergence of local states of all
% replicas; it simply means that there is one replica ($r$) whose
% local state is the upper bound of local states of all replicas, and
% there exists a set containing atleast one effect ($\eff$), which is
% their lower bound.

\subsection{Soundness of Operational Semantics}

Our main meta-theoretic result guarantees the soundness of our operational
semantics in enforcing $\ecc$, $\ccc$, and $\scc$ consistency guarantess by
ensuring that every reduction step preserves correctness of operational
execution.

\begin{theorem}[Soundness preservation]
\label{lem:core-preservation}
Let $\E$ be a well-formed operational execution state, and $\tau$ denote a
contract class.  If:
\begin{smathpar}
\E; \Theta; \langle s,i,\langle op,\tau \rangle;
\sigma \rangle \pll \Sigma \xhookrightarrow{\eff} \E'; \Theta';
\langle s,i,\sigma \rangle \Sigma'
\end{smathpar}
then (\romannumeral 1) $\E'$ is well-formed, and (\romannumeral 2)
$\E' \models \cv_{\tau}[\eff/\cureff]$
\end{theorem}
