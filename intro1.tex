\section{Introduction}

Many real-world web services --- such as those built and maintained by
Amazon, Facebook, Google, Twitter, etc. --- replicate application state and
logic across multiple \emph{replicas} within and across data
centers. Replication is intended not only to improve application throughput
and reduce user-perceived latency, but also to tolerate partial failures
without compromising overall service availability. Traditionally programmers
have relied on \emph{strong consistency} guarantees such as
linerarizability~\cite{Herlihy1990} or
serializability~\cite{Serializability} in order to build correct
applications. While strong consistency is an easily stated property, it
masks the reality underlying large-scale distributed systems with respect to
non-uniform latency (i.e., availability) and network
partitions~\cite{Brewer2000,Gilbert}.  Indeed, modern web services, which
aim to provide an "always on" experience, overwhelmingly favour availability
and partition tolerance over strong consistency. Several \emph{weak
  consistency} models such as eventual consistency, causal consistency,
session guarantees, and timeline consistency have consequently been
proposed.

Under weak consistency, the developer needs to be aware of concurrent
conflicting updates, and has to pay careful attention to avoid unwanted
inconsistencies (e.g., negative balances in a bank account, or having an
item appear in a shopping cart after it has been
removed~\cite{Dynamo}). Oftentimes, the inconsistency leaks from the
application and is witnessed by the user.  Ultimately, the developer must
decide the consistency level appropriate for a particular operation; this is
understandably an error-prone process requiring intricate knowledge of both
the application as well as the semantics and implementation of the
underlying data store.  More often than not, only informal descriptions are
available~\cite{}.  Picking the correct consistency level is critical not
only for correctness but also for scalability of the application. While
choosing a weaker consistency level than required may introduce program
errors and anomalies, choosing a stronger one than necessary level can
negatively impact program scalability.


%--KC--
% The choice of correct consistency level is critical to the correctness and
% scalability of the application. Choosing a weaker consistency level than
% required introduces anomalies, while choosing a stronger than necessary level
% affects program scalability.
%\KC{Do we include the LWW-Register number here to add emphasis to the fact that
%choosing the correct consistency level is essential for program scalability}.


Moreover, weak consistency hinders compositional reasoning about programs.
While an application might be naturally expressed in terms of
well-understood and expressive data types such as maps, trees, queues, or
graphs, geo-distributed stores typically only provide a minimal set of data
types with in-built conflict resolution strategies such as last-writer-wins
(LWW) registers, counters, and sets~\cite{Cassandra,DynamoDB}.  Furthermor,
while traditional database systems enable composability through
transactions, existing geo-distributed stores lack unrestricted
transactional access to the data.  Working with geo-distributed stores thus
requires application state to be suitably coerced to work within the
capabilities of the store.

To address these issues, we describe \name, a declarative programming model
and implementation for eventually consistent stores. The key novelty of
\name\ is an expressive \emph{contract language} to declare and verify
fine-grained application-level consistency properties.  The programmer uses
the contract language to axiomatically describe the set of legal executions
allowed over the replicated data type.  Contracts are constructed using
primitive consistency relations such as \emph{visibility} and \emph{session
  order} along with standard logical and relational operators.  A
\emph{contract enforcement system} automatically maps each operation to a
particular consistency level available on the store.  In addition, the
system statically reports an error if the given application-level
consistency property cannot be satisfied by the store.  

The paper makes the following contributions:

\begin{itemize}

\item We introduce \name, a language for declaratively describing eventually
  consistent programs that manipulate replicated data types. Contracts are
  used to specify fine-grained application-level consistency properties, and
  are analyzed to assign the most efficient and sound store consistency
  level to the corresponding operation.

\item \name supports coordination-free transactions over arbitrary
  datatypes.  We extend our contract language to express fine-grained
  transaction isolation guarantees, and utilize the contract enforcement
  system to automatically assign the correct and most efficient isolation
  level for a transaction.

\item Metatheory that certifies the soundness of our contract enforcement
  system, and ensures that an operation is only executed if the required
  conditions on consistency are met is also given.

\item \name is implemented as a shim layer over (and importantly without
  modifying) Cassandra, a well-known general-purpose data store.  An
  extensive experimental evaluation over a collection of real-world
  applications, including a Twitter-like micro-blogging site and an
  eBay-like auction site illustrates the practicality of our approach.

\end{itemize}

The rest of the paper is organized as follows. Section~\ref{sec:motivation}
motivates \name and introduces the language through a series of examples.
Section~\ref{sec:lang} describes the contract language, and
Section~\ref{sec:opsem} presents an operational semantics.
Section~\ref{sec:txns} introduces transaction contracts and
classification. Section~\ref{sec:impl} describes the implementation and goes
into detail about the optimizations needed to make the system practical.
Section~\ref{sec:results} discusses experimental evaluation.
Section~\ref{sec:related} and~\ref{sec:concl} present related work and
conclusions.
